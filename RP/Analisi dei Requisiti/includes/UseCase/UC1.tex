\newpage
\subsection{Caso d'uso UC1: Main pre-autenticazione}
\label{UC1}
\begin{figure}[ht]
	\centering
	\includegraphics[scale=0.45]{UML/UC1.png}
	\caption{UC1: Main pre-autenticazione}
\end{figure}

\renewcommand*{\arraystretch}{1.6}
\begin{longtable}{ l | p{11cm}}
	\hline
	\rowcolor{Gray}
	\multicolumn{2}{c}{UC1 - Main pre-autenticazione} \\
	\hline
	\textbf{Attori} & Utente non autenticato  \\
	\textbf{Descrizione} & L'attore si trova nella schermata principale dell'applicazione ed accede alle funzionalità a lui disponibili: la registrazione, il login, il recupero password, la ricerca API \\
	\textbf{Pre-Condizioni} & L'attore ha avviato l'applicazione web e non si è ancora autenticato. La piattaforma mostra le pagine preposte agli utenti non autenticati \\
	\textbf{Post-Condizioni} & L'applicazione ha eseguito le richieste dell'attore \\
	\textbf{Scenario Principale} & 
	\begin{enumerate*}[label=(\arabic*.),itemjoin={\newline}]
		\item L'attore può registrarsi all'applicazione (UC3)
		\item L'attore può effettuare il login all'applicazione (UC4)
		\item L'attore può recuperare la propria password (UC5)
		\item L'attore può effettuare una ricerca sulle API presenti nell'applicazione (UC6)
	\end{enumerate*}\\
	\textbf{Scenari Alternativi} & 
	\begin{enumerate*}[label=(\arabic*.),itemjoin={\newline}]
		\item L'attore può scegliere di effettuare il login tramite API Market (UC4.1)
		\item L'attore può scegliere di effettuare il login tramite Facebook (UC4.2)
		\item L'attore può scegliere di effettuare il login tramite Twitter (UC4.3)
		\item L'attore può scegliere di effettuare il login tramite LinkedIn (UC4.4)
		\item L'attore può scegliere di effettuare il login tramite Google+ (UC4.5)
	\end{enumerate*}\\
\end{longtable}