\newpage

\section{Consuntivi di periodo}

In questa porzione del documento verrà analizzato lo scostamento tra il preventivo calcolato e le ore effettive impiegate, sia in termini puramente di tempistiche, che di preventivo vero e proprio. Si potrà avere, dunque, uno scostamento in \textbf{Positivo}, qualora le ore impiegate siano minori delle ore preventivate, o in \textbf{Pari}, se il preventivo risulta corretto, oppure in \textbf{Negativo}, se la pianificazione per un determinato periodo è stata sottostimata.

\subsection{Analisi dei Requisiti}

Di seguito, si analizzano le ore realmente impiegate per ciascun componente, con l'eventuale scostamento in positivo o negativo indicato tra parentesi.

\begin{table}[H]
	\begin{center}
		\begin{tabular}{|c|c|c|c|c|c|c|c|}
			\hline
			\textbf{Nome} & \multicolumn{6}{c|}{\textbf{Ore per ruolo}} & \textbf{Ore totali} \\\cline{2-7}
			& \textbf{Resp} & \textbf{Amm} & \textbf{An} & \textbf{Proj} & \textbf{Prog} & \textbf{Ver} & \\
			\hline
			\MC			&		&		&	16	&		&		&	15 (-1)	&	31 (-1)	\\
			\hline
			\AN			&		&	3	&	6	&	 	&		&	22 (-2)	& 	31 (-1)	\\
			\hline
			\DAN		&		&	3	&	29 	&		&		&		&	32	\\
			\hline
			\AS			&	20	&	 	&	12 (+1) 	&		&	 	& 		&	32 (+1)	\\
			\hline
			\NS 		&	18 (-1)	&	3	&	11	&		&		& 		&	32 (-1)	\\
			\hline
			\DS			& 		&	2	&	5	&		&		&	24 (-2)	&	31 (-2)	\\
			\hline
		\end{tabular}
	\end{center}
	\caption{Scostamento ore per componente, \AdR}
\end{table}

Di seguito, è presente, invece, la tabella relativa al calcolo orario complessivo. Come si evince, l'impegno di alcune ore in meno rispetto a quanto preventivato, per quanto concerne specialmente l'attività di Verifica, ha prodotto uno scostamento in positivo del preventivo di € 80.

\begin{table}[H]
	\begin{center}
		\begin{tabular}{|c|c|c|}
			\hline
			\textbf{Ruolo}	& \textbf{Ore}	& \textbf{Scostamento preventivo} \\
			\hline
			\Res	&	38 (-1)	&  +30 € \\
			\hline
			\Amm	&	11	&  0 \\
			\hline
			\Ana	&	79 (+1)	&  -25 € \\
			\hline
			\Ver	&	61 (-5)	&  +75 €\\
			\hline
			\textbf{Totale} & \textbf{189 (-2)} & \textbf{+80 €}\\
			\hline
		\end{tabular}
	\end{center}
	\caption{Scostamento ore totale, periodo di Analisi dei Requisiti}
\end{table}

\subsection{Analisi dei Requisiti Dettagliata}

Di seguito, vengono analizzate le ore realmente impiegate per ciascun componente, in relazione al periodo di Analisi dei Requisiti Dettagliata.

\begin{table}[H]
	\begin{center}
		\begin{tabular}{|c|c|c|c|c|c|c|c|}
			\hline
			\textbf{Nome} & \multicolumn{6}{c|}{\textbf{Ore per ruolo}} & \textbf{Ore totali} \\\cline{2-7}
			& \textbf{Resp} & \textbf{Amm} & \textbf{An} & \textbf{Proj} & \textbf{Prog} & \textbf{Ver} & \\
			\hline
			\MC			&		&	1	&	 	&		&		&	2 	&	 3	\\
			\hline
			\AN			&		&		&	3 (-1) 	&	 	&		&	 	& 	 3 (-1)	\\
			\hline
			\DAN		&		&	 	&	2 	&		&		&		&	 2	\\
			\hline
			\AS			&	2	&	 	&	  	&		&	 	& 		&	 2	\\
			\hline
			\NS 		&	2	&		&	 	&		&		& 		&	 2	\\
			\hline
			\DS			& 		&	 	&	 	&		&		&	3 	&	 3	\\
			\hline
		\end{tabular}
	\end{center}
	\caption{Scostamento ore per componente, \ARD}
\end{table}

Nella seguente tabella, relativa al calcolo orario complessivo, si può constatare come la differenza, seppur minima, per l'attività di Analisi, ha prodotto un ulteriore scostamento in positivo di € 25 per questo periodo.

\begin{table}[H]
	\begin{center}
		\begin{tabular}{|c|c|c|}
			\hline
			\textbf{Ruolo}	& \textbf{Ore}	& \textbf{Scostamento preventivo} \\
			\hline
			\Res	&   4 	&  0  \\
			\hline
			\Amm	&   1	&  0	\\
			\hline
			\Ana	&   5 (-1)	&  +25 €	\\
			\hline
			\Ver	&   5	&  0	\\
			\hline
			\textbf{Totale} & \textbf{15 (-1)} & \textbf{+25 €}\\
			\hline
		\end{tabular}
	\end{center}
	\caption{Scostamento ore totale, periodo di Analisi dei Requisiti Dettagliata}
\end{table}

\subsection{Progettazione Architetturale}

Di seguito, si analizzano le ore realmente impiegate per ciascun componente, in relazione al periodo di Progettazione Architetturale.

\begin{table}[H]
	\begin{center}
		\begin{tabular}{|c|c|c|c|c|c|c|c|}
			\hline
			\textbf{Nome} & \multicolumn{6}{c|}{\textbf{Ore per ruolo}} & \textbf{Ore totali} \\\cline{2-7}
			& \textbf{Resp} & \textbf{Amm} & \textbf{An} & \textbf{Proj} & \textbf{Prog} & \textbf{Ver} & \\
			\hline
			\MC			&		&	3	&		&	31	&		&		&   34	\\
			\hline
			\AN			&	3	&		&		&	31	&		&		& 	34	\\
			\hline
			\DAN		&		&	2	&		&	17	&		&	14	&	33	\\
			\hline
			\AS			&		&	 	&	 	&	14 (+2)	&	 	& 	19	&	33 (+2)	\\
			\hline
			\NS 		&		&		&		&	14 (+2)	&		& 	19	&	33 (+2)	\\
			\hline
			\DS			& 	2	&		&		&	31	&		&		&	33	\\
			\hline
		\end{tabular}
	\end{center}
	\caption{Scostamento ore per componente, Progettazione Architetturale}
\end{table}

In maniera differente dalle attività precedenti, a causa di alcune incomprensioni progettuali e necessità di studi maggiormente approfonditi, si è verificato uno scostamento in negativo di € 100, a causa di un maggior numero di ore impiegato durante l'attività di Progettazione.

\begin{table}[H]
	\begin{center}
		\begin{tabular}{|c|c|c|}
			\hline
			\textbf{Ruolo}	& \textbf{Ore}	& \textbf{Scostamento preventivo} \\
			\hline
			\Res	&	5	&		\\
			\hline
			\Amm	&	5	&		\\
			\hline
			\Prog   &	138 (+4)  &	-100€	\\
			\hline
			\Ver	&	52	&	\\
			\hline
			\textbf{Totale} & \textbf{200 (+4)} & \textbf{-100 €}\\
			\hline
		\end{tabular}
	\end{center}
	\caption{Scostamento ore totale, periodo di Progettazione Architetturale}
\end{table}

\subsection{Consuntivo parziale}

Analizzando i dati ottenuti, si ottiene un consuntivo parziale per il periodo rendicontato. Esso è consultabile nella tabella sottostante:

\begin{table}[H]
	\begin{center}
		\begin{tabular}{|c|c|c|}
			\hline
			\textbf{Periodo analizzato}	& \textbf{Scostamento ore}	& \textbf{Scostamento preventivo} \\
			\hline
			\AdR	&	-5	&	+ 80 €	\\
			\hline
			\ARD	&	-1	&	+ 25€	\\
			\hline
			\PA   &		+4  &	- 100 €	\\
			\hline
			\textbf{Totale} & \textbf{-2} & \textbf{+5 €} \\
			\hline
		\end{tabular}
	\end{center}
	\caption{Scostamento ore parziale}
\end{table}

Il conteggio finale, in data attuale, mostra come le previsioni, nonostante gli imprevisti nel periodo di Progettazione Architetturale, risultano in linea con le aspettative. Il saldo parziale risulta, infatti, in attivo di € 5: questo dato, seppur di moderata entità, ha l'importanza di mostrare come le stime effettuate siano corrette (per il periodo rendicontato) ai fini del calcolo di un preventivo.