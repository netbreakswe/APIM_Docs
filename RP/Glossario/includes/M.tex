\newglossaryentry{machine learning}
{
	name=machine learning,
	description={Letteralmente significa apprendimento automatico, ovvero l'abilità dei calcolatori nell'apprendere qualcosa senza essere stati esplicitamente programmati}
}	

\newglossaryentry{marketplace}
{
	name=marketplace,
	description={Mercato online che raggruppa i prodotti di diversi venditori o siti web, consentendo la compravendita di tali beni o servizi}
}

\newglossaryentry{MeteorJS}
{
	name=MeteorJS,
	description={Framework web per JavaScript, gratuito ed open source, scritto in Node.js. Permette una prototipazione rapida e la produzione di codice multipiattaforma. Si integra con MongoDB, utilizza il protocollo DDP (Distributed Data Protocol) e un pattern di tipo publish-subscribe per la propagazione automatica delle modifiche dei dati ai clienti, senza richiedere allo sviluppatore di scrivere codice di sincronizzazione}
}		

\newglossaryentry{microservizio}
{
	name=microservizio,
	description={Unità software specializzata che viene generalmente eseguita su un processo di sistema. \MakeUppercase{è} prevista comunicazione tra i microservizi e può avvenire attraverso la rete o sulla stessa macchina. Ogni microservizio si propone all’esterno come una black-box, infatti espone solo un API, astraendo rispetto al dettaglio di come le funzionalità siano effettivamente implementate e dallo specifico linguaggio o tecnologia utilizzati. Ciò mira a far sì che il cambiamento di ciascun microservizio non abbia impatto sugli altri microservizi comunicanti}
}

\newglossaryentry{Microsoft Azure}
{
	name=Microsoft Azure,
	description={Piattaforma di cloud computing aperta, flessibile e di fascia Enterprise in continua evoluzione, usata da sviluppatori e i professionisti IT per creare, distribuire e gestire applicazioni tramite la nostra rete globale di data center. Azure supporta un'ampia gamma di sistemi operativi, linguaggi di programmazione, framework, database e dispositivi}
}	

\newglossaryentry{milestone}
{
	name=milestone,
	description={Indica importanti traguardi intermedi nello svolgimento del progetto. Molto spesso è rappresentata da eventi, cioè da attività con durata nulla o di un giorno massimo, e viene evidenziata in maniera diversa dalle altre attività nell'ambito dei documenti di progetto}
}		

\newglossaryentry{MongoDB}
{
	name=MongoDB,
	description={DBMS non relazionale, orientato ai documenti. Classificato come un database di tipo NoSQL, MongoDB si allontana dalla struttura tradizionale basata su tabelle dei database relazionali in favore di documenti in stile JSON con schema dinamico (BSON), rendendo l'integrazione di dati di alcuni tipi di applicazioni più facile e veloce}
}	

