\newglossaryentry{Java}
{
	name=Java,
	description={Java è un linguaggio di programmazione orientato agli oggetti a tipizzazione statica, specificatamente progettato per essere il più possibile indipendente dalla piattaforma di esecuzione}
}

\newglossaryentry{JavaScript}
{
	name=JavaScript,
	description={Linguaggio di scripting orientato agli oggetti e agli eventi, comunemente utilizzato nella programmazione web lato client per la creazione di effetti dinamici interattivi, tramite funzioni di script invocate da eventi innescati a loro volta in vari modi dall'utente sulla pagina web in uso (mouse, tastiera, caricamento della pagina, etc.)}
}

\newglossaryentry{Java Virtual Machine}
{
	name=Java Virtual Machine,
	description={La Java Virtual Machine o JVM, è il componente della piattaforma Java che esegue i programmi tradotti in bytecode dopo una prima compilazione}
}

\newglossaryentry{JavaScript ES6}
{
	name=JavaScript ES6,
	description={Sesta edizione del linguaggio di programmazione standardizzato e mantenuto da Ecma International nell'ECMA-262 ed ISO/IEC 16262}
}


\newglossaryentry{Jolie}
{
	name=Jolie,
	description={Acronimo per Java Orchestration Language Interpreter Engine, è un linguaggio di programmazione open source per lo sviluppo di applicazioni distribuite basate su microservizi. Nel paradigma a microservizi proposto da Jolie, ogni programma è un servizio che può comunicare con altri programmi tramite lo scambio di messaggi attraverso la rete. Jolie sfrutta un interprete implementato in linguaggio Java ed è, inoltre, supportato da più sistemi operativi, quali Linux, OS X e Windows}
}

\newglossaryentry{jQuery}
{
	name=jQuery,
	description={Libreria JavaScript per applicazioni web che semplifica la selezione, la manipolazione, la gestione degli eventi e l'animazione di elementi DOM in pagine HTML, ed implementa funzionalità AJAX. \MakeUppercase{è} un framework gratuito, distribuito sotto i termini della licenza MIT}
}

\newglossaryentry{JSON}
{
	name=JSON,
	description={Acronimo per JavaScript Object Notation, nell'ambito della programmazione web, è un formato adatto all'interscambio di dati fra applicazioni client-server}
}
