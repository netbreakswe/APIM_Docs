\newpage
\section{Test}
Al fine di produrre del software di qualità, nelle successive attività di progetto ,il team ha l'obiettivo di strutturare dei test volti a verificare che il software prodotto rispecchi le funzionalità richieste.
Tutte le attività di testing prodotte devono poter essere ripetibili e deterministiche, al fine di poter fornire informazioni utili a intraprendere azioni correttive, nel caso si ottengano dei risultati diversi da quelli attesi.
Per avere un tracciamento dei test prodotti e dei risultati ottenuti, si è scelto di rappresentare delle tabelle di log di facile consultazione, le quali forniscono un'indicazione degli output delle attività di verifica, eventuali errori e/o risultati non coerenti con quanto fissato.
Allo stato di avanzamento attuale del progetto, non sono ancora stati definiti i test specifici, che dovranno essere tracciati con i relativi requisiti. 

	\subsection{Test di unità}
	Questa tipologia di test serve a verificare il corretto comportamento dei singoli metodi o funzioni implementate.
	
	\subsection{Test di integrazione}
	Questa tipologia di test serve a verificare il corretto funzionamento delle singole componenti di sistema progettate durante l'attività di progettazione ad alto livello.
	Per questa tipologia di test, l'idea è quella di utilizzare un approccio top-down, il quale rispecchia la strategia incrementale.
	
	\subsection{Test di sistema}
	Questa tipologia di test serve a verificare il corretto comportamento e funzionamento dell’architettura.
	
	\subsection{Test di validazione}
	Questa tipologia di test serve a verificare che il prodotto soddisfi le richieste del proponente attraverso le funzionalità implementate.