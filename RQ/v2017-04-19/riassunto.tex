\section{Introduzione}

	\begin{itemize}
		\item Tipologia: riunione informativa;
		\item Redatto da: \MC;
		\item Data: 19 aprile 2017;
		\item Luogo: 1BC45 - Torre Archimede e Skype;
		\item Ora inizio: 10.00;
		\item Ora fine: 11.00;
		\item Presenti: \MC, \DAN, \DS, \NS, \AS;	
		\item Assenti: \AN.
	\end{itemize}

\section{Riassunto}
Il giorno mercoledì 19 aprile, il team \gruppo\ ed il proponente si confrontano via Skype per alcuni chiarimenti riguardanti il front-end dell'applicazione web e un particolare caso di possibile errore che il gateway dovrebbe gestire.\\
Il gruppo propone il marketplace di Mashape \url{https://market.mashape.com/} come spunto di ispirazione per la codifica del front-end dell'applicazione ed il proponente concorda in modo più che positivo.\\
Successivamente, viene introdotto un quesito riguardante un probabile caso di errore nell'API Gateway, ovvero l'aggregazione di più microservizi in un'unica interfaccia, con possibili conflitti nei nomi delle rispettive \textit{operation}.\\
La soluzione suggerita dal proponente consiste nell'utilizzo del tool \textit{jolie2surface} (\url{http://docs.jolie-lang.org/#!documentation/other_tools/jolie2surface.html}) che viene dato con Jolie, il quale in automatico restituisce l'interfaccia totale di un servizio visibile su una \textit{outputPort}.
Assieme a questo tool, viene suggerito l'utilizzo di due javaservice interni a Jolie: \textit{MetaJolie} e \textit{Parser}.\\
Con il primo vengono estratti i metadati di una porta di un servizio (decodificati in un value Jolie), mentre con il secondo c'è una API che genera l'interfaccia a partire dai metadati.

\section{Tracciamento decisioni}
Di seguito, vengono evidenziate le principali decisioni prese durante la riunione esterna del 19 aprile 2017.

\begin{longtable}{|>{\centering\arraybackslash}p{4cm}|>{\centering\arraybackslash}p{9cm}|}
	\hline \rowcolor{Gray}
	\textbf{Identificativo} & \textbf{Descrizione}\\
	\hline
	\endhead
			VE\_17	&  Marketplace Mashape come fonte di ispirazione per il front-end dell'applicazione web \\
			\hline
			VE\_18 &  Scelta del tool \textit{jolie2surface} e dei javaservice \textit{MetaJolie} e \textit{Parser} per la documentazione di una API\\
			\hline
		\caption{Tracciamento decisioni riunione 19 aprile 2017}
\end{longtable}