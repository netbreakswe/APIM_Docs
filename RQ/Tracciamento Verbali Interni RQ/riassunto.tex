\section{Introduzione}
Lo scopo di questo documento è quello di fornire una rapida consultazione riepilogativa di tutte le scelte significative prese durante le riunioni interne del team \gruppo, come spiegato nel documento \textsc{Norme Di Progetto 3\_0\_0.pdf}.\\
Per facilitarne la comprensione, viene presentata una tabella composta di tre colonne, una per identificare il verbale interno al quale si riferisce, una per l'identificativo e l'ultima per una breve descrizione.

\section{Tracciamento Verbali Interni}
Di seguito, vengono raccolte tutte le principali decisioni prese durante le varie riunioni interne sostenute dal team nel periodo che ha preceduto la \RQ.

\normalsize
\begin{longtable}{|>{\centering\arraybackslash}p{4.7cm}|>{\centering\arraybackslash}p{2.5cm} | >{\centering\arraybackslash}p{7cm}|}
	\hline \rowcolor{Gray}
	\textbf{Verbale} & \textbf{Id Decisione} & \textbf{Descrizione}\\
	\hline
	\endhead
			Verbale Interno 2017-03-16 & VI\_01	&  Valutazione e scambio di opinioni riguardo l'esito della revisione sostenuta \\
			\hline
			Verbale Interno 2017-03-16 & VI\_02 &  Assegnamento dei compiti per ciascun componente, in modo da attuare le azioni correttive necessarie \\
			\hline
			Verbale Interno 2017-03-31 & VI\_03	& Creazione di due sotto-gruppi (uno per il lato back-end e uno per il lato front-end) per un avanzamento parallelo ed ordinato	\\
			\hline
			Verbale Interno 2017-03-31 & VI\_04 & Assegnamento dei componenti del team a ciascun sotto-gruppo formato \\
			\hline
			Verbale Interno 2017-03-31 & VI\_05 & Viene avviata la stesura del nuovo documento Definizione di Prodotto \\
			\hline
			Verbale Interno 2017-04-28 & VI\_06	&  Verifica del front-end prodotto \\
			\hline
			Verbale Interno 2017-04-28 & VI\_07 &  Viene fatto un riepilogo della documentazione prodotta con controllo dei test fatti \\
			\hline
		\caption{Tracciamento Verbali Interni RQ}
\end{longtable}