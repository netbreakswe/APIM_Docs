\newpage
\section{T}

\begin{itemize}
	\item \textbf{Task}: attività da svolgere in un determinato periodo da un individuo. All'interno di un progetto questa attività è assegnata dal \textit{Responsabile di Progetto}.
	\item \textbf{Team}: gruppo di persone che lavorano insieme per uno stesso fine.
	\item \textbf{Telegram}: servizio di messaggistica istantanea basato su cloud, completamente gratuito e disponibile per diverse piattaforme. Il codice sorgente della parte lato client è open source, mentre quello lato server no, a differenza di Rocket.chat, per esempio.
	\item \textbf{TeXMaker}: editor \LaTeX gratuito, moderno e multi-piattaforma per i sistemi Linux, MacOSX e Windows che integra molti strumenti necessari per sviluppare documenti.
	\item \textbf{TeXStudio}: editor \LaTeX gratuito, moderno e multi-piattaforma per Linux, sistemi Mac OS e Microsoft Windows che integra molti strumenti utili per sviluppare documenti in \LaTeX. TeXstudio è uno strumento facile da usare e da configurare.
	\item \textbf{Tomcat}: software open source, utilizzato come web application server, ovvero un server capace di gestire e supportare le pagine JSP e le servlet nel rispetto delle specifiche 2.4 (per le servlet) e 2.0 (per le JSP).
\end{itemize}
