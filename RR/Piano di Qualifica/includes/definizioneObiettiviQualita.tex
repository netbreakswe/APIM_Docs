\newpage
\section{Definizione obiettivi qualità}
	
	Prendendo come riferimento lo standard \textit{ISO/IEC 9126\ped{G}} il \textit{team\ped{G}} si impegna a
	garantire nel prodotto \textit{API Market\ped{G}} le seguenti qualità:
	
	\subsection{Funzionalità}
		Per soddisfare questo obiettivo di qualità, il prodotto deve soddisfare i requisiti descritti nel documento \textsc{AnalisiDeiRequisiti 1\_0\_0.pdf}.  
		
		\begin{itemize}
			\item \textbf{Misura: }quantità di requisiti mappati;
			\item \textbf{Metrica: }la sufficienza è definita dal numero di requisiti obbligatori. Ogni requisito secondario andrà ad aumentare la qualità del prodotto, ma la priorità dovrà essere data ai requisiti obbligatori;
			\item \textbf{Strumenti: }il sistema dovrà superare tutti i test previsti. Gli strumenti utilizzati sono descritti nel documento \textsc{NormeDiProgetto 1\_0\_0.pdf}.
			
		\end{itemize}
	
	\subsection{Affidabilità}
		Il sistema dovrà dimostrarsi il più possibile robusto e garantire, in modo particolare, la disponibilità del servizio di \textit{API Gateway\ped{G}} in percentuale più alta possibile. 
		
		\begin{itemize}
			\item \textbf{Misura:} l'unità di misura utilizzata sarà il numero di chiamate a \textit{microservizi\ped{G}} andate a buon fine diviso il numero di chiamate totali;
			\item \textbf{Metrica: }il limite di sufficienza riguarderà il tasso di successo delle chiamate e il tempo che il sistema impiegherà per portarle a buon fine;
			\item \textbf{Strumenti: }il sistema dovrà superare tutti i test previsti.
			
		\end{itemize}
	
	\subsection{Usabilità}
		L’interfaccia web deve essere di facile utilizzo per l’utente, tenendo però in considerazione il target di utenza prevista. Si presume che il sistema verrà utilizzato da utenti con un livello di conoscenza informatica abbastanza evoluto. 
		
		\begin{itemize}
			\item \textbf{Misura: }l’unità di misura usata sarà una valutazione soggettiva dell’usabilità. Questo
			è dovuto all’inesistenza di una metrica oggettiva adatta allo scopo;
			\item \textbf{Metrica: }non esiste una metrica adeguata per determinare la sufficienza;
			\item \textbf{Strumenti: }si veda il documento \textsc{NormeDiProgetto 1\_0\_0.pdf}.
			
		\end{itemize}
	
	\subsection{Efficienza}
		Il sistema deve fornire tutte le funzionalità nel più breve tempo possibile, riducendo al minimo l’utilizzo di risorse.
		
		\begin{itemize}
			\item \textbf{Misura: }il tempo di latenza di chiamata ad un \textit{microservizio\ped{G}}, e il tempo di latenza nel caricamento delle pagine web;
			\item \textbf{Metrica: }i valori di latenza massimi verranno definiti in base al \textit{microservizio\ped{G}};
			\item \textbf{Strumenti: }si veda il documento \textsc{NormeDiProgetto 1\_0\_0.pdf}.
			
		\end{itemize}
	
	\subsection{Manutenibilità}
		Il sistema deve essere comprensibile ed estensibile in modo facile e verificabile.
		
		\begin{itemize}
			\item \textbf{Misura: }l’unità di misura utilizzata saranno le metriche per il codice descritte nella sezione 3.6.3;
			\item \textbf{Metrica: }il prodotto dovrà avere la sufficienza su tutte le metriche descritte nella sezione 3.6.3;
			\item \textbf{Strumenti: }si veda il documento \textsc{NormeDiProgetto 1\_0\_0.pdf}.
			
		\end{itemize}
	
	\subsection{Portabilità}
		Il \textit{front-end\ped{G}} dovrà essere utilizzabile dalle versioni più recenti dei più comuni browsers in commercio. 
		Il \textit{back-end\ped{G}}, basandosi sul linguaggio \textit{Jolie\ped{G}}, dipenderà dalle risorse necessarie all’utilizzo di questo linguaggio.
		
		\begin{itemize}
			\item \textbf{Misura: }l’unità di misura utilizzata saranno le metriche per il codice descritte nella sezione 3.6.3;
			\item \textbf{Metrica: }il prodotto dovrà avere la sufficienza su tutte le metriche descritte nella sezione 3.6.3;
			\item \textbf{Strumenti: }si veda il documento \textsc{NormeDiProgetto 1\_0\_0.pdf}.
			
		\end{itemize}
	
	\subsection{Altre qualità}
		Saranno inoltre importanti per il prodotto le seguenti qualità:
		
		\begin{itemize}
			\item \textbf{Incapsulamento: }applicare le tecniche di incapsulamento per aumentare la manutenibilità
			e la possibilità di riutilizzo del codice. Sarà quindi favorito l’uso di interfacce ove
			possibile;
			\item \textbf{Coesione: }riguarda le funzionalità che collaborano al fine di raggiungere uno stesso obiettivo. Esse devono risiedere nello stesso componente, ed hanno lo scopo di ridurre l’indice di dipendenza, favorire la semplicità e la manutenibilità.
			
		\end{itemize}
		
	
	
	