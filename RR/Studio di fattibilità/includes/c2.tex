\newpage
\section{Capitolato C2}

\subsection{Descrizione}

Il capitolato proposto da zero12 riguarda la creazione di un'applicazione
web che permetta agli ospiti di un'azienda di interrogare un assistente
vocale dedicato all'accoglienza. Il software sfrutterà il sistema
di comunicazione già esistente, basato su \textit{Slack\ped{G}}, per notificare l'arrivo dei visitatore. Tra le funzionalità obbligatorie da fornire,
vi è la richiesta dei dati personali dell'ospite e di sue eventuali
necessità (moduli, informazioni, caffè, etc.), che verranno inoltrate
all'interessato tramite \textit{Slack\ped{G}}. Il software dovrà utilizzare obbligatoriamente
la lingua inglese per adattarsi agli assistenti virtuali già disponibili.
L'applicativo sarà suddiviso in tre parti: l'interfaccia web per interagire
con l'utente, i servizi \textit{AWS Lambda\ped{G}} per l'interazione con l'\textit{API\ped{G}} dell'assistente
virtuale e l'interfacciamento. Sarà importante non solo la realizzazione
del software, ma anche uno studio del mercato attuale e l'analisi
del comportamento degli utenti, così da migliorarne le prestazioni
e la soddisfazione nell'utilizzo.

\subsection{Dominio applicativo}

Il progetto risulta interessante per qualunque organizzazione che voglia
gestire in modo automatico l'accoglienza dei visitatori presso la sua sede. In particolare,
è destinato a tutte le aziende che utilizzano \textit{Slack\ped{G}}: un buon punto di forza vista
la sua ampia diffusione. Inoltre, il recente interesse nel campo della
sintesi vocale potrebbe fornire un trampolino di lancio al progetto.
Se il software saprà garantire la soddisfazione dei visitatori e miglioramenti
nell'organizzazione dell'azienda utilizzatrice, potrà espandersi
anche in altri settori.

\subsection{Tecnologie}

Le tecnologie consigliate per il raggiungimento degli scopi del progetto proposto sono:
\begin{itemize}
	\item \textbf{\textit{HTML5\ped{G}}}, \textbf{\textit{CSS3\ped{G}}} e \textbf{\textit{JavaScript\ped{G}}}, per lo sviluppo dell'interfaccia web;
	\item \textbf{\textit{Bootstrap 3\ped{G}}}, come \textit{framework\ped{G}} \textit{CSS\ped{G}};
	\item \textbf{\textit{Node.js\ped{G}}} o \textbf{\textit{Swift\ped{G}}}, come linguaggio di programmazione per lo
	sviluppo dell'applicazione;
	\item \textbf{\textit{Express\ped{G}}}, come eventuale \textit{framework\ped{G}} \textit{Node.js\ped{G}};
	\item \textbf{\textit{Amazon Web Services\ped{G}}}, come infrastruttura di \textit{cloud computing\ped{G}};
	\item \textbf{\textit{MongoDB\ped{G}}} o \textbf{\textit{DynamoDB\ped{G}}}, come \textit{database NoSQL\ped{G}};
	\item \textbf{\textit{SiriSDK\ped{G}}} o \textbf{\textit{AlexaSDK\ped{G}}}, come assistenti virtuali.
\end{itemize}

\subsection{Aspetti critici}

Un punto cruciale del progetto riguarda la capacità di progettare
correttamente le interfacce rivolte ai servizi \textit{AWS Lambda\ped{G}}. Questi gestiranno lo scambio di dati tra gli utenti e gli altri elementi
interni quali \textit{Slack\ped{G}}, database ed assistente virtuale. L'apprendimento e l'integrazione di un numero relativamente alto di nuove tecnologie potrebbe, inizialmente, disorientare e far perdere la visione d'insieme del progetto.
Di natura totalmente diversa è l'analisi del comportamento degli
utenti: un'attività che richiede flessibilità nel rivalutare il proprio
punto di vista, mediando tra le necessità di sviluppatori ed utenti,
e la disponibilità di un numero ragionevole di volontari, il quanto più differenti tra loro.

\subsection{Considerazioni conclusive}

Il capitolato richiede la conoscenza e lo studio di un buon numero di tecnologie attuali e diffuse, permettendo nel contempo di realizzare un'applicazione concreta ed utile. Nonostante ciò, alcuni punti che l'hanno resa meno appetibile ai nostri occhi sono:
\begin{itemize}
	\item Sfruttare in modo superficiale le tante tecnologie richieste in fase di sviluppo;
	\item La maggior parte delle tecnologie da utilizzare richiede molto tempo per la comprensione e l'apprendimento.
\end{itemize}

La varietà delle tecnologie è controbilanciata dall'uso superficiale che se ne farà: queste non saranno sfruttate al massimo delle loro possibilità, poichè ciò che realmente interessa nel progetto è la comunicazione tra esse. Vista l'eterogeneità del gruppo, si è deciso di puntare ad un capitolato che impiegasse un minor numero di differenti tecnologie, ma che allo stesso tempo permettesse un utilizzo approfondito e dedicato. Inoltre, l'analisi di mercato e lo studio del comportamento
degli utenti appaiono come compiti non banali per chi non abbia esperienza nel settore della sintesi vocale. Per i motivi sopracitati, dunque, questo capitolato è passato in secondo piano rispetto agli altri proposti.