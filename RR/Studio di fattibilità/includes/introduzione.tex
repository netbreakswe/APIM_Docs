\newpage
\section{Introduzione}

\subsection{Scopo del documento}
Lo scopo del documento è quello di presentare una breve analisi di tutti i capitolati proposti, con le motivazioni che hanno portato il gruppo a scegliere il capitolato C1. Tutti i capitolati son stati analizzati con la medesima metodologia,  evidenziando le tecnologie necessarie, il dominio applicativo e le criticità, e dando un giudizio finale con le opinioni raccolte all'interno del gruppo.

\subsection{Scopo del prodotto}
Lo scopo del prodotto è la realizzazione di un \textit{API Market\ped{G}} per l'acquisto e la vendita di \textit{microservizi\ped{G}}. Il sistema offrirà la possibilità di registrare nuove \textit{API\ped{G}} per la vendita, permetterà la consultazione e la ricerca di \textit{API\ped{G}} ai potenziali acquirenti, gestendo i permessi di accesso ed utilizzo tramite creazione e controllo di relative \textit{API key\ped{G}}. Il sistema, oltre alla web app stessa, sarà corredato di un \textit{API Gateway\ped{G}} per la gestione delle richieste e il controllo delle chiavi, e fornirà funzionalità avanzate di statistiche per il gestore della piattaforma e per i fornitori dei \textit{microservizi\ped{G}}.

\subsection{Riferimenti normativi}
\begin{itemize}
	\item \textsc{NormeDiProgetto1\_0\_0.pdf}
\end{itemize}

\subsection{Riferimenti informativi}
\begin{itemize}
	\item \textbf{Capitolato d'appalto C1:} APIM: An API Market Platform \\ \url{http://www.math.unipd.it/~tullio/IS-1/2016/Progetto/C1.pdf}
	\item \textbf{Capitolato d'appalto C2:} AtAVi: Accoglienza tramite Assistente Virtuale \\ \url{http://www.math.unipd.it/~tullio/IS-1/2016/Progetto/C2.pdf}
	\item \textbf{Capitolato d'appalto C3:} DeGeOP: A Designer and Geo-localizer Web App for Organizational Plants \\
	\url{http://www.math.unipd.it/~tullio/IS-1/2016/Progetto/C3.pdf}
	\item \textbf{Capitolato d'appalto C4:} eBread: applicazione di lettura per dislessici \\
	\url{http://www.math.unipd.it/~tullio/IS-1/2016/Progetto/C4.pdf}
	\item \textbf{Capitolato d'appalto C5:} Monolith: an interactive bubble provider \\
	\url{http://www.math.unipd.it/~tullio/IS-1/2016/Progetto/C5.pdf}
	\item \textbf{Capitolato d'appalto C6:} SWEDesigner: editor di diagrammi UML con generazione di codice \\
	\url{http://www.math.unipd.it/~tullio/IS-1/2016/Progetto/C6.pdf}
\end{itemize}

\subsection{Glossario}
Per semplificare la consultazione e disambiguare alcune terminologie tecniche, le voci indicate con la lettera \textit{G} a pedice sono descritte approfonditamente nel documento \textsc{Glossario1\_0\_0.pdf}.