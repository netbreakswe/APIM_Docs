\newpage

\section{Processi organizzativi}

	\subsection{Gestione Organizzativa}
		
		\subsubsection{Comunicazioni}
		Al fine di ottenere comunicazioni chiare e concise il \textit{\RdP} ha l'obbligo di gestirle in modo strutturato, utilizzando la forma che meglio si adatta alla situazione. Le comunicazioni possono essere interne o esterne al \textit{team\ped{G}}.
			
			\paragraph{Comunicazioni interne}
			Le comunicazioni interne sono uno strumento ad uso esclusivo dei membri del gruppo e sono in stile informale, in forma scritta o orale. 
			Se la comunicazione avviene in forma orale tra alcuni membri del gruppo, è loro dovere riportarla in forma scritta se si tratta di un'informazione o una decisione importante riguardo lo sviluppo del progetto, utilizzando l’applicazione di instant messaging \textit{Telegram\ped{G}}, sulla quale è presente la chat del gruppo.

			\paragraph{Comunicazioni esterne}
			Per tutte le comunicazioni esterne, il gruppo \textit{\gruppo} si è dotato di una email ufficiale, necessaria per gestire le comunicazioni con committente e proponente. La gestione di tale casella di posta elettronica è affidata al \textit{\RdP}, il quale ha il compito di tenere informati i membri del gruppo riguardo le comunicazioni importanti. La casella di posta elettronica è:
			\begin{center}
				\url{netbreakswe@gmail.com} 
			\end{center}
		\paragraph{Coordinamento con secondo gruppo}
			Il proponente richiede una collaborazione tra i gruppi intenzionati a sviluppare il progetto proposto. In particolare, \proponente\ richiede un'intesa sull'architettura di massima del progetto. Per agevolare la comunicazione tra i gruppi è stato creato un gruppo di coordinamento per il capitolato C1 su\textit{Telegram\ped{G}}.
			
		\subsubsection{Riunioni}
		Le riunioni sono un elemento essenziale per il corretto svolgimento del progetto. Il loro compito è quello di permettere a tutti i membri del gruppo di confrontarsi tra loro, nel caso di riunioni interne, e con il proponente e il committente nel caso di riunione esterne. La corretta gestione dell'andamento di una riunione permette di risparmiare tempo e sfruttare al meglio le risorse umane.
		
			\paragraph{Riunioni interne}
			Il \textit{\RdP} ha la facoltà di decidere e convocare una riunione del gruppo. Tale convocazione deve essere inviata tramite posta elettronica a tutti i membri del gruppo, con almeno due giorni di anticipo, e deve contenere l’ordine del giorno. Tutti i membri del gruppo hanno il dovere di confermare la loro presenza entro ventiquattro ore dalla ricezione della comunicazione. Se impossibilitati a partecipare, devono fornire al \textit{\RdP} una valida motivazione alla loro assenza. Il \textit{\RdP}, inoltre, ha il compito di dirigere la riunione, tenendo fede all’ordine del giorno descritto nella mail di invito ed evitare perdite di tempo. Tutti i partecipanti devono presentarsi alla riunione preparati, aver rispettato le scadenze decise nella precendente riunione e caricato il materiale assegnato sulla piattaforma di web storage indicata. Le scadenze sono visibili sull'applicazione multipiattaforma \textit{Asana\ped{G}}.
			All’inizio di ogni riunione il \textit{\RdP} nominerà un segretario, che avrà il compito di tenere la minuta dell’incontro, stilare un verbale informale a disposizione esclusivamente del gruppo e creare le attività su \textit{Asana\ped{G}} con relative scadenze concordate.  
		
			\paragraph{Riunioni esterne}
			Queste riunioni sono fondamentali per il corretto svolgimento e progessivo andamento del progetto. Prima di ogni riunione, il \textit{\RdP} nominerà un segretario, incaricato di prendere nota dei temi discussi e redigere un verbale, evidenziando gli aspetti più rilevanti e/o le modifiche da apportare. Tale documento è riservato ad un uso esclusivo del \textit{team\ped{G}}.
			A questi incontri parteciperanno tutti i membri del gruppo, salvo impedimenti di rilievo, e sarà compito del \textit{\RdP} coordinare e gestire l’interazione tra il gruppo e l’interlocutore. Durante la riunione, i membri del gruppo potranno effettuare domande e/o chiedere chiarimenti. 
			
			\paragraph{Verbali}
			Il gruppo ha deciso di redigere dei verbali per ogni riunione che viene effettuata. I verbali sono un documento interno al gruppo, utile per tenere traccia degli argomenti discussi e delle decisioni prese. All'inizio della riunione, il \textit{\RdP} nomina un segretario, il quale ha il compito di procedere alla stesura del documento. Il documento contiene un abstract che indica il motivo della convocazione della riunione, un elenco puntato con infomazioni sulla riunione e un riassunto dei temi trattati. L'elenco contiene le seguenti informazioni:
			\begin{itemize}
				\item Generalità del segretario;
				\item Data di svolgimento della riunione;
				\item Luogo dove si svolge la riunione. Le riunioni possono avvenire sia fisicamente, sia tramite \textit{Skype\ped{G}}. Quest'ultimo metodo sarà utilizzato per brevi comunicazioni e aggiornamenti sullo svolgimento del progetto;
				\item Orario inizio riunione;
				\item Orario fine riunione;
				\item Membri assenti alla riunione.
			\end{itemize}

	\subsection{Ruoli e mansioni}
	Nel corso del progetto, ogni membro del \textit{team\ped{G}} svolgerà diverse mansioni. I ruoli sono standard, stabiliti con criterio e con mansioni mutuamente esclusive. Lo scopo di questa sezione è descrivere tutti i possibili ruoli che un membro del \textit{team\ped{G}} può assumere durante lo svolgimento del progetto. Essi verranno organizzati in modo tale da:
	\begin{itemize}
		\item Evitare conflitti di interesse, quali concomitanza dei ruoli di stesura e verifica di uno stesso documento;
		\item Ruotare i ruoli per permettere a ciascun membro di assumere qualsiasi posizione durante il corso del progetto;
		\item Permettere ad una persona di assumere più mansioni contemporaneamente.
	\end{itemize}

		\subsubsection{\RdP}
		Il \textit{\RdP} organizza il lavoro interno del \textit{team\ped{G}}, occupandosi della pianificazione delle attività. Egli mantiene i contatti con tutti gli enti esterni, gestisce le risorse, i rischi e la documentazione. Di quest'ultima, è l'unico incaricato di apporre l'approvazione definitiva. Le ulteriori mansioni organizzative del \textit{\RdP} sono quelle di: garantire il rispetto dei ruoli ed assicurarsi che le attività assegnate vengano svolte scrupolosamente nei tempi stabiliti. I documenti di sua diretta responsabilità sono \textsc{PianoDiProgetto 1\_0\_0.pdf} e \textsc{PianoDiQualifica 1\_0\_0.pdf}.

		\subsubsection{\Amm}
		L'\textit{\Amm} svolge una attività di controllo sull'ambiente di lavoro come mansione principale. Altre dirette responsabilità dell'\textit{\Amm} sono relative al versionamento di prodotto e alla documentazione. I documenti di sua diretta responsabilità sono \hbox{\textsc{NormeDiProgetto 1\_0\_0.pdf}} e \textsc{PianoDiProgetto 1\_0\_0.pdf}.
		
		\subsubsection{\Ana}
		La figura di \textit{\Ana} ha il ruolo di esaminare caratteristiche e problematiche del prodotto. Si occupa, dunque, della fase di Analisi per ogni aspetto del progetto. Si occupa direttamente della \textit{Specifica di Progetto} e del documento \textsc{AnalisiDeiRequisiti 1\_0\_0.pdf}.

		\subsubsection{\Prog}
		Il ruolo di \textit{\Prog} è una mansione che ha il compito primario di realizzare una produzione astratta del progetto, effettuando le scelte necessarie a mantenere il prodotto modulare, di facile manutenzione e ampliamento.
		
		\subsubsection{\Progr}
		Il \textit{\Progr} svolge l'attività di produzione di codice che implementa la soluzione finale. Egli ha il compito di seguire in maniera attenta la progettazione pregressa, effettuando la codifica seguendo le convenzioni descritte nel documento \hbox{\textsc{NormeDiProgetto 1\_0\_0.pdf}}. Si occupa, inoltre, di predisporre la fase di Test per la successiva fase di Verifica, e di documentare quanto viene prodotto.
		
		\subsubsection{\Ver}
		Il \textit{\Ver} svolge l'attività di Verifica della documentazione prodotta. Si attiene scrupolosamente a \hbox{\textsc{NormeDiProgetto 1\_0\_0.pdf}} per assicurarsi che i documenti prodotti siano conformi agli standard e alle convenzioni prefissate.

	\subsection{Strumenti}

		\subsubsection{GitHub}
		Per il controllo di versione e la gestione semplificata dei file di progetto e della documentazione, si è deciso di utilizzare \textbf{\textit{GitHub\ped{G}}}. La scelta è ricaduta su questo strumento, poichè altre piattaforme non consentono la collaborazione di sei membri o più in modo gratuito. Il \textit{team\ped{G}} lavorerà, dunque, su un repository pubblico. I membri del \textit{team\ped{G}} potranno, tramite questa soluzione, collaborare simultaneamente, sincronizzando le versioni in tempo reale o in un secondo momento, qualora mancasse la connessione. \textit{GitHub\ped{G}} offre, inoltre, una sezione wiki per la gestione di eventuale documentazione, e un sistema di issue tracking sufficientemente performante ai fini del progetto.

		\subsubsection{Google Drive}
		Per lo storage di ulteriori file, documenti e per tutto ciò che non è strettamente correlato alla produzione e consegna del progetto, sarà utilizzato il sistema di cloud storage \textbf{\textit{Google Drive\ped{G}}}. Ciò permette l'accesso dei file a tutti i membri e la possibilità di modificarli, pur non avendo una gestione capillare della versione, che, per l'appunto, è affidata a \textit{GitHub\ped{G}}.

		\subsubsection{Asana}
		Per permettere una più semplice suddivisione del lavoro e una conseguente assegnazione delle mansioni, si è scelto di utilizzare il software \textbf{\textit{Asana\ped{G}}}. Esso permette la creazione, modifica e assegnazione di \textit{task\ped{G}} e \textit{subtask\ped{G}}. Per ognuno di essi va indicato un assegnatario e una data di scadenza. La creazione di task è limitata alla sola figura di \textit{\RdP}, che coordina il lavoro tra i membri del gruppo, evitando squilibri e conflitti di interessi. I \textit{task\ped{G}} devono essere quanto più specifici possibile, e devono descrivere nel dettaglio i ruoli che l'assegnatario andrà a svolgere. Ogni componente del gruppo è tenuto a verificare le mansioni assegnate con frequenza giornaliera, ad aggiornare lo stato dei \textit{task\ped{G}} assegnatigli e notificare l'eventuale completamento, e a esporre o rispondere a dubbi e interrogativi sollevati inerenti all'attività assegnata, tramite l'apposita conversazione disponibile per ogni singolo \textit{task\ped{G}}.