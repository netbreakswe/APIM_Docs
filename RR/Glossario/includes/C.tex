\hypertarget{C}{}

\newglossaryentry{Camel Case}
{
	name=Camel Case,
	description={La Notazione a Cammello o in inglese Camel Case è la pratica nata durante gli anni settanta di scrivere parole composte o frasi unendo tutte le parole tra loro, ma lasciando le loro iniziali maiuscole}
}

\newglossaryentry{client}
{
	name=client,
	description={componente che accede ai servizi o alle risorse messe a disposizione da un server. Esso fa parte dell'architettura logica di rete client-server. Inoltre, il termine client indica anche il software usato sul computer-client per accedere alle funzionalità offerte da un server}
}

\newglossaryentry{cloud computing}
{
	name=cloud computing,
	description={letteralmente "nuvola informatica", indica un paradigma di erogazione di risorse informatiche come l'archiviazione, l'elaborazione o la trasmissione di dati, caratterizzato dalla disponibilità on demand attraverso Internet, a partire da un insieme di risorse preesistenti e configurabili}
}

\newglossaryentry{CSS}
{
	name=CSS,
	description={acronimo per Cascading Style Sheets (letteralmente fogli di stile a cascata), è un linguaggio usato per definire la formattazione di documenti HTML, XHTML e XML}
}

\newglossaryentry{CSS3}
{
	name=CSS3,
	description={ultima versione dello standard CSS. \MakeUppercase{è} retrocompatibile con le precedenti versioni di CSS}
}

\newglossaryentry{CSSLint}
{
	name=CSSLint,
	description={strumento che aiuta a rilevare possibili errori nel codice CSS}
}

\newglossaryentry{customer communication dashboard}
{
	name=customer communication dashboard,
	description={interfaccia grafica che organizza e presenta le informazioni circa l'acquisto o la disdetta di API in modo semplice, intuitivo ed immediato, consentendo al management di agire tempestivamente nella correzione della strategia in caso di necessità}
}



	
