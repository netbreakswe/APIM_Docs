\newpage

\section{Specifica Back-End}

\subsection{APIMarket::Back-End}

	\subsubsection{Informazioni generali}
	
	\begin{figure}[H]
		\centering
		\includegraphics
		[width=0.7\linewidth]
		{UML/DiagrammiPackage/BackEnd.png}
		\caption{Package APIM::BackEnd}
	\end{figure}
	
	\begin{itemize}
		\item \textbf{Descrizione:} Il package \textit{BackEnd} contiene le componenti del lato back-end dell'applicazione web.
		\item \textbf{Packages contenuti:}
		\begin{itemize}
			\item \textbf{Gateway:} package riguardante la gestione delle chiamate ai microservizi;
			\item \textbf{Services:} package riguardante la comunicazione con i database di API Market.
		\end{itemize}
	\end{itemize}
	
	\subsubsection{Interfacce}
	
	\paragraph{ServiceInteractionHandler}
	\begin{figure}[H]
		\centering
		\includegraphics
		[width=0.7\linewidth]
		{images/APIM/BackEnd/Interfacce/serviceInteractionHandler.png}
		\caption{Package APIM::BackEnd::ServiceInteractionHandler}
	\end{figure}
	
	\begin{itemize}
		\item \textbf{Descrizione:} L'interfaccia \textit{ServiceInteractionHandler} contiene le operazioni riguardanti la gestione delle sessioni couriers e dei metadati dei microservizi. Viene utilizzata dal \textit{Gateway} e da \textit{MicroservicesDB}.
	\end{itemize}

\subsection{APIMarket::Back-End::Gateway}

\subsubsection{Informazioni generali}

\begin{figure}[H]
	\centering
	\includegraphics
	[width=0.7\linewidth]
	{UML/DiagrammiPackage/gateway.png}
	\caption{Package APIM::BackEnd::Gateway}
\end{figure}

\begin{itemize}
	\item \textbf{Descrizione:} Il package \textit{Gateway} contiene le componenti per l'API Gateway dell'applicazione web.
	\item \textbf{Packages contenuti:}
	\begin{itemize}
		\item \textbf{Couriers:} package riguardante l'archiviazione delle sessioni couriers;
		\item \textbf{Interfaces:} package riguardante le interfacce necessarie alla classe Gateway.
	\end{itemize}
\end{itemize}

\subsubsection{Classi}

\paragraph{Gateway}
\begin{figure}[H]
	\centering
	\includegraphics
	[width=0.7\linewidth]
	{images/APIM/BackEnd/Classi/gateway.png}
	\caption{Package APIM::BackEnd::Gateway::Gateway}
\end{figure}

\begin{itemize}
	\item \textbf{Descrizione:} La classe Gateway genera le sessioni couriers di ciascun microservizio presente in API Market al suo avvio, ed in seguito alla registrazione di un nuovo microservizio.\\
	Inoltre, si occupa della verifica della chiamata al microservizio rispetto ai dati utente ed API Key, dei rilevamenti casuali di rispetto della SLA e della loro archiviazione, della redirection verso il microservizio e l'operazione desiderati.
	\item \textbf{Relazioni:}
		\begin{itemize}
			\item La classe Gateway implementa l'interfaccia RedirectorInterface;
			\item La classe Gateway utilizza le operazioni esposte dall'interfaccia ServiceInteractionHandler;
			\item La classe Gateway utilizza le operazione esposte dalle interfacce dei servizi di comunicazione con i database, presenti nel package Services;
			\item La classe Gateway genera le sessioni courier che vengono archiviate nel package Couriers.
		\end{itemize}
	\item \textbf{Operazioni:}
		\begin{itemize}
			\item \textbf{setNewRedirection( AService )( string ):} Genera il file di courier di un microservizio e ne imposta la redirection.
				\begin{description}
    				\item[\textbf{Parametri:}]
				\end{description}
				\begin{itemize}
					\item \textbf{AService:} Dati del microservizio dal quale generare un file courier. 
				\end{itemize}
			\item \textbf{validateCall( CallData )( boolean ):} Valida la chiamata al microservizio, verificando utente e relativa apikey. 
				\begin{description}
    				\item[\textbf{Parametri:}]
				\end{description}
				\begin{itemize}
					\item \textbf{CallData:} Dati della chiamata al microservizio.
				\end{itemize}
			\item \textbf{randomSLASurvey(  )( boolean ):} Algoritmo casuale per scegliere se effettuare un sondaggio della SLA.
			\item \textbf{manageCall( CallData )( boolean ):} Gestisce la chiamata al microservizio.
				\begin{description}
    				\item[\textbf{Parametri:}]
				\end{description}
				\begin{itemize}
					\item \textbf{CallData:} Dati della chiamata al microservizio.
				\end{itemize}
		\end{itemize}
\end{itemize}


\paragraph{ServiceInteractionHandler}
\begin{figure}[H]
	\centering
	\includegraphics
	[width=0.7\linewidth]
	{images/APIM/BackEnd/Classi/ServiceInteractionHandler.png}
	\caption{Package APIM::BackEnd::Gateway::ServiceInteractionHandler}
\end{figure}

\begin{itemize}
	\item \textbf{Descrizione:} La classe ServiceInteractionHandler implementa l'interfaccia contenuta in APIM::BackEnd::Gateway::ServiceInteractionHandler.
	\item \textbf{Relazioni:}
		\begin{itemize}
			\item La classe ServiceInteractionHandler espone le proprie operazioni alla classe Gateway;
			\item La classe ServiceInteractionHandler espone le proprie operazioni alla classe MicroservicesDB.
		\end{itemize}
	\item \textbf{Operazioni:}
		\begin{itemize}
			\item \textbf{GetServiceMetaFromCourier( CourierIdR )( MetaInfoAPIMService ):} Ricava da una courier i metadati del servizio, cioè tipi utilizzati ed operazioni fornite.
				\begin{description}
    				\item[\textbf{Parametri:}]
				\end{description}
				\begin{itemize}
					\item \textbf{CourierIdR:} Directory del file courier da cui estrarre le metainfo;
					\item \textbf{MetaInfoAPIMService:} Metadati di un microservizio registrato in API Market.
				\end{itemize}
			\item \textbf{generateClientInterface( MetaInfoAPIMService )( ClientInterface ):} Genera l'interfaccia esposta al client del microservizio.
				\begin{description}
    				\item[\textbf{Parametri:}]
				\end{description}
				\begin{itemize}
					\item \textbf{MetaInfoAPIMService:} Metadati di un microservizio registrato in API Market;
					\item \textbf{ClientInterface:} Rappresentazione come stringa dell'interfaccia esposta al client del microservizio.
				\end{itemize}
			\item \textbf{generateCourier( AService )( CourierService ):} Genera la rappresentazione come stringa del file courier del microservizio.
				\begin{description}
    				\item[\textbf{Parametri:}]
				\end{description}
				\begin{itemize}
					\item \textbf{AService:} Dati del microservizio dal quale generare un file courier;
					\item \textbf{CourierService:} Rappresentazione come stringa del file courier del microservizio.
				\end{itemize}
		\end{itemize}
\end{itemize}


\subsection{APIMarket::Back-End::Gateway::GatewayInterfaces}

\subsubsection{Informazioni generali}

\begin{figure}[H]
	\centering
	\includegraphics
	[width=0.7\linewidth]
	{UML/DiagrammiPackage/gatewayinterfaces.png}
	\caption{Package APIM::BackEnd::Gateway::GatewayInterfaces}
\end{figure}

\begin{itemize}
	\item \textbf{Descrizione:} Il package GatewayInterfaces contiene l'interfaccia necessarie al funzionamento del Gateway. In particolare si occupa delle attività di verifica, di gestione SLA e di redirection, trattando i dati riguardanti interfaccia ed operazione della chiamata ad un microservizio.
\end{itemize}

\subsubsection{Interfacce}

\paragraph{RedirectorInterface}
\begin{figure}[H]
	\centering
	\includegraphics
	[width=0.7\linewidth]
	{images/APIM/BackEnd/Interfacce/RedirectorInterface.png}
	\caption{Package APIM::BackEnd::GatewayInterfaces::RedirectorInterface}
\end{figure}

\begin{itemize}
	\item \textbf{Descrizione:} L'interfaccia RedirectorInterface si occupa delle attività di redirection di una chiamata ad un microservizio. Sfruttando la classe ServiceInteractionHandler per generare una sessione courier, ne crea la classe Courier e le indirizza la redirection. Viene implementata dalla classe Gateway.
\end{itemize}

\subsection{APIMarket::Back-End::Services}

\subsubsection{Informazioni generali}

\begin{figure}[H]
	\centering
	\includegraphics
	[width=0.7\linewidth]
	{UML/DiagrammiPackage/services.png}
	\caption{Package APIM::BackEnd::Services}
\end{figure}

\begin{itemize}
	\item \textbf{Descrizione:} Il package Services contiene le componenti per la comunicazione con i database di API Market.
	\item \textbf{Packages contenuti:}
	\begin{itemize}
		\item \textbf{MicroservicesDB:} package riguardante la comunicazione con il database dei microservizi registrati in API Market;
		\item \textbf{UsersDB:} package riguardante la comunicazione con il database degli utenti in API Market;
		\item \textbf{TransactionsDB:} package riguardante la comunicazione con il database delle transazioni in API Market;
		\item \textbf{SLADB:} package riguardante la comunicazione con il database della SLA in API Market.
		\item \textbf{FileHandlerDB:} package riguardante la comunicazione con il database dei file in API Market.
	\end{itemize}
\end{itemize}

\subsection{APIMarket::Back-End::Services::MicroservicesDB}

\subsubsection{Informazioni generali}

\begin{figure}[H]
	\centering
	\includegraphics
	[width=0.7\linewidth]
	{UML/DiagrammiPackage/microservicesDB.png}
	\caption{Package APIM::BackEnd::Services::MicroservicesDB}
\end{figure}

\begin{itemize}
	\item \textbf{Descrizione:} Il package MicroservicesDB contiene le componenti per la comunicazione con i database dei microservizi di API Market.
\end{itemize}

\subsubsection{Interfacce}

\paragraph{MicroservicesDBInterface}
\begin{figure}[H]
	\centering
	\includegraphics
	[width=0.7\linewidth]
	{images/APIM/BackEnd/Interfacce/microservicesDBInterface.png}
	\caption{Package APIM::BackEnd::Services::MicroservicesDBInterface}
\end{figure}

\begin{itemize}
	\item \textbf{Descrizione:} L'interfaccia MicroservicesDB contiene le operazioni riguardanti lettura e scrittura dei dati dei microservizi, delle interfacce, delle categorie dei microservizi. Viene utilizzata dal Gateway.
\end{itemize}

\subsubsection{Classi}

\paragraph{MicroservicesDB}
\begin{figure}[H]
	\centering
	\includegraphics
	[width=0.7\linewidth]
	{images/APIM/BackEnd/Classi/microservicesDB.png}
	\caption{Package APIM::BackEnd::Services::MicroservicesDB}
\end{figure}

\begin{itemize}
	\item \textbf{Descrizione:} La classe MicroservicesDB implementa l'interfaccia contenuta in \\
	\begin{minipage} 
		{\linewidth} APIM::BackEnd::Services::MicroservicesInterface.
	\end{minipage}

	\item \textbf{Relazioni:}
		\begin{itemize}
			\item La classe MicroservicesDB implementa l'interfaccia MicroservicesDBInterface;
			\item La classe MicroservicesDB utilizza le operazioni esposte dall'interfaccia ServiceInteractionHandlerInterface;
			\item La classe MicroservicesDB espone le proprie operazioni al FrontEnd.
		\end{itemize}
	\item \textbf{Operazioni:}
		\begin{itemize}
		
			\item \textbf{retrieveAllMSGatewayMeta( void )( ListServices ):} Ricava le informazioni dei microservizi per il gateway in API Market.
			\begin{description}
				\item[\textbf{Parametri:}]
			\end{description}
			\begin{itemize}
				\item \textbf{ListServices:} Lista dei microservizi con relative informazioni.
			\end{itemize}
		
			\item \textbf{retrieveClientInterfaceFromID( Id )( InfoInterf ):} Ricava le informazioni di ogni cliente tramite il suo ID in API Market.
			\begin{description}
				\item[\textbf{Parametri:}]
			\end{description}
			\begin{itemize}
				\item \textbf{Id:} Id del cliente di cui ricavare le informazioni;
				\item \textbf{InfoInterf:} informazioni del cliente.
			\end{itemize}
		
			\item \textbf{homepageMSlist( void )( HomeMSList ):} Ricava le informazioni di ogni microservizio registrato in API Market.
				\begin{description}
    				\item[\textbf{Parametri:}]
				\end{description}
				\begin{itemize}
					\item \textbf{HomeMSList:} Lista dei microservizi per la homepage.
				\end{itemize}
			
			\item \textbf{retreiveAllMSInfo( void )( MSData ):} Ricava le informazioni di tutti microservizi.
				\begin{description}
					\item[\textbf{Parametri:}]
				\end{description}
				\begin{itemize}
					\item \textbf{MSData:} Informazioni di tutti i  microservizi.
				\end{itemize}
				
			\item \textbf{retreiveMSInfo( Id )( MSData ):} Ricava le informazioni del microservizio a partire dal suo id.
				\begin{description}
    				\item[\textbf{Parametri:}]
				\end{description}
				\begin{itemize}
					\item \textbf{Id:} Id del microservizio;
					\item \textbf{MSData:} Informazioni del microservizio.
				\end{itemize}
				
			\item \textbf{checkMSIsCompliant( CompliantData )( boolean ):} controlla se un microsrvizio i soddisfa i rquisiti SLA. 
				\begin{description}
    				\item[\textbf{Parametri:}]
				\end{description}
				\begin{itemize}
					\item \textbf{CompliantData:} Id del microservizio da controllare;
					\item \textbf{bool.}
				\end{itemize}
			
			\item \textbf{checkMSIsActive( Id )( boolean ):} controlla se un microservizio è attivo. 
				\begin{description}
					\item[\textbf{Parametri:}]
				\end{description}
				\begin{itemize}
					\item \textbf{CompliantData:} Id del microservizio da controllare;
					\item \textbf{bool.}
				\end{itemize}
		
			\item \textbf{retrieveMSPolicy( Id )( int ):} ritorna la policy del microservizio. 
			\begin{description}
				\item[\textbf{Parametri:}]
			\end{description}
			\begin{itemize}
				\item \textbf{Id:} Id del microservizio da controllare;
				\item \textbf{int:} interno che identifica la policy.
			\end{itemize}

			\item \textbf{retrieveIntfInfo( Id )( IntfData ):} recupera tutte le informazioni dell'interfaccia. 
			\begin{description}
				\item[\textbf{Parametri:}]
			\end{description}
			\begin{itemize}
				\item \textbf{Id:} Id del microservizio da controllare;
				\item \textbf{IntfData:} ritorna le informazioni dell'interfaccia.
			\end{itemize}
				
			\item \textbf{retrieveMSFromDeveloperId( StringId )( MSDevList ):} Ricava le informazioni di un microservizio a partire dall'Id di uno sviluppatore.
				\begin{description}
    				\item[\textbf{Parametri:}]
				\end{description}
				\begin{itemize}
					\item \textbf{StringId:} Id del microservizio;
					\item \textbf{MSDevList:} informazioni microservizio.
				\end{itemize}
				
			\item \textbf{retrieveInterfacesOfMS( Id )( ListInterfaces ):} Ricava l'interfaccia del microservizio a partire dall'id.
				\begin{description}
    				\item[\textbf{Parametri:}]
				\end{description}
				\begin{itemize}
					\item \textbf{Id:} Id dell'interfaccia;
					\item \textbf{ListInterfaces:} Interfaccia del microservizio.
				\end{itemize}
				
			\item \textbf{retrieveMSFromInterface( Id )( MSIdData ):} Ricava le informazioni di un microservizio a partire dalla sua interfaccia.
				\begin{description}
    				\item[\textbf{Parametri:}]
				\end{description}
				\begin{itemize}
					\item \textbf{Id:} Id dell'interfaccia;
					\item \textbf{MSIdData:} Lista degli id dei microservizi appartenenti alla categoria.
				\end{itemize}
			
			\item \textbf{retrieveCategoryInfo( Id )( CategoryData ):} Ricava le informazioni di una categoria a partire dal suo Id.
			\begin{description}
				\item[\textbf{Parametri:}]
			\end{description}
			\begin{itemize}
				\item \textbf{Id:} Id della categoria;
				\item \textbf{CategoryData:} informazioni di una categoria.
			\end{itemize}
				
			\item \textbf{retrieveCategoryList( void )( CategoryList ):} Ricava le informazioni di una categoria a partire dal suo Id.
			\begin{description}
				\item[\textbf{Parametri:}]
			\end{description}
			\begin{itemize}
				\item \textbf{CategoryList:} lista API della categoria.
			\end{itemize}
				
			\item \textbf{retrieveCategoryOfMS( Id )( CategoryDataList ):} Ricava gli id delle categorie attribuite al microservizio, a partire dall'id del microservizio.
				\begin{description}
    				\item[\textbf{Parametri:}]
				\end{description}
				\begin{itemize}
					\item \textbf{Id:} Id del microservizio;
					\item \textbf{CategoryDataList:} Lista degli id delle categorie attribuite al microservizio.
				\end{itemize}
				
			\item \textbf{retrieveMSNumberFromDevId( StringId )( int ):} Ricava il numero di microservizi registrati da un developer.
			\begin{description}
				\item[\textbf{Parametri:}]
			\end{description}
			\begin{itemize}
				\item \textbf{StringId:} stringa Id associata a uno sviluppatore;
				\item \textbf{int:} numero di microservizi registrati da un developer.
			\end{itemize}	
		
			\item \textbf{retrieveActiveMSNumberFromDevId( StringId )( int ):} Ricava il numero di microservizi attivi di un developer.
			\begin{description}
				\item[\textbf{Parametri:}]
			\end{description}
			\begin{itemize}
				\item \textbf{StringId:} stringa Id associata a uno sviluppatore;
				\item \textbf{int:} numero di microservizi attivi di un developer.
			\end{itemize}
		
			\item \textbf{retrieveInactiveMSNumberFromDevId( StringId )( int ):} Ricava il numero di microservizi inattivi di un developer.
			\begin{description}
				\item[\textbf{Parametri:}]
			\end{description}
			\begin{itemize}
				\item \textbf{StringId:} stringa Id associata a uno sviluppatore;
				\item \textbf{int:} numero di microservizi inattivi di un developer.
			\end{itemize}	
			
			\item \textbf{microserviceRegistration( MSDataW )( int ):} Registra un nuovo microservizio nel database dei microservizi.
			\begin{description}
				\item[\textbf{Parametri:}]
			\end{description}
			\begin{itemize}
				\item \textbf{MSDataW:} Informazioni del nuovo microservizio;
				\item \textbf{int.}
			\end{itemize}
			
			\item \textbf{microserviceUpdate( MSUpData )( void ):} Aggiorna le informazioni di un microservizio esistente nel database dei microservizi.
			\begin{description}
				\item[\textbf{Parametri:}]
			\end{description}
			\begin{itemize}
				\item \textbf{MSUpData:} Informazioni aggiornate di un microservizio esistente.
			\end{itemize}
			
			\item \textbf{interfaceUpdate( intfUpData )( void ):} Aggiorna le informazioni di una interfaccia esistente nel database dei microservizi.
			\begin{description}
				\item[\textbf{Parametri:}]
			\end{description}
			\begin{itemize}
				\item \textbf{intfupdata:} Informazioni aggiornate di una interfaccia esistente.
			\end{itemize}
			
			\item \textbf{addCategoryToMS( CategoryDataW )( void ):} Attribuisce una categoria ad un microservizio.
			\begin{description}
				\item[\textbf{Parametri:}]
			\end{description}
			\begin{itemize}
				\item \textbf{CategoryDataW:} Id del microservizio e della categoria da attribuirgli.
			\end{itemize}
		
			\item \textbf{removeCategoryFromMS( CategoryDataW )( void ):} Rimuove una categoria da un microservizio.
			\begin{description}
				\item[\textbf{Parametri:}]
			\end{description}
			\begin{itemize}
				\item \textbf{CategoryDataW:} Id del microservizio e della categoria da rimuovergli.
			\end{itemize}
		
			\item \textbf{updateClientInterfaceById( InfoInterfUpdate )( void ):} Aggiorna le informazioni di una interfaccia esistente nel database dei microservizi.
			\begin{description}
				\item[\textbf{Parametri:}]
			\end{description}
			\begin{itemize}
				\item \textbf{InfoInterfUpdate:} Informazioni aggiornate di una interfaccia esistente.
			\end{itemize}
		
			\item \textbf{changeIsActive( AvailabilityData )( void ):} aggiorna lo stato di un microservizio.
			\begin{description}
				\item[\textbf{Parametri:}]
			\end{description}
			\begin{itemize}
				\item \textbf{AvailabilityData:} Informazioni di un microservizio.
			\end{itemize}
			
			\item \textbf{deleteMS( Id )( bool ):} Aggiorna le informazioni di una interfaccia esistente nel database dei microservizi.
			\begin{description}
				\item[\textbf{Parametri:}]
			\end{description}
			\begin{itemize}
				\item \textbf{Id:} Id del microservizio da eliminare;
				\item \textbf{bool}.
			\end{itemize}

		\end{itemize}
\end{itemize}

\subsection{APIMarket::Back-End::Services::UsersDB}

\subsubsection{Informazioni generali}

\begin{figure}[H]
	\centering
	\includegraphics
	[width=0.7\linewidth]
	{UML/DiagrammiPackage/usersDB.png}
	\caption{Package APIM::BackEnd::Services::UsersDB}
\end{figure}

\begin{itemize}
	\item \textbf{Descrizione:} Il package \textit{UsersDB} contiene le componenti per la comunicazione con il database degli utenti di API Market.
\end{itemize}

\subsubsection{Interfacce}

\paragraph{UsersDBInterface}
\begin{figure}[H]
	\centering
	\includegraphics
	[width=0.7\linewidth]
	{images/APIM/BackEnd/Interfacce/usersDBInterface.png}
	\caption{Package APIM::BackEnd::Services::UsersDBInterface}
\end{figure}

\begin{itemize}
	\item \textbf{Descrizione:} L'interfaccia UsersDBInterface contiene le operazioni riguardanti lettura e scrittura dei dati degli utenti (sia admin che clienti), dei tipi di cliente, delle moderazioni attuate, dei tipi di moderazione. Viene utilizzata dal Gateway.
\end{itemize}

\subsubsection{Classi}

\paragraph{UsersDB}
\begin{figure}[H]
	\centering
	\includegraphics
	[width=0.7\linewidth]
	{images/APIM/BackEnd/Classi/usersDB.png}
	\caption{Package APIM::BackEnd::Services::UsersDB}
\end{figure}

\begin{itemize}
	\item \textbf{Descrizione:} La classe UsersDB implementa l'interfaccia contenuta in 
	\begin{minipage}{\linewidth}
			APIM::BackEnd::Services::UsersDBInterface.
	\end{minipage}

	
	\item \textbf{Relazioni:}
		\begin{itemize}
			\item La classe UsersDB implementa l'interfaccia UsersDBInterface;
			\item La classe UsersDB espone le proprie operazioni alla classe FrontendMS.
		\end{itemize}
	\item \textbf{Operazioni:}
		\begin{itemize}
		
			\item \textbf{emailExists( Email )( bool ):} Controlla se l'email esista nel database utenti.
				\begin{description}
					\item[\textbf{Parametri:}]
				\end{description}
				\begin{itemize}
					\item \textbf{Email:} Email del cliente;
					\item \textbf{bool}.
				\end{itemize}
		
			\item \textbf{userExists( LoginInfo )( bool ):} Controlla se il cliente esista nel database utenti.
				\begin{description}
    				\item[\textbf{Parametri:}]
				\end{description}
				\begin{itemize}
					\item \textbf{LoginInfo:} Email ed Password del cliente;
					\item \textbf{bool}.
				\end{itemize}
			
			\item \textbf{adminExists( LoginInfo )( bool ):} Controlla se l'aaministratore esista nel database admin.
				\begin{description}
					\item[\textbf{Parametri:}]
				\end{description}
				\begin{itemize}
					\item \textbf{LoginInfo:} Email ed Password dell'amministratore;
					\item \textbf{bool}.
				\end{itemize}

			\item \textbf{retrieveAdminInfo( StringID )( AdminData ):} Ricava le informazioni dell'admin, a partire dal suo id.
				\begin{description}
    				\item[\textbf{Parametri:}]
				\end{description}
				\begin{itemize}
					\item \textbf{StringID:} Id dell'admin;
					\item \textbf{AdminData:} Informazioni dell'admin.
				\end{itemize}
			
			\item \textbf{retrieveAdminInfoFromEmail( Email )( AdminData ):} Ricava le informazioni dell'admin, a partire dalla sua email.
				\begin{description}
					\item[\textbf{Parametri:}]
				\end{description}
				\begin{itemize}
					\item \textbf{Email:} email dell'admin;
					\item \textbf{AdminData:} Informazioni dell'admin.
				\end{itemize}	
				
			\item \textbf{retrieveClientInfo( StringId )( UserData ):} Ricava le informazioni del cliente, a partire dal suo id.
				\begin{description}
    				\item[\textbf{Parametri:}]
				\end{description}
				\begin{itemize}
					\item \textbf{clientid:} Id del cliente.
					\item \textbf{clientdata:} Informazioni del cliente.
				\end{itemize}
			
			\item \textbf{retrieveAllClientInfo( void )( AllUserData ):} Ricava tutte le informazioni dei clienti.
				\begin{description}
					\item[\textbf{Parametri:}]
				\end{description}
				\begin{itemize}
					\item \textbf{AllUserData:} Informazioni dei clienti.
				\end{itemize}
			
			\item \textbf{retrieveClientInfoFromEmail( Email )( UserData ):} Ricava le informazioni del cliente, a partire dalla sua email.
				\begin{description}
					\item[\textbf{Parametri:}]
				\end{description}
				\begin{itemize}
					\item \textbf{Email:} email del cliente;
					\item \textbf{AdminData:} Informazioni del cliente.
				\end{itemize}	
			
				
			\item \textbf{retrieveClientAnagraphics ( StringId )( Anagraphics ):} Ricava Name e Surname del client, a partire dal suo id.
				\begin{description}
    				\item[\textbf{Parametri:}]
				\end{description}
				\begin{itemize}
					\item \textbf{StringId:} Id del cliente;
					\item \textbf{Anagraphics:} anagrafica completa del cliente.
				\end{itemize}
				
			\item \textbf{retrieveClientType( Id )( TypeIdData ):} Ricava il tipo di account del cliente, a partire dal suo id.
				\begin{description}
    				\item[\textbf{Parametri:}]
				\end{description}
				\begin{itemize}
					\item \textbf{Id:} Id del cliente;
					\item \textbf{TypeIdData:} Id del tipo di account del cliente.
				\end{itemize}
			
			\item \textbf{retrieveModerationInfo( Id )( EntryData ):} Ricava le informazioni della moderazione, a partire dal suo id.
				\begin{description}
    				\item[\textbf{Parametri:}]
				\end{description}
				\begin{itemize}
					\item \textbf{Id:} Id della moderazione;
					\item \textbf{EntryData:} Informazioni della moderazione.
				\end{itemize}
				
			\item \textbf{retrieveModTypeInfo( Id )( ModTypeData ):} Ricava le informazioni del tipo di moderazione, a partire dal suo id.
				\begin{description}
    				\item[\textbf{Parametri:}]
				\end{description}
				\begin{itemize}
					\item \textbf{Id:} Id del tipo di moderazione;
					\item \textbf{ModTypeData:} Informazioni del tipo di moderazione.
				\end{itemize}
				
			\item \textbf{retrieveClientTypeInfo( Id )( ClientTypeData ):} Ricava le informazioni del tipo di account del cliente, a partire dal suo id.
				\begin{description}
    				\item[\textbf{Parametri:}]
				\end{description}
				\begin{itemize}
					\item \textbf{Id:} Id del tipo di account del cliente;
					\item \textbf{ClientTypeData:} Informazioni del tipo di account del cliente.
				\end{itemize}
			
			\item \textbf{RetrieveAllDevId( void )( DevIdList ):} Ricava le informazioni di tutti gli sviluppatori.
			\begin{description}
				\item[\textbf{Parametri:}]
			\end{description}
			\begin{itemize}
				\item \textbf{DevIdList:} informazioni di tutti gli sviluppatori.
			\end{itemize}
								
			\item \textbf{basicClientRegistration( BasicClientData )( void ):} Registra un nuovo account cliente di tipo base.
				\begin{description}
    				\item[\textbf{Parametri:}]
				\end{description}
				\begin{itemize}
					\item \textbf{BasicClientData:} Informazioni del nuovo cliente.
				\end{itemize}
				
			\item \textbf{developerUpgrade( StringId )( void ):} Effettua l'upgrade ad account sviluppatore di un cliente.
				\begin{description}
    				\item[\textbf{Parametri:}]
				\end{description}
				\begin{itemize}
					\item \textbf{StringId:} Informazioni dei campi aggiuntivi da sviluppatore.
				\end{itemize}
				
			\item \textbf{basicClientDowngrade( StringId )( void ):} Effettua il downgrade ad account base di un cliente sviluppatore.
				\begin{description}
    				\item[\textbf{Parametri:}]
				\end{description}
				\begin{itemize}
					\item \textbf{StringId:} Id del cliente.
				\end{itemize}
				
			\item \textbf{clientModeration( SmallEntryData )( void ):} Inserisce una nuova moderazione nel database degli utenti.
				\begin{description}
    				\item[\textbf{Parametri:}]
				\end{description}
				\begin{itemize}
					\item \textbf{SmallEntryData:} Informazioni della nuova moderazione.
				\end{itemize}
				
			\item \textbf{clientUpdate( UserUpData )( void ):} Aggiorna le informazioni di un cliente.
				\begin{description}
    				\item[\textbf{Parametri:}]
				\end{description}
				\begin{itemize}
					\item \textbf{userdata:} Informazioni aggiornate del cliente.
				\end{itemize}
			
			\item \textbf{clientePasswordChange( PassUpData )( void ):} Cambia la password di un utente.
				\begin{description}
					\item[\textbf{Parametri:}]
				\end{description}
				\begin{itemize}
					\item \textbf{PassUpData:} Cambia la password di un utente.
				\end{itemize}
		
			\item \textbf{clientePasswordRecovery( RecoverPassUpData )( void ):} avvia il recupero password di un cliente.
				\begin{description}
					\item[\textbf{Parametri:}]
				\end{description}
				\begin{itemize}
					\item \textbf{RecoverPassUpData:} nuova password del cliente del cliente.
				\end{itemize}
			
			
				
			\item \textbf{clientDelete( StringId )( void ):} Elimina un cliente da API Market.
				\begin{description}
    				\item[\textbf{Parametri:}]
				\end{description}
				\begin{itemize}
					\item \textbf{clientDelete:} Id del cliente.
				\end{itemize}
					
		\end{itemize}
\end{itemize}

\subsection{APIMarket::Back-End::Services::TransactionsDB}

\subsubsection{Informazioni generali}

\begin{figure}[H]
	\centering
	\includegraphics
	[width=0.7\linewidth]
	{UML/DiagrammiPackage/transactionsDB.png}
	\caption{Package APIM::BackEnd::Services::TransactionsDB}
\end{figure}

\begin{itemize}
	\item \textbf{Descrizione:} Il package \textit{TransactionsDB} contiene le componenti per la comunicazione con il database delle transazioni di API Market.
\end{itemize}

\subsubsection{Interfacce}

\paragraph{TransactionsDBInterface}
\begin{figure}[H]
	\centering
	\includegraphics
	[width=0.7\linewidth]
	{images/APIM/BackEnd/Interfacce/transactionsDBInterface.png}
	\caption{Package APIM::BackEnd::Services::TransactionsDBInterface}
\end{figure}

\begin{itemize}
	\item \textbf{Descrizione:} L'interfaccia TransactionsDBInterface contiene le operazioni riguardanti lettura e scrittura dei dati delle apikey, degli acquisti. Viene utilizzata dal Gateway.
\end{itemize}

\subsubsection{Classi}

\paragraph{TransactionsDB}
\begin{figure}[H]
	\centering
	\includegraphics
	[width=0.7\linewidth]
	{images/APIM/BackEnd/Classi/transactionsDB.png}
	\caption{Package APIM::BackEnd::Services::TransactionsDB}
\end{figure}

\begin{itemize}
	\item \textbf{Descrizione:} La classe TransactionsDB implementa l'interfaccia contenuta in 
		\begin{minipage}{\linewidth}
			APIM::BackEnd::Services::TransactionsDBInterface.
	\end{minipage}

	
	
	\item \textbf{Relazioni:}
		\begin{itemize}
			\item La classe TransactionsDB implementa l'interfaccia TransactionsDBInterface;
		\end{itemize}
	\item \textbf{Operazioni:}
		\begin{itemize}
			
			\item \textbf{checkAPIKeyExist( APIKey )( boolean ):} verifica che l'API Key esista.
				\begin{description}
					\item[\textbf{Parametri:}]
				\end{description}
				\begin{itemize}
					\item \textbf{APIKey:} Informazioni dell'apikey;
					\item \textbf{bool.}
				\end{itemize}
		
			\item \textbf{retrieveAPIKeyInfo( APIKeyLicense )( APIkeyData ):} Ricava le informazioni dell'apikey, a partire dal suo id.
				\begin{description}
    				\item[\textbf{Parametri:}]
				\end{description}
				\begin{itemize}
					\item \textbf{APIKeyLicense:} Id dell'apikey;
					\item \textbf{APIkeyData:} Informazioni dell'apikey.
				\end{itemize}
			
			item \textbf{checkAPIKeyIsActive( APIKeyActiveData )( boolean ):} verifica che l'API Key sia attiva.
				\begin{description}
					\item[\textbf{Parametri:}]
				\end{description}
				\begin{itemize}
					\item \textbf{APIKey:} Informazioni dell'apikey;
					\item \textbf{bool.}
				\end{itemize}
			
			\item \textbf{retrieveActiveAPIKeyFromUserId( StringId )( APIKeyList ):} Ricava la lista con le api attive del cliente, a partire dal suo id.
				\begin{description}
					\item[\textbf{Parametri:}]
				\end{description}
				\begin{itemize}
					\item \textbf{StringId:} Id del cliente;
					\item \textbf{APIKeyList:} lista con le api attive del cliente.
				\end{itemize}
			
			\item \textbf{retrieveAPIKeyFromMSIdAndClient( IdMSAndClient )( int ):} Ricava la lista delle API da un Id di un microservizio e del cliente.
			\begin{description}
				\item[\textbf{Parametri:}]
			\end{description}
			\begin{itemize}
				\item \textbf{IdMSAndClient:} Id del microservizio e cliente.
				\item \textbf{int.} 
			\end{itemize}	
				
			\item \textbf{retrieveActiveAPIKeyNumberFromMSId( Id )( APIKeyNumber ):} Ricava il numero di APIKey attive di un microservizio, a partire dal suo Id.
				\begin{description}
    				\item[\textbf{Parametri:}]
				\end{description}
				\begin{itemize}
					\item \textbf{Id:} Id del microservizio;
					\item \textbf{APIKeyNumber:} numero di APIKey attive.
				\end{itemize}
			
			\item \textbf{retrieveActiveAPIKeyUserFromMSId( StringId )( UserIdList ):} Ricavala lista di clienti con una API Key attiva di un microservizio, a partire dal suo Id.
				\begin{description}
					\item[\textbf{Parametri:}]
				\end{description}
				\begin{itemize}
					\item \textbf{Id:} Id del microservizio;
					\item \textbf{UserIdList:} utenti con una APIKey attive.
				\end{itemize}
			
			\item \textbf{retrievePurchasesListFromUserId( StringId )( PurchasesList ):} Ricava la lista di microservizi acquistati da un cliente, a partire dal suo Id.
				\begin{description}
					\item[\textbf{Parametri:}]
				\end{description}
				\begin{itemize}
					\item \textbf{StringId:} Id del cliente;
					\item \textbf{UserIdList:} lista API acquistate.
				\end{itemize}
	
			\item \textbf{retrievePurchasesSumFromMSId( StringId )( PurchasesSumData ):} recupera la somma dell'ammontare degli acquisti di un microservizio, a partire dal suo Id.
				\begin{description}
					\item[\textbf{Parametri:}]
				\end{description}
				\begin{itemize}
					\item \textbf{StringId:} Id del microservizio;
					\item \textbf{PurchasesSumData:} somma dell'ammontare degli acquisti.
				\end{itemize}
			
			\item \textbf{retrieveMonthlyPurchasesSumFromMSId( StringId )( PurchasesSumData ):} recupera la somma dell'ammontare degli acquisti di un microservizio in un mese, a partire dal suo Id.
				\begin{description}
					\item[\textbf{Parametri:}]
				\end{description}
				\begin{itemize}
					\item \textbf{StringId:} Id del microservizio;
					\item \textbf{PurchasesSumData:} somma dell'ammontare degli acquisti mensile.
				\end{itemize}
		
			\item \textbf{retrieveMSListFromClientId( StringId )( MSremainingList ):} recupera la lista dei microservizi con apikey attiva, a partire dall'id del cliente.
				\begin{description}
					\item[\textbf{Parametri:}]
				\end{description}
				\begin{itemize}
					\item \textbf{StringId:} Id del cliente;
					\item \textbf{MSremainingList:} sommalista microservizi con APIKey attiva.
				\end{itemize}
				
			\item \textbf{APIKeyRegistration( apikeydataw )( bool ):} Registra una nuova apikey nel database delle transazioni.
				\begin{description}
    				\item[\textbf{Parametri:}]
				\end{description}
				\begin{itemize}
					\item \textbf{apikeydataw:} Informazioni della nuova apikey.
					\item \textbf{bool}.
				\end{itemize}
				
			\item \textbf{purchaseRegistration( purchasedata )( bool ):} Inserisce una nuova transazione nel database delle transazioni.
				\begin{description}
    				\item[\textbf{Parametri:}]
				\end{description}
				\begin{itemize}
					\item \textbf{purchasedata:} Informazioni della nuova transazione.
					\item \textbf{bool}.
				\end{itemize}
			
			\item \textbf{APIKeyUpdate( APIKeyUpdateData )( void ):} aggiorna l'apikey con una nuova apikey.
			\begin{description}
				\item[\textbf{Parametri:}]
			\end{description}
			\begin{itemize}
				\item \textbf{APIKeyUpdateData:} nuova apikey registrata.
			\end{itemize}
				
			\item \textbf{APIKeyRemainingUpdate( apikeyremainingdata )( bool ):} Aggiorna il Remaining della apikey, a partire dal suo id e dal valore del cambiamento.
				\begin{description}
    				\item[\textbf{Parametri:}]
				\end{description}
				\begin{itemize}
					\item \textbf{apikeyremainingdata:} Id dell'apikey e valore del cambiamento.
					\item \textbf{bool}.
				\end{itemize}
				
		\end{itemize}
\end{itemize}

\subsection{APIMarket::Back-End::Services::SLADB}

\subsubsection{Informazioni generali}

\begin{figure}[H]
	\centering
	\includegraphics
	[width=0.7\linewidth]
	{UML/DiagrammiPackage/SLADB.png}
	\caption{Package APIM::BackEnd::Services::SLADB}
\end{figure}

\begin{itemize}
	\item \textbf{Descrizione:} Il package \textit{SLADB} contiene le componenti per la comunicazione con il database della SLA di API Market.
\end{itemize}

\subsubsection{Interfacce}

\paragraph{SlaDBInterface}
\begin{figure}[H]
	\centering
	\includegraphics
	[width=0.7\linewidth]
	{images/APIM/BackEnd/Interfacce/slaDBInterface.png}
	\caption{Package APIM::BackEnd::Services::SlaDBInterface}
\end{figure}

\begin{itemize}
	\item \textbf{Descrizione:} L'interfaccia SlaDBInterface contiene le operazioni riguardanti lettura e scrittura dei dati dei sondaggi SLA. Viene utilizzata dal Gateway.
\end{itemize}

\subsubsection{Classi}

\paragraph{SlaDB}
\begin{figure}[H]
	\centering
	\includegraphics
	[width=0.7\linewidth]
	{images/APIM/BackEnd/Classi/slaDB.png}
	\caption{Package APIM::BackEnd::Services::SlaDB}
\end{figure}

\begin{itemize}
	\item \textbf{Descrizione:} La classe SlaDB implementa l'interfaccia contenuta in // APIM::BackEnd::Services::SlaDBInterface.
	\item \textbf{Relazioni:}
		\begin{itemize}
			\item La classe SlaDB implementa l'interfaccia SlaDBInterface;
		\end{itemize}
	\item \textbf{Operazioni:}
		\begin{itemize}
		
			\item \textbf{retrieveAPIKeySLASurveyList( Id )( SLASurveyList ):} Ricava la lista dei sondaggi SLA dell'apikey con relative informazioni, a partire dall'id dell'apikey.
				\begin{description}
    				\item[\textbf{Parametri:}]
				\end{description}
				\begin{itemize}
					\item \textbf{apikeyid:} Id dell'apikey;
					\item \textbf{SLASurveyList:} Lista dei sondaggi SLA con relative informazioni.
				\end{itemize}
				
			\item \textbf{retrieveMSSLASurveyList( Id )( SLASurveyListMS ):} Ricava la lista dei sondaggi SLA del microservizio con relative informazioni, a partire dall'id del microservizio.
				\begin{description}
    				\item[\textbf{Parametri:}]
				\end{description}
				\begin{itemize}
					\item \textbf{Id:} Id del microservizio;
					\item \textbf{SLASurveyListMS:} Lista dei sondaggi SLA con relative informazioni.
				\end{itemize}
				
			\item \textbf{retrieveSLASurveyInfo( Id )( SLASurvey ):} Ricava le informazioni del sondaggio SLA, a partire dal suo id.
				\begin{description}
    				\item[\textbf{Parametri:}]
				\end{description}
				\begin{itemize}
					\item \textbf{Id:} Id del sondaggio SLA;
					\item \textbf{SLASurvey:} Informazioni del sondaggio SLA.
				\end{itemize}
				
			\item \textbf{retrieveSLASurveyIsCompliant( Id )( boolean ):} Ricava IsCompliant del sondaggio SLA, a partire dal suo id.
				\begin{description}
    				\item[\textbf{Parametri:}]
				\end{description}
				\begin{itemize}
					\item \textbf{Id:} Id del sondaggio SLA;
					\item \textbf{bool}.
				\end{itemize}
				
			\item \textbf{retrieveAverageResponseTimeFromMSId( Id )( long ):} Ricava il tempo medio di risposta di un microservizio, a partire dal suo id.
				\begin{description}
					\item[\textbf{Parametri:}]
				\end{description}
				\begin{itemize}
					\item \textbf{Id:} Id del microservizio;
					\item \textbf{long}.
				\end{itemize}
			
			
			\item \textbf{SLASurveyInsert( SLASurveyDataW )( void ):} Inserisce un nuovo sondaggio SLA nel database della SLA.
				\begin{description}
    				\item[\textbf{Parametri:}]
				\end{description}
				\begin{itemize}
					\item \textbf{SLASurveyDataW:} Informazioni del nuovo sondaggio SLA.
				\end{itemize}
				
		\end{itemize}
\end{itemize}

\subsection{APIMarket::Back-End::Services::FilehandlerDB}

\subsubsection{Informazioni generali}

\begin{figure}[H]
	\centering
	\includegraphics
	[width=0.7\linewidth]
	{UML/DiagrammiPackage/filehandlerDB.png}
	\caption{Package APIM::BackEnd::Services::FilehandlerDB}
\end{figure}

\begin{itemize}
	\item \textbf{Descrizione:} Il package FilehandlerDB contiene le componenti per la comunicazione con il database dei file di API Market.
\end{itemize}

\subsubsection{Interfacce}

\paragraph{FilehandlerInterface}
\begin{figure}[H]
	\centering
	\includegraphics
	[width=0.7\linewidth]
	{images/APIM/BackEnd/Interfacce/fileHandlerInterface.png}
	\caption{Package APIM::BackEnd::Services::FilehandlerDB::FilehandlerInterface}
\end{figure}

\begin{itemize}
	\item \textbf{Descrizione:} L'interfaccia FilehandlerDBInterface contiene le operazioni riguardanti lettura e scrittura dei dati dei file. Viene utilizzata dal Gateway.
\end{itemize}

\subsubsection{Classi}

\paragraph{FileHandlerDB}
\begin{figure}[H]
	\centering
	\includegraphics
	[width=0.7\linewidth]
	{images/APIM/BackEnd/Classi/fileHandler.png}
	\caption{Package APIM::BackEnd::Services::FileHandlerDB::FileHandler}
\end{figure}

\begin{itemize}
	\item \textbf{Descrizione:} La classe Filehandler implementa l'interfaccia contenuta in APIM::BackEnd::Services::FileHandlerDBInterface.
	\item \textbf{Relazioni:}
		\begin{itemize}
			\item La classe FileHandlerDB implementa l'interfaccia FileHandlerDBInterface.
		\end{itemize}
	\item \textbf{Operazioni:}
		\begin{itemize}
		
			\item \textbf{setFile( SetFileRequest )( string ):} Archivia il file, se non è già presente nell'archivio, salvandolo con un nome univoco e ricava l'uri corrispondente.
				\begin{description}
    				\item[\textbf{Parametri:}]
				\end{description}
				\begin{itemize}
					\item \textbf{SetFileRequest:} Informazioni del file;
					\item \textbf{string}.
				\end{itemize}
				
			\item \textbf{fileNameExists( string )( boolean ):} Controlla se nel database dei file esista un file con il nome specificato.
				\begin{description}
    				\item[\textbf{Parametri:}]
				\end{description}
				\begin{itemize}
					\item \textbf{string};
					\item \textbf{bool}.
				\end{itemize}
				
		\end{itemize}
\end{itemize}