\newpage
\section{Resoconto Qualità di Processo}

In questa sezione del documento vengono descritti e analizzati gli esiti delle metriche previste per il controllo della qualità di processo, in base alla revisione di progetto.\\
La prima tabella contiene le metriche con associato esito applicate sull'intero sistema, mentre le successive tabelle sono una per ogni metrica che si applica a singole componenti.

	\subsection{Revisione di Qualifica}
	
		\begin{longtable}{|>{\centering\arraybackslash}p{2cm}|>{\centering\arraybackslash}p{5cm}|>{\centering\arraybackslash}p{3cm}|>{\centering\arraybackslash}p{3cm}|}
			\hline
			\rowcolor{Gray}
			\textbf{Id} & \textbf{Metrica} & \textbf{Valore} & \textbf{Esito} \\
			\hline
			\hyperlink{MPC1}{MPC1} & \textit{Disponibilità \textit{NetBreakDB}} & 95\% & \textcolor{Green}{\textit{Superato}}\\
			\hline
			\hyperlink{MPC2}{MPC2} & \textit{Schedule Variance} & 0 & \textcolor{Green}{\textit{Superato}}\\
			\hline
			\hyperlink{MPC3}{MPC3} & \textit{Budget Variance} & 75 & \textcolor{Green}{\textit{Superato}}\\
			\hline
			\hyperlink{MPC4}{MPC4} & \textit{Rischi non preventivati} & 1 & \textcolor{Green}{\textit{Superato}}\\
			\hline
			\hyperlink{MPC5}{MPC5} & \textit{Adempimento requisiti obbligatori} & 100\% & \textcolor{Green}{\textit{Superato}}\\
			\hline
			\hyperlink{MPC13}{MPC13} & \textit{Linee di commento} & 20\% & \textcolor{Green}{\textit{Superato}}\\
			\hline
			\hyperlink{MPC14}{MPC14} & \textit{Test di Unità} & 96\% & \textcolor{Green}{\textit{Superato}}\\
			\hline
			\hyperlink{MPC15}{MPC15} & \textit{Test di Integrazione} & 70\% & \textcolor{Green}{\textit{Superato}}\\
			\hline
			\hyperlink{MPC16}{MPC16} & \textit{Test di Sistema} & 80\% & \textcolor{Green}{\textit{Superato}}\\
			\hline
			\hyperlink{MPC18}{MPC18} & \textit{Test superati} & 88\% & \textcolor{Green}{\textit{Superato}}\\
			\hline
			\hyperlink{MPC20}{MPC20} & \textit{Code Coverage} & 67\% & \textcolor{Green}{\textit{Superato}}\\
			\hline
		
			\caption{Resoconto esiti metriche - Qualità di processo RQ}
		\end{longtable}
	
\subsubsection{Numero di metodi per classe}
La metrica \textit{Numero di metodi per classe} (\hyperlink{MPC8}{MPC8}), si applica ad ogni classe progettata ed implementata.

\begin{longtable}{|>{\centering\arraybackslash}p{5cm}|>{\centering\arraybackslash}p{3cm}|>{\centering\arraybackslash}p{3cm}|}
	\hline
	\rowcolor{Gray}
	\textbf{Classe} & \textbf{Valore} & \textbf{Esito} \\
	\hline
	
	\textit{AppRouter} & 1 & \textcolor{Green}{\textit{Superato}}\\
	\hline
	\textit{AppRun} & 1 & \textcolor{Green}{\textit{Superato}}\\
	\hline
	\textit{ProfileManagerController} & 3 & \textcolor{Green}{\textit{Superato}}\\
	\hline
	\textit{APIRegistrationController} & 2 & \textcolor{Green}{\textit{Superato}}\\
	\hline
	\textit{SellingPolicyController} & 2 & \textcolor{Green}{\textit{Superato}}\\
	\hline
	\textit{UserRegistrationController} & 3 & \textcolor{Green}{\textit{Superato}}\\
	\hline
	\textit{APIRegisteredController} & 3 & \textcolor{Green}{\textit{Superato}}\\
	\hline
	\textit{SearchController} & 7 & \textcolor{Green}{\textit{Superato}}\\
	\hline
	\textit{TransactionsListController} & 2 & \textcolor{Green}{\textit{Superato}}\\
	\hline
	\textit{AdminManagerController} & 3 & \textcolor{Green}{\textit{Superato}}\\
	\hline
	\textit{AdminModerationController} & 7 & \textcolor{Green}{\textit{Superato}}\\
	\hline
	\textit{LoginController} & 4 & \textcolor{Green}{\textit{Superato}}\\
	\hline
	\textit{VirtualAccountController} & 4 & \textcolor{Green}{\textit{Superato}}\\
	\hline
	\textit{PasswordRecoveryController} & 3 & \textcolor{Green}{\textit{Superato}}\\
	\hline
	\textit{ResetPasswordController} & 3 & \textcolor{Green}{\textit{Superato}}\\
	\hline
	\textit{APIPurchasedController} & 2 & \textcolor{Green}{\textit{Superato}}\\
	\hline
	\textit{Gateway} & 4 & \textcolor{Green}{\textit{Superato}}\\
	\hline
	\textit{ServiceInteractionHandler} & 3 & \textcolor{Green}{\textit{Superato}}\\
	\hline
	\textit{Microservices\_db} & 15 & \textcolor{Green}{\textit{Superato}}\\
	\hline
	\textit{Users\_db} & 14 & \textcolor{Green}{\textit{Superato}}\\
	\hline
	\textit{Transactions\_db} & 5 & \textcolor{Green}{\textit{Superato}}\\
	\hline
	\textit{SLA\_db} & 5 & \textcolor{Green}{\textit{Superato}}\\
	\hline
	\textit{FileHandler} & 2 & \textcolor{Green}{\textit{Superato}}\\
	\hline
	\textit{FrontEndMS} & 2 & \textcolor{Green}{\textit{Superato}}\\
	\hline
	
	\caption{Resoconto esiti metrica Numero di metodi per classe - RQ}
\end{longtable}

\subsubsection{Numero di parametri per metodo}
La metrica \textit{Numero di parametri per metodo} (\hyperlink{MPC9}{MPC9}), si applica ad ogni metodo progettato ed implementato.

\begin{longtable}{|>{\centering\arraybackslash}p{10cm}|>{\centering\arraybackslash}p{2cm}|>{\centering\arraybackslash}p{3cm}|}
	\hline
	\rowcolor{Gray}
	\textbf{Metodo} & \textbf{Valore} & \textbf{Esito} \\
	\hline
	
	\textit{AppRouter::AppRouter()} & 2 & \textcolor{Green}{\textit{Superato}}\\
	\hline
	\textit{AppRun::initialSetting()} & 0 & \textcolor{Green}{\textit{Superato}}\\
	\hline
	\textit{ProfileManagerController::ProfileManagerController()} & 3 & \textcolor{Green}{\textit{Superato}}\\
	\hline
	\textit{ProfileManagerController::confirm()} & 2 & \textcolor{Green}{\textit{Superato}}\\
	\hline
	\textit{ProfileManagerController::getUserDetails()} & 1 & \textcolor{Green}{\textit{Superato}}\\
	\hline
	\textit{APIRegistrationController::APIRegistrationController()} & 4 & \textcolor{Green}{\textit{Superato}}\\
	\hline
	\textit{APIRegistrationController::insertMicroservice()} & 1 & \textcolor{Green}{\textit{Superato}}\\
	\hline
	\textit{SellingPolicyController::SellingPolicyController()} & 4 & \textcolor{Green}{\textit{Superato}}\\
	\hline
	\textit{SellingPolicyController::selectPolicy()} & 1 & \textcolor{Green}{\textit{Superato}}\\
	\hline
	\textit{UserRegistrationController::UserRegistrationController()} & 3 & \textcolor{Green}{\textit{Superato}}\\
	\hline
	\textit{UserRegistrationController::signUp()} & 1 & \textcolor{Green}{\textit{Superato}}\\
	\hline
	\textit{UserRegistrationController::login()} & 0 & \textcolor{Green}{\textit{Superato}}\\
	\hline
	\textit{APIRegisteredController::APIRegisteredController()} & 3 & \textcolor{Green}{\textit{Superato}}\\
	\hline
	\textit{APIRegisteredController::getMicroservicesDetails()} & 1 & \textcolor{Green}{\textit{Superato}}\\
	\hline
	\textit{APIRegisteredController::setMicroservice()} & 1 & \textcolor{Green}{\textit{Superato}}\\
	\hline
	\textit{SearchController::SearchController()} & 3 & \textcolor{Green}{\textit{Superato}}\\
	\hline
	\textit{SearchController::searchUser()} & 1 & \textcolor{Green}{\textit{Superato}}\\
	\hline
	\textit{SearchController::searchMicroservice()} & 1 & \textcolor{Green}{\textit{Superato}}\\
	\hline
	\textit{SearchController::viewUser()} & 1 & \textcolor{Green}{\textit{Superato}}\\
	\hline
	\textit{SearchController::viewMicroserviceInfo()} & 1 & \textcolor{Green}{\textit{Superato}}\\
	\hline
	\textit{SearchController::buyMicroservice()} & 1 & \textcolor{Green}{\textit{Superato}}\\
	\hline
	\textit{SearchController::goToResultsPage()} & 1 & \textcolor{Green}{\textit{Superato}}\\
	\hline
	\textit{TransactionsListController::TransactionsListController()} & 4 & \textcolor{Green}{\textit{Superato}}\\
	\hline
	\textit{TransactionsListController::getTransactions()} & 1 & \textcolor{Green}{\textit{Superato}}\\
	\hline
	\textit{AdminManagerController::AdminManagerController()} & 4 & \textcolor{Green}{\textit{Superato}}\\
	\hline
	\textit{AdminManagerController::confirm()} & 2 & \textcolor{Green}{\textit{Superato}}\\
	\hline
	\textit{AdminManagerController::getAdminDetails()} & 1 & \textcolor{Green}{\textit{Superato}}\\
	\hline
	\textit{AdminModerationController::AdminModerationController()} & 4 & \textcolor{Green}{\textit{Superato}}\\
	\hline
	\textit{AdminModerationController::selectUser()} & 1 & \textcolor{Green}{\textit{Superato}}\\
	\hline
	\textit{AdminModerationController::selectMicroservice()} & 1 & \textcolor{Green}{\textit{Superato}}\\
	\hline
	\textit{AdminModerationController::suspendUser()} & 1 & \textcolor{Green}{\textit{Superato}}\\
	\hline
	\textit{AdminModerationController::suspendMicroservice()} & 1 & \textcolor{Green}{\textit{Superato}}\\
	\hline
	\textit{AdminModerationController::removeUser()} & 1 & \textcolor{Green}{\textit{Superato}}\\
	\hline
	\textit{AdminModerationController::removeMicroservice()} & 1 & \textcolor{Green}{\textit{Superato}}\\
	\hline
	\textit{LoginController::LoginController()} & 5 & \textcolor{Green}{\textit{Superato}}\\
	\hline
	\textit{LoginController::login()} & 2 & \textcolor{Green}{\textit{Superato}}\\
	\hline
	\textit{LoginController::registration()} & 0 & \textcolor{Green}{\textit{Superato}}\\
	\hline
	\textit{LoginController::recoveryPassword()} & 0 & \textcolor{Green}{\textit{Superato}}\\
	\hline
	\textit{VirtualAccountController::VirtualAccountController()} & 4 & \textcolor{Green}{\textit{Superato}}\\
	\hline
	\textit{VirtualAccountController::confirm()} & 1 & \textcolor{Green}{\textit{Superato}}\\
	\hline
	\textit{VirtualAccountController::getBalance()} & 1 & \textcolor{Green}{\textit{Superato}}\\
	\hline
	\textit{VirtualAccountController::rechargeCredits()} & 2 & \textcolor{Green}{\textit{Superato}}\\
	\hline
	\textit{PasswordRecoveryController::PasswordRecoveryController()} & 2 & \textcolor{Green}{\textit{Superato}}\\
	\hline
	\textit{PasswordRecoveryController::passwordForgot()} & 0 & \textcolor{Green}{\textit{Superato}}\\
	\hline
	\textit{PasswordRecoveryController::login()} & 0 & \textcolor{Green}{\textit{Superato}}\\
	\hline
	\textit{ResetPasswordController::ResetPasswordController()} & 3 & \textcolor{Green}{\textit{Superato}}\\
	\hline
	\textit{ResetPasswordController::setNewPassword()} & 0 & \textcolor{Green}{\textit{Superato}}\\
	\hline
	\textit{ResetPasswordController::confirmNewPassword()} & 0 & \textcolor{Green}{\textit{Superato}}\\
	\hline
	\textit{APIPurchasedController::APIPurchasedController()} & 5 & \textcolor{Green}{\textit{Superato}}\\
	\hline
	\textit{APIPurchasedController::getMicroservices()} & 1 & \textcolor{Green}{\textit{Superato}}\\
	\hline
	\textit{Gateway::setNewRedirection()} & 1 & \textcolor{Green}{\textit{Superato}}\\
	\hline
	\textit{Gateway::validateCall()} & 1 & \textcolor{Green}{\textit{Superato}}\\
	\hline
	\textit{Gateway::random\_SLA\_survey()} & 0 & \textcolor{Green}{\textit{Superato}}\\
	\hline
	\textit{Gateway::manage\_call()} & 1 & \textcolor{Green}{\textit{Superato}}\\
	\hline
	\textit{ServiceInteractionHandler::getServiceMetaFromCourier()} & 1 & \textcolor{Green}{\textit{Superato}}\\
	\hline
	\textit{ServiceInteractionHandler::generateClientInterface()} & 1 & \textcolor{Green}{\textit{Superato}}\\
	\hline
	\textit{ServiceInteractionHandler::generateCourier()} & 1 & \textcolor{Green}{\textit{Superato}}\\
	\hline
	\textit{Microservices\_db::retrieve\_all\_ms\_info()} & 1 & \textcolor{Green}{\textit{Superato}}\\
	\hline
	\textit{Microservices\_db::retrieve\_ms\_info()} & 1 & \textcolor{Green}{\textit{Superato}}\\
	\hline
	\textit{Microservices\_db::retrieve\_intf\_info()} & 1 & \textcolor{Green}{\textit{Superato}}\\
	\hline
	\textit{Microservices\_db::retrieve\_interfaces\_of\_ms()} & 1 & \textcolor{Green}{\textit{Superato}}\\
	\hline
	\textit{Microservices\_db::retrieve\_ms\_from\_interface()} & 1 & \textcolor{Green}{\textit{Superato}}\\
	\hline
	\textit{Microservices\_db::retrieve\_msidlist\_from\_category()} & 1 & \textcolor{Green}{\textit{Superato}}\\
	\hline
	\textit{Microservices\_db::retrieve\_category\_info()} & 1 & \textcolor{Green}{\textit{Superato}}\\
	\hline
	\textit{Microservices\_db::retrieve\_categories\_of\_ms()} & 1 & \textcolor{Green}{\textit{Superato}}\\
	\hline
	\textit{Microservices\_db::retrieve\_last\_registered\_ms()} & 1 & \textcolor{Green}{\textit{Superato}}\\
	\hline
	\textit{Microservices\_db::retrieve\_last\_registered\_ms\_info()} & 1 & \textcolor{Green}{\textit{Superato}}\\
	\hline
	\textit{Microservices\_db::microservice\_registration()} & 1 & \textcolor{Green}{\textit{Superato}}\\
	\hline
	\textit{Microservices\_db::microservice\_update()} & 1 & \textcolor{Green}{\textit{Superato}}\\
	\hline
	\textit{Microservices\_db::interface\_registration()} & 1 & \textcolor{Green}{\textit{Superato}}\\
	\hline
	\textit{Microservices\_db::interface\_update()} & 1 & \textcolor{Green}{\textit{Superato}}\\
	\hline
	\textit{Microservices\_db::add\_category\_to\_ms()} & 1 & \textcolor{Green}{\textit{Superato}}\\
	\hline
	\textit{Microservices\_db::remove\_category\_from\_ms()} & 1 & \textcolor{Green}{\textit{Superato}}\\
	\hline
	\textit{Users\_db::user\_exists()} & 1 & \textcolor{Green}{\textit{Superato}}\\
	\hline
	\textit{Users\_db::retrieve\_admin\_info()} & 1 & \textcolor{Green}{\textit{Superato}}\\
	\hline
	\textit{Users\_db::retrieve\_client\_info()} & 1 & \textcolor{Green}{\textit{Superato}}\\
	\hline
	\textit{Users\_db::retrieve\_client\_fullname()} & 1 & \textcolor{Green}{\textit{Superato}}\\
	\hline
	\textit{Users\_db::retrieve\_client\_type()} & 1 & \textcolor{Green}{\textit{Superato}}\\
	\hline
	\textit{Users\_db::retrieve\_moderation\_info()} & 1 & \textcolor{Green}{\textit{Superato}}\\
	\hline
	\textit{Users\_db::retrieve\_modtype\_info()} & 1 & \textcolor{Green}{\textit{Superato}}\\
	\hline
	\textit{Users\_db::retrieve\_clienttype\_info()} & 1 & \textcolor{Green}{\textit{Superato}}\\
	\hline
	\textit{Users\_db::basicclient\_registration()} & 1 & \textcolor{Green}{\textit{Superato}}\\
	\hline
	\textit{Users\_db::developer\_upgrade()} & 1 & \textcolor{Green}{\textit{Superato}}\\
	\hline
	\textit{Users\_db::basicclient\_downgrade()} & 1 & \textcolor{Green}{\textit{Superato}}\\
	\hline
	\textit{Users\_db::client\_moderation()} & 1 & \textcolor{Green}{\textit{Superato}}\\
	\hline
	\textit{Users\_db::client\_update()} & 1 & \textcolor{Green}{\textit{Superato}}\\
	\hline
	\textit{Users\_db::client\_delete()} & 1 & \textcolor{Green}{\textit{Superato}}\\
	\hline
	\textit{Transactions\_db::retrieve\_apikey\_info()} & 1 & \textcolor{Green}{\textit{Superato}}\\
	\hline
	\textit{Transactions\_db::retrieve\_purchases\_list()} & 1 & \textcolor{Green}{\textit{Superato}}\\
	\hline
	\textit{Transactions\_db::apikey\_registration()} & 1 & \textcolor{Green}{\textit{Superato}}\\
	\hline
	\textit{Transactions\_db::purchase\_registration()} & 1 & \textcolor{Green}{\textit{Superato}}\\
	\hline
	\textit{Transactions\_db::apikey\_remaining\_update()} & 1 & \textcolor{Green}{\textit{Superato}}\\
	\hline
	\textit{SLA\_db::retrieve\_apikey\_slasurvey\_list()} & 1 & \textcolor{Green}{\textit{Superato}}\\
	\hline
	\textit{SLA\_db::retrieve\_ms\_slasurvey\_list()} & 1 & \textcolor{Green}{\textit{Superato}}\\
	\hline
	\textit{SLA\_db::retrieve\_slasurvey\_info()} & 1 & \textcolor{Green}{\textit{Superato}}\\
	\hline
	\textit{SLA\_db::retrieve\_slasurvey\_iscompliant()} & 1 & \textcolor{Green}{\textit{Superato}}\\
	\hline
	\textit{SLA\_db::slasurvey\_insert()} & 1 & \textcolor{Green}{\textit{Superato}}\\
	\hline
	\textit{FileHandler::setFile()} & 1 & \textcolor{Green}{\textit{Superato}}\\
	\hline
	\textit{FileHandler::fileNameExists()} & 1 & \textcolor{Green}{\textit{Superato}}\\
	\hline
	\textit{FrontEndMS::homepage\_ms\_list()} & 1 & \textcolor{Green}{\textit{Superato}}\\
	\hline
	\textit{FrontEndMS::homepage\_filter\_list()} & 1 & \textcolor{Green}{\textit{Superato}}\\
	\hline
	
	\caption{Resoconto esiti metrica Numero di parametri per metodo}
\end{longtable}

\subsubsection{Numero di attributi per classe}
La metrica \textit{Numero di attributi per classe} (\hyperlink{MPC10}{MPC10}), si applica per ogni classe progettata ed implementata.

\begin{longtable}{|>{\centering\arraybackslash}p{5cm}|>{\centering\arraybackslash}p{3cm}|>{\centering\arraybackslash}p{3cm}|}
	\hline
	\rowcolor{Gray}
	\textbf{Classe} & \textbf{Valore} & \textbf{Esito} \\
	\hline
	
	\textit{AppRouter} & 2 & \textcolor{Green}{\textit{Superato}}\\
	\hline
	\textit{AppRun} & 4 & \textcolor{Green}{\textit{Superato}}\\
	\hline
	\textit{ProfileManager} & 6 & \textcolor{Green}{\textit{Superato}}\\
	\hline
	\textit{API} & 1 & \textcolor{Green}{\textit{Superato}}\\
	\hline
	\textit{APIRegistered} & 5 & \textcolor{Green}{\textit{Superato}}\\
	\hline
	\textit{RegisterAPI} & 4 & \textcolor{Green}{\textit{Superato}}\\
	\hline
	\textit{SellingPolicy} & 2 & \textcolor{Green}{\textit{Superato}}\\
	\hline
	\textit{APIPurchased} & 4 & \textcolor{Green}{\textit{Superato}}\\
	\hline
	\textit{TransactionsList} & 1 & \textcolor{Green}{\textit{Superato}}\\
	\hline
	\textit{APIList} & 2 & \textcolor{Green}{\textit{Superato}}\\
	\hline
	\textit{RegisterUser} & 5 & \textcolor{Green}{\textit{Superato}}\\
	\hline
	\textit{AdminManager} & 5 & \textcolor{Green}{\textit{Superato}}\\
	\hline
	\textit{Login} & 4 & \textcolor{Green}{\textit{Superato}}\\
	\hline
	\textit{VirtualAccount} & 3 & \textcolor{Green}{\textit{Superato}}\\
	\hline
	\textit{PasswordRecovery} & 3 & \textcolor{Green}{\textit{Superato}}\\
	\hline
	\textit{ResetPassword} & 4 & \textcolor{Green}{\textit{Superato}}\\
	\hline
	\textit{AdminModeration} & 3 & \textcolor{Green}{\textit{Superato}}\\
	\hline
	\textit{UserDetailsModel} & 13 & \textcolor{Green}{\textit{Superato}}\\
	\hline
	\textit{TransactionModel} & 6 & \textcolor{Green}{\textit{Superato}}\\
	\hline
	\textit{MicroserviceModel} & 13 & \textcolor{Green}{\textit{Superato}}\\
	\hline
	\textit{SLAMicroserviceModel} & 6 & \textcolor{Green}{\textit{Superato}}\\
	\hline
	\textit{ProfileManagerController} & 5 & \textcolor{Green}{\textit{Superato}}\\
	\hline
	\textit{APIRegistrationController} & 3 & \textcolor{Green}{\textit{Superato}}\\
	\hline
	\textit{SellingPolicyController} & 3 & \textcolor{Green}{\textit{Superato}}\\
	\hline
	\textit{UserRegistrationController} & 4 & \textcolor{Green}{\textit{Superato}}\\
	\hline
	\textit{APIRegisteredController} & 4 & \textcolor{Green}{\textit{Superato}}\\
	\hline
	\textit{SearchController} & 5 & \textcolor{Green}{\textit{Superato}}\\
	\hline
	\textit{TransactionsListController} & 5 & \textcolor{Green}{\textit{Superato}}\\
	\hline
	\textit{AdminManagerController} & 5 & \textcolor{Green}{\textit{Superato}}\\
	\hline
	\textit{AdminModerationController} & 4 & \textcolor{Green}{\textit{Superato}}\\
	\hline
	\textit{LoginController} & 5 & \textcolor{Green}{\textit{Superato}}\\
	\hline
	\textit{VirtualAccountController} & 5 & \textcolor{Green}{\textit{Superato}}\\
	\hline
	\textit{PasswordRecoveryController} & 4 & \textcolor{Green}{\textit{Superato}}\\
	\hline
	\textit{ResetPasswordController} & 4 & \textcolor{Green}{\textit{Superato}}\\
	\hline
	\textit{APIPurchasedController} & 5 & \textcolor{Green}{\textit{Superato}}\\
	\hline
	\textit{Gateway} & 0 & \textcolor{Green}{\textit{Superato}}\\
	\hline
	\textit{ServiceInteractionHandler} & 0 & \textcolor{Green}{\textit{Superato}}\\
	\hline
	\textit{Microservices\_db} & 0 & \textcolor{Green}{\textit{Superato}}\\
	\hline
	\textit{Users\_db} & 0 & \textcolor{Green}{\textit{Superato}}\\
	\hline
	\textit{Transactions\_db} & 0 & \textcolor{Green}{\textit{Superato}}\\
	\hline
	\textit{SLA\_db} & 0 & \textcolor{Green}{\textit{Superato}}\\
	\hline
	\textit{FileHandler} & 0 & \textcolor{Green}{\textit{Superato}}\\
	\hline
	\textit{FrontEndMS} & 0 & \textcolor{Green}{\textit{Superato}}\\
	\hline
	
	\caption{Resoconto esiti metrica Numero di attributi per classe}
\end{longtable}

\subsection{Revisione di Accettazione}

\begin{longtable}{|>{\centering\arraybackslash}p{2cm}|>{\centering\arraybackslash}p{5cm}|>{\centering\arraybackslash}p{3cm}|>{\centering\arraybackslash}p{3cm}|}
	\hline
	\rowcolor{Gray}
	\textbf{Id} & \textbf{Metrica} & \textbf{Valore} & \textbf{Esito} \\
	\hline
	\hyperlink{MPC1}{MPC1} & \textit{Disponibilità \textit{NetBreakDB}} & 98\% & \textcolor{Green}{\textit{Superato}}\\
	\hline
	\hyperlink{MPC2}{MPC2} & \textit{Schedule Variance} & 0 & \textcolor{Green}{\textit{Superato}}\\
	\hline
	\hyperlink{MPC3}{MPC3} & \textit{Budget Variance} & xx & \textcolor{Green}{\textit{Superato}}\\
	\hline
	\hyperlink{MPC4}{MPC4} & \textit{Rischi non preventivati} & 1 & \textcolor{Green}{\textit{Superato}}\\
	\hline
	\hyperlink{MPC5}{MPC5} & \textit{Adempimento requisiti obbligatori} & 100\% & \textcolor{Green}{\textit{Superato}}\\
	\hline
	\hyperlink{MPC13}{MPC13} & \textit{Linee di commento} & 35\% & \textcolor{Green}{\textit{Superato}}\\
	\hline
	\hyperlink{MPC14}{MPC14} & \textit{Test di Unità} & 100\% & \textcolor{Green}{\textit{Superato}}\\
	\hline
	\hyperlink{MPC15}{MPC15} & \textit{Test di Integrazione} & 90\% & \textcolor{Green}{\textit{Superato}}\\
	\hline
	\hyperlink{MPC16}{MPC16} & \textit{Test di Sistema} & 85\% & \textcolor{Green}{\textit{Superato}}\\
	\hline
	\hyperlink{MPC17}{MPC17} & \textit{Test di Validazione} & 100\% & \textcolor{Green}{\textit{Superato}}\\
	\hline
	\hyperlink{MPC18}{MPC18} & \textit{Test superati} & 94\% & \textcolor{Green}{\textit{Superato}}\\
	\hline
	\hyperlink{MPC20}{MPC20} & \textit{Code Coverage} & 85\% & \textcolor{Green}{\textit{Superato}}\\
	\hline
	
	\caption{Resoconto esiti metriche - Qualità di processo RA}
\end{longtable}

\subsubsection{Fan In}
La metrica \textit{Fan In} (\hyperlink{MPC6}{MPC6}) si applica ad ogni modulo software per misurare e stabilire il livello di riuso implementato.\\
Per una consultazione più semplice e rapida, di seguito viene riportata una tabella indicante il valore minimo di \textit{Fan In} individuato tra tutti i moduli implementati.

\begin{longtable}{|>{\centering\arraybackslash}p{4cm}|>{\centering\arraybackslash}p{3cm}|}
	\hline
	\rowcolor{Gray}
	\textbf{Valore minimo} & \textbf{Esito} \\
	\hline
	
	2 & \textcolor{Green}{\textit{Superato}}\\
	\hline
	
	\caption{Resoconto esiti metrica Fan In - RA}
\end{longtable}

\subsubsection{Fan Out}
La metrica \textit{Fan Out} (\hyperlink{MPC7}{MPC7}) si applica ad ogni modulo software per misurare e stabilire il livello di accoppiamento implementato.\\
Per una consultazione più semplice e rapida, di seguito viene riportata una tabella indicante il valore massimo di \textit{Fan Out} individuato tra tutti i moduli implementati.

\begin{longtable}{|>{\centering\arraybackslash}p{4cm}|>{\centering\arraybackslash}p{3cm}|}
	\hline
	\rowcolor{Gray}
	\textbf{Valore massimo} & \textbf{Esito} \\
	\hline
	
	4 & \textcolor{Green}{\textit{Superato}}\\
	\hline
	
	\caption{Resoconto esiti metrica Fan Out - RA}
\end{longtable}

\subsubsection{Numero di metodi per classe}
La metrica \textit{Numero di metodi per classe} (\hyperlink{MPC8}{MPC8}) si applica ad ogni classe progettata ed implementata.\\
Per una consultazione più semplice e rapida, di seguito viene riportata una tabella indicante il valore minimo, massimo e medio di \textit{Numero di metodi per classe}, individuati tra tutte le classe implementate.\\
\textbf{N.B.:} nel caso in cui il valore massimo non rientri nel range di accettazione di tale metrica, verrà indicata la classe che necessita delle dovute correzioni.

\begin{longtable}{|>{\centering\arraybackslash}p{3cm}|>{\centering\arraybackslash}p{3cm}|>{\centering\arraybackslash}p{3cm}|>{\centering\arraybackslash}p{3cm}|}
	\hline
	\rowcolor{Gray}
	\textbf{Valore minimo} & \textbf{Valore massimo} & \textbf{Valore medio} & \textbf{Esito} \\
	\hline
	
	1 & 15 & 4 & \textcolor{Green}{\textit{Superato}}\\
	\hline
	
	\caption{Resoconto esiti metrica Numero di metodi per classe - RA}
\end{longtable}

\subsubsection{Numero di parametri per metodo}
La metrica \textit{Numero di parametri per metodo} (\hyperlink{MPC9}{MPC9}) si applica ad ogni metodo progettato ed implementato.\\
Per una consultazione più semplice e rapida, di seguito viene riportata una tabella indicante il valore minimo, massimo e medio di \textit{Numero di parametri per metodo}, individuati tra tutti i metodi implementati.\\
\textbf{N.B.:} nel caso in cui il valore massimo non rientri nel range di accettazione di tale metrica, verrà indicato il metodo (con relativa classe che lo contiene) che necessita delle dovute correzioni.

\begin{longtable}{|>{\centering\arraybackslash}p{3cm}|>{\centering\arraybackslash}p{3cm}|>{\centering\arraybackslash}p{3cm}|>{\centering\arraybackslash}p{3cm}|}
	\hline
	\rowcolor{Gray}
	\textbf{Valore minimo} & \textbf{Valore massimo} & \textbf{Valore medio} & \textbf{Esito} \\
	\hline
	
	0 & 5 & 2 & \textcolor{Green}{\textit{Superato}}\\
	\hline
	
	\caption{Resoconto esiti metrica Numero di parametri per metodo - RA}
\end{longtable}

\subsubsection{Numero di attributi per classe}
La metrica \textit{Numero di attributi per classe} (\hyperlink{MPC10}{MPC10}) si applica per ogni classe progettata ed implementata.\\
Per una consultazione più semplice e rapida, di seguito viene riportata una tabella indicante il valore minimo, massimo e medio di \textit{Numero di attributi per classe}, individuati tra tutte le classi implementate.\\
\textbf{N.B.:} nel caso in cui il valore massimo non rientri nel range di accettazione di tale metrica, verrà indicata la classe che necessita delle dovute correzioni.

\begin{longtable}{|>{\centering\arraybackslash}p{3cm}|>{\centering\arraybackslash}p{3cm}|>{\centering\arraybackslash}p{3cm}|>{\centering\arraybackslash}p{3cm}|}
	\hline
	\rowcolor{Gray}
	\textbf{Valore minimo} & \textbf{Valore massimo} & \textbf{Valore medio} & \textbf{Esito} \\
	\hline
	
	0 & 13 & 4 & \textcolor{Green}{\textit{Superato}}\\
	\hline
	
	\caption{Resoconto esiti metrica Numero di attributi per classe - RA}
\end{longtable}

\subsubsection{Complessità Ciclomatica}
La metrica \textit{Complessità Ciclomatica} (\hyperlink{MPC11}{MPC11}) si applica a metodi, funzioni, moduli, classi, etc..\\
Per una consultazione più semplice e rapida, di seguito viene riportata una tabella indicante il valore massimo di \textit{Complessità Ciclomatica}, individuato tra tutti/e i/le metodi, funzioni, moduli e classi implementati/e.\\
\textbf{N.B.:} nel caso in cui il valore massimo non rientri nel range di accettazione di tale metrica, verrà indicato/a il metodo/funzione/classe/modulo che necessita delle dovute correzioni.

\begin{longtable}{|>{\centering\arraybackslash}p{4cm}|>{\centering\arraybackslash}p{3cm}|}
	\hline
	\rowcolor{Gray}
	\textbf{Valore massimo} & \textbf{Esito} \\
	\hline

	10 & \textcolor{Green}{\textit{Superato}}\\
	\hline

	\caption{Resoconto esiti metrica Complessità Ciclomatica - RA}
\end{longtable}

\subsubsection{Numero di livelli di annidamento}
La metrica \textit{Numero di livelli di annidamento} (\hyperlink{MPC12}{MPC12}) si applica ad ogni metodo implementato.\\
Per una consultazione più semplice e rapida, di seguito viene riportata una tabella indicante il valore massimo di \textit{Numero di livelli di annidamento}, individuato tra tutti i metodi implementati.\\
\textbf{N.B.:} nel caso in cui il valore massimo non rientri nel range di accettazione di tale metrica, verrà indicato il metodo (con relativa classe che lo contiene) che necessita delle dovute correzioni.

\begin{longtable}{|>{\centering\arraybackslash}p{4cm}|>{\centering\arraybackslash}p{3cm}|}
	\hline
	\rowcolor{Gray}
	\textbf{Valore massimo} & \textbf{Esito} \\
	\hline
	
	6 & \textcolor{Green}{\textit{Superato}}\\
	\hline
	
	\caption{Resoconto esiti metrica Numero di livelli di annidamento - RA}
\end{longtable}