\newpage
\section{Introduzione}

\subsection{Scopo del documento}
Questo documento descrive le scelte e le strategie attuate per permettere di raggiungere determinati obiettivi di qualità misurabili. A questo scopo, sarà necessario un continuo processo di verifica, orientato ad individuare e correggere errori ed eventuali sprechi di risorse.
Per conseguire dei risultati concreti, il processo di verifica dovrà fornire dei dati quantificabili per poter valutare se gli obiettivi sono stati raggiunti o meno. Per facilitarne la valutazione, per ogni metrica sarannno indicati due range:
\begin{itemize}
	\item \textbf{Range accettazione:} rappresenta l'intervallo di valori minimi richiesti per il raggiungimento degli obiettivi di qualità definiti;
	\item \textbf{Range ottimale:} rappresenta l'intervallo di valori desiderati, entro cui dovrebbe collocarsi la misurazione. Nel caso in cui non si rientrasse in questo range, sarà necessario effettuare una verifica più accurata, al fine di individuarne le cause e poter applicare le dovute correzioni.
\end{itemize}

\subsection{Scopo del prodotto}
Lo scopo del prodotto è la realizzazione di un \textit{API Market\ped{G}} per l'acquisto e la vendita di \textit{microservizi\ped{G}}. Il sistema offrirà la possibilità di registrare nuove \textit{API\ped{G}} per la vendita, permetterà la consultazione e la ricerca di API ai potenziali acquirenti, gestendo i permessi di accesso ed utilizzo tramite creazione e controllo di relative \textit{API key\ped{G}}. Il sistema, oltre alla web app stessa, sarà corredato di un \textit{API Gateway\ped{G}} per la gestione delle richieste e il controllo delle chiavi, e fornirà funzionalità avanzate di statistiche per il gestore della piattaforma e per i fornitori dei microservizi.

\subsection{Riferimenti normativi}
\begin{itemize}
\item \textsc{NormeDiProgetto 4\_0\_0.pdf};
\item \textbf{Capitolato d’appalto C1:} APIM: An API Market Platform\\ \url{http://www.math.unipd.it/~tullio/IS-1/2016/Progetto/C1.pdf};
\end{itemize}

\subsection{Riferimenti informativi}
\begin{itemize}
	\item \textsc{PianoDiProgetto 4\_0\_0.pdf};
	\item \textbf{Slide del corso riguardo la qualità di prodotto}\\ \url{http://www.math.unipd.it/~tullio/IS-1/2016/Dispense/L10.pdf};
	\item \textbf{Slide del corso riguardo la qualità di processo}\\ \url{http://www.math.unipd.it/~tullio/IS-1/2016/Dispense/L11.pdf};
	\item \textbf{Standard ISO/IEC 12207:2008}\\ \url{https://www.iso.org/obp/ui/#iso:std:iso-iec:12207:ed-2:v1:en};
	\item \textbf{Standard ISO 9001}\\ \url{https://www.iso.org/iso-9001-quality-management.html};
	\item \textbf{Standard ISO/IEC 9126:2001}\\ \url{https://en.wikipedia.org/wiki/ISO/IEC_9126};
	\item \textbf{Standard ISO/IEC 15504}\\ \url{https://en.wikipedia.org/wiki/ISO/IEC_15504};
	\item \textbf{Indice Gulpease}\\ \url{https://it.wikipedia.org/wiki/Indice_Gulpease};	
\end{itemize}

\subsection{Glossario}
Per semplificare la consultazione e disambiguare alcune terminologie tecniche, le voci indicate con la lettera \textit{G} a pedice sono descritte approfonditamente nel documento \textsc{Glossario 3\_0\_0.pdf} e specificate solo alla prima occorrenza all'interno del suddetto documento.