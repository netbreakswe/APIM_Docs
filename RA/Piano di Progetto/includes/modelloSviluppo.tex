\newpage
\section{Modello di sviluppo}
Il modello di ciclo di vita scelto è il \textbf{Modello Incrementale}. Esso presenta le seguenti caratteristiche:
\begin{itemize}
	\item Le attività di Analisi e Progettazione Architetturale non vengono ripetute: i requisiti principali e l'architettura del sistema vengono identificati e fissati definitivamente, dando modo di pianificare i cicli di incremento;
	\item L'ordine di implementazione delle diverse parti del sistema è determinato nelle fasi preliminari di Progettazione;
	\item Le attività di Progettazione di Dettaglio, Codifica e Verifica sono, invece, ripetute più volte, al fine di migliorare parti del sistema già esistenti o aggiungere nuove funzionalità per soddisfare man mano tutti i requisiti.
	\item La Manutenzione è un'attività di evoluzione continua, che ha lo scopo di rendere completo il prodotto.
\end{itemize}
I vantaggi che porta l'adozione di questo modello sono:
\begin{itemize}
	\item I requisiti utente vengono trattati in base alla loro importanza strategica, ovvero prima si parte da quelli primari;
	\item Ogni incremento può creare valore, arricchendo di funzionalità il prototipo di prodotto in via di sviluppo;
	\item Ogni incremento riduce il rischio di fallimento, poichè viene sfruttata la base consolidata dai vari incrementi effettuati nelle versioni precedenti del prototipo, grazie ai feedback ricevuti; 
	\item \MakeUppercase{è} possibile eseguire dei test più dettagliati, con risultati più soddisfacenti;
	\item Sono previsti rilasci multipli e in successione di prototipi di prodotto via via sempre più completi e conformi ai requisiti richiesti, che consentono al proponente di dare delle valutazioni al lavoro in fase di produzione.
\end{itemize}