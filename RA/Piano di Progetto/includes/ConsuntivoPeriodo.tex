\newpage

\section{Consuntivi di periodo}

In questa porzione del documento verrà analizzato lo scostamento tra il preventivo calcolato e le ore effettive impiegate, sia in termini puramente di tempistiche, che di preventivo vero e proprio. Si potrà avere, dunque, uno scostamento in \textbf{Positivo}, qualora le ore impiegate siano minori delle ore preventivate, o in \textbf{Pari}, se il preventivo risulta corretto, oppure in \textbf{Negativo}, se la pianificazione per un determinato periodo è stata sottostimata.

\subsection{Analisi dei Requisiti}

Di seguito, si analizzano le ore realmente impiegate per ciascun componente, con l'eventuale scostamento in positivo o negativo indicato tra parentesi.

\begin{table}[H]
	\begin{center}
		\begin{tabular}{|c|c|c|c|c|c|c|c|}
			\hline
			\textbf{Nome} & \multicolumn{6}{c|}{\textbf{Ore per ruolo}} & \textbf{Totale} \\\cline{2-7}
			& \textbf{Resp} & \textbf{Amm} & \textbf{An} & \textbf{Proj} & \textbf{Prog} & \textbf{Ver} & \\
			\hline
			\MC			&		&		&	16	&		&		&	15 (-1)	&	31 (-1)	\\
			\hline
			\AN			&		&	3	&	6	&	 	&		&	22 (-2)	& 	31 (-2)	\\
			\hline
			\DAN		&		&	3	&	29 	&		&		&		&	32	\\
			\hline
			\AS			&	20	&	 	&	12 (+1) 	&		&	 	& 		&	32 (+1)	\\
			\hline
			\NS 		&	18 (-1)	&	3	&	11	&		&		& 		&	32 (-1)	\\
			\hline
			\DS			& 		&	2	&	5	&		&		&	24 (-2)	&	31 (-2)	\\
			\hline
			\textbf{Totali per ruolo}	& 	38 (-1)	&	11	&	79 (+1)	&		&		&	61 (-5)	&	189 (-5)	\\
			\hline
			\textbf{Scostamento economico}	& 	+30€	&		&	-25€	&		&		&	+75€	&	+80€	\\
			\hline
		\end{tabular}
	\end{center}
	\caption{Scostamento ore per componente, \AdR}
\end{table}

\subsubsection{Considerazioni}
Nel periodo di Analisi dei Requisiti, il team si è dovuto scontrare con la realtà di un progetto totalmente innovativo per ciascuno dei membri. In particolare, seppur le tempistiche previste tenessero conto di un errore di approssimazione, l'analisi dei requisiti necessari ha richiesto del tempo aggiuntivo. Il team, infatti, ha dovuto passare al vaglio le funzionalità da offrire in un marketplace, congiuntamente ai vincoli del proponente e alla richiesta che il progetto venisse realizzato con architettura a microservizi. L'attività di verifica, al contrario, è risultata sovrastimata, pertanto si è preferito concentrare le ore in ruoli differenti.

\subsubsection{Prospetto economico}
Come si evince, l'impegno di alcune ore in meno rispetto a quanto preventivato, per quanto concerne specialmente l'attività di Verifica, ha prodotto uno scostamento in positivo del preventivo di € 80. Tale costo è ininfluente ai fini del preventivo a finire, in quando il costo dell'\textit{Analisi dei Requisiti}\ non è a carico del \textit{Proponente}.


\newpage
\subsection{Analisi dei Requisiti Dettagliata}

Di seguito, vengono analizzate le ore realmente impiegate per ciascun componente, in relazione al periodo di Analisi dei Requisiti Dettagliata.

\begin{table}[H]
	\begin{center}
		\begin{tabular}{|c|c|c|c|c|c|c|c|}
			\hline
			\textbf{Nome} & \multicolumn{6}{c|}{\textbf{Ore per ruolo}} & \textbf{Totale} \\\cline{2-7}
			& \textbf{Resp} & \textbf{Amm} & \textbf{An} & \textbf{Proj} & \textbf{Prog} & \textbf{Ver} & \\
			\hline
			\MC			&		&	1	&	 	&		&		&	2 	&	 3	\\
			\hline
			\AN			&		&		&	3 (-1) 	&	 	&		&	 	& 	 3 (-1)	\\
			\hline
			\DAN		&		&	 	&	2 	&		&		&		&	 2	\\
			\hline
			\AS			&	2	&	 	&	  	&		&	 	& 		&	 2	\\
			\hline
			\NS 		&	2	&		&	 	&		&		& 		&	 2	\\
			\hline
			\DS			& 		&	 	&	 	&		&		&	3 	&	 3	\\
			\hline
			\textbf{Totali per ruolo}	& 	4	&	1	&	5 (-1)	&		&		&	5	&	15 (-1)	\\
			\hline
			\textbf{Scostamento economico}	& 		&		&	+25€	&		&		&	&	+25€	\\
			\hline
		\end{tabular}
	\end{center}
	\caption{Scostamento ore per componente, \ARD}
\end{table}

\subsubsection{Considerazioni}
Forti dell'analisi supplementare svolta nel periodo antecedente alla Revisione dei Requisiti, il team ha avuto un risparmio in sede di analisi e l'attuazione dei risultati emersi dalla Revisione ha richiesto meno tempo del previsto. Non sono state intraprese misure correttive particolari, in quanto durante l'Analisi dei Requisiti Dettagliata il team registrava un consuntivo di ore impiegate inferiore a quelle preventivate.

\subsubsection{Prospetto economico}
Il periodo dell'Analisi dei Requisiti Dettagliata ha prodotto un risparmio minimo sul costo preventivato. Il positivo di € 25 per questo periodo, è un risparmio del team, in quanto il costo non è a carico del \textit{Proponente}\ e quindi questo risparmio non va ad intaccare il preventivo a finire.

\newpage
\subsection{Progettazione Architetturale}

Di seguito, si analizzano le ore realmente impiegate per ciascun componente, in relazione al periodo di Progettazione Architetturale.

\begin{table}[H]
	\begin{center}
		\begin{tabular}{|c|c|c|c|c|c|c|c|}
			\hline
			\textbf{Nome} & \multicolumn{6}{c|}{\textbf{Ore per ruolo}} & \textbf{Totale} \\\cline{2-7}
			& \textbf{Resp} & \textbf{Amm} & \textbf{An} & \textbf{Proj} & \textbf{Prog} & \textbf{Ver} & \\
			\hline
			\MC			&		&	3 (-1)	&		&	31		&		&		&   33	(-1)\\
			\hline
			\AN			&3 (-1)	&			&		&	31		&		&		& 	34 (-1)	\\
			\hline
			\DAN		&		&	2		&		&	17		&		&	14	&	33	\\
			\hline
			\AS			&		&		 	&	 	&	13 (+1)	&	 	& 	19	&	32 (+1)	\\
			\hline
			\NS 		&		&			&		&	14 (+2)	&		& 	18 (-1)	&	32 (+1)	\\
			\hline
			\DS			& 	2	&			&		&	30 (-1)	&		&		&	32 (-1)	\\
			\hline
			\textbf{Totali per ruolo}	& 	5 (-1)	&	5 (-1)	&		&	137 (+2)	&		&	50 (-1)	&	197 (-1)	\\
			\hline
			\textbf{Scostamento economico}	& 	+30€	&	+20€	&		&	-50€	&		&	+15€	&	+15€	\\
			\hline
		\end{tabular}
	\end{center}
	\caption{Scostamento ore per componente, Progettazione Architetturale}
\end{table}

\subsubsection{Prospetto economico}
Durante la Progettazione Architetturale, il team si è affacciato al mondo dei microservizi per l'aspetto progettuale. Tale nuova metodologia di lavoro ha causato un leggero ritardo dei lavori per quanto concerne tale attività. Il consuntivo delle ore risulta tuttavia ancora in parità, valutando il quadro generale. Il numero di ore in verifica preventivato è risultato più adeguato del periodo precedente.

\subsubsection{Considerazioni}
L'attività di progettazione ha causato diversi problemi e quindi un numero maggiore di ore impiegate in Progettazione. A causa di alcune incomprensioni progettuali e necessità di approfondire alcune situazioni critiche con il Proponente, si è verifcato uno scostamento in negativo di € 50 nella progettazione, ma la gestione del gruppo e del progetto ha portato a una riduzione delle ore del Responsabile, dell'Amministratore e del Verificatore.

\newpage
\subsection{Progettazione Architetturale Dettagliata}

Di seguito, si analizzano le ore realmente impiegate per ciascun componente, in relazione al periodo di Progettazione Architetturale Dettagliata.

\begin{table}[H]
	\begin{center}
		\begin{tabular}{|c|c|c|c|c|c|c|c|}
			\hline
			\textbf{Nome} & \multicolumn{6}{c|}{\textbf{Ore per ruolo}} & \textbf{Totale} \\\cline{2-7}
			& \textbf{Resp} & \textbf{Amm} & \textbf{An} & \textbf{Proj} & \textbf{Prog} & \textbf{Ver} & \\
			\hline
			\MC			&	2	&		&		&	6	&		&	11	&	19	\\
			\hline
			\AN			&		&		&		&	8	&   	&	11	& 	19	\\
			\hline
			\DAN		&	3	&		&		&	17	&		&		&	20	\\
			\hline
			\AS			&		&	3	&	 	&	17	&	 	& 		&	20	\\
			\hline
			\NS 		&		&	3	&		&	17 (+1)	&		& 		&	20 (+1)	\\
			\hline
			\DS			& 		&		&		&	7 (+2)	&		&	13	&	20 (+2)	\\
			\hline
			\textbf{Totali per ruolo}	& 	5 	&	6 	&		&	72 (+3)	&		&	50 	&	131 (+3)	\\
			\hline
			\textbf{Scostamento economico}	& 		&		&		&	-75€	&		&		&	-75€	\\
			\hline
		\end{tabular}
	\end{center}
	\caption{Scostamento ore per componente, Progettazione Architetturale Dettagliata}
\end{table}


\subsubsection{Considerazioni}
Anche nella Progettazione Architetturale Dettagliata, il team NetBreak si è scontrato con ostiche e nuove tematiche. Per ottimizzare l'approccio al problema, è stata intensificata la comunicazione con il Proponente e con il team esterno che si è aggiudicato il medesimo capitolato.

\subsubsection{Prospetto economico}
L'attività di progettazione dettagliata ha visto diverse ulteriori lacune da colmare, anche in seguito alle segnalazioni emerse in sede di Revisione di Progettazione. Ciò ha richiesto di conseguenza un numero maggiore di ore impiegate in Progettazione, anche in questo caso. Si registra dunque in questa sede uno scostamento in negativo di € 75, mantenendo invariati gli altri ruoli.

\newpage
\subsection{Codifica}

Di seguito, si analizzano le ore realmente impiegate per ciascun componente, per quanto concerne il periodo di Codifica.

\begin{table}[H]
	\begin{center}
		\begin{tabular}{|c|c|c|c|c|c|c|c|}
			\hline
			\textbf{Nome} & \multicolumn{6}{c|}{\textbf{Ore per ruolo}} & \textbf{Totale} \\\cline{2-7}
			& \textbf{Resp} & \textbf{Amm} & \textbf{An} & \textbf{Proj} & \textbf{Prog} & \textbf{Ver} & \\
			\hline
			\MC			&		&		&		&		&	18	&	18	&	36	\\
			\hline
			\AN			&	3	&		&		&	 	&	32	&		& 	35	\\
			\hline
			\DAN		&		&		&		&		&	14	&	21 (-1)	&	35 (-1)	\\
			\hline
			\AS			&		&	 	&	 	&		&	14 	& 	22 (-1)	&	36	(-1)\\
			\hline
			\NS 		&		&	3	&		&		&	32	& 		&	35	\\
			\hline
			\DS			& 	3	&		&		&		&	33	&		&	36	\\
			\hline
			\textbf{Totali per ruolo}	& 	6 	&	3 	&		&		&	143	&	50 (-2) 	&	202 (-2)	\\
			\hline
			\textbf{Scostamento economico}	& 		&		&		&		&		&	+30€	&	+30€	\\
			\hline
		\end{tabular}
	\end{center}
	\caption{Scostamento ore per componente, Periodo di Codifica}
\end{table}


\subsubsection{Considerazioni}
Le ore di lavoro preventivate in codifica, previste con ampio margine, sono state completamente utilizzate. Il lavoro è stato suddiviso tenendo presente le abilità di ciascun membro del gruppo ed evitare quindi più lunghi periodi di auto-formazione.

\subsubsection{Prospetto economico}
Durante l'attività di Codifica tutto si è svolto in maniera relativamente ordinaria, permettendoci di rilevare a seguito della consuntivazione un risparmio di 2 ore nell'attività di verifica del codice prodotto. 

\newpage
\subsection{Verifica e Validazione}

Di seguito, si analizzano le ore realmente impiegate per ciascun componente, per quanto concerne il periodo di Codifica.

\begin{table}[H]
	\begin{center}
		\begin{tabular}{|c|c|c|c|c|c|c|c|}
			\hline
			\textbf{Nome} & \multicolumn{6}{c|}{\textbf{Ore per ruolo}} & \textbf{Ore totali} \\\cline{2-7}
			& \textbf{Resp} & \textbf{Amm} & \textbf{An} & \textbf{Proj} & \textbf{Prog} & \textbf{Ver} & \\
			\hline
			\MC			&		&	3	&		&	7	&	(+2)	&	6	&	16 (+2)	\\
			\hline
			\AN			&		&		&		&	 	&	(+2)	&	17	& 	17	(+2)\\
			\hline
			\DAN		&	3	&		&		&		&		&	14	&	17	\\
			\hline
			\AS			&		&	 	&	 	&	4	&	 	& 	12	&	16	\\
			\hline
			\NS 		&		&		&		&	4	&		& 	13	&	17	\\
			\hline
			\DS			& 		&		&		&		&		&	16	&	16	\\
			\hline
			\textbf{Totali per ruolo}	& 	3	&	3	&		&  15	&	(+4)	&	78	&	99 (+4)	\\
			\hline
			\textbf{Scostamento economico}	& 		&		&		&		&	-60€	&		&	-60€	\\
			\hline
		\end{tabular}
	\end{center}
	\caption{Scostamento ore per componente, Periodo di Verifica}
\end{table}

\subsubsection{Considerazioni}
Durante il periodo di Verifica e Validazione è risultato necessario attuare interventi correttivi e migliorativi su parte del codice. Tale attività ha richiesto un tempo aggiuntivo non inizialmente preventivato. Pur essendo in possesso di un modesto surplus, tale incombenza ha, di fatto, parificato le ore risparmiate durante i periodi precedenti.

\subsubsection{Prospetto economico}
Durante l'attività di Verifica e Validazione, l'eccezione ha riguardato solamente una breve attività di Codifica per migliorie e incrementi di funzionalità pregresse, e non finalizzate. Come considerato nella sezione soprastante, il margine creatoci nei precedenti periodi si appiana. Il team, ad ogni modo, ha portato a termine il progetto con tempistiche e costi ben allineati (e con previsione lievemente ottimistica) rispetto a quanto ipotizzato. 

\newpage
\subsection{Consuntivo finale}

Analizzando i dati ottenuti, si ottiene un consuntivo parziale per il periodo rendicontato. Esso è consultabile nella tabella sottostante:

\begin{table}[H]
	\begin{center}
		\begin{tabular}{|c|c|c|}
			\hline
			\textbf{Periodo analizzato}	& \textbf{Scostamento ore}	& \textbf{Scostamento preventivo} \\
			\hline
			\AdR	&	-5	&	+ 80 €	\\
			\hline
			\ARD	&	-1	&	+ 25€	\\
			\hline
			\PA   &		-1  &	+ 15 €	\\
			\hline
			\PD   &	+3  &	- 75 €	\\
			\hline
			\CO   &		-2  &	+ 30 €	\\
			\hline
			\VV   &		+4  &	- 60 €	\\
			\hline
			\textbf{Totale} & \textbf{-2} & \textbf{+15 €} \\
			\hline
		\end{tabular}
	\end{center}
	\caption{Scostamento ore complessivo}
\end{table}

Il conteggio finale, in data attuale, mostra come le previsioni, nonostante gli imprevisti nel periodo di Progettazione Architetturale e Progettazione Architetturale Dettagliata e del periodo di Verifica e Validazione, risultano in linea con le aspettative. Il saldo parziale risulta, infatti, in attivo di € 15: questo dato, seppur di moderata entità, ha l'importanza di mostrare come le stime effettuate siano corrette (per il periodo rendicontato) ai fini del calcolo di un preventivo e come le stime in difetto o in eccesso nei vari periodi si siano poi stabilizzate al valore ipotizzato inizialmente.

