\newpage

\section{Consuntivi di periodo}

In questa porzione del documento verrà analizzato lo scostamento tra il preventivo calcolato e le ore effettive impiegate, sia in termini puramente di tempistiche, che di preventivo vero e proprio. Si potrà avere, dunque, uno scostamento in \textbf{Positivo}, qualora le ore impiegate siano minori delle ore preventivate, o in \textbf{Pari}, se il preventivo risulta corretto, oppure in \textbf{Negativo}, se la pianificazione per un determinato periodo è stata sottostimata.

\subsection{Analisi dei Requisiti}

Di seguito, si analizzano le ore realmente impiegate per ciascun componente, con l'eventuale scostamento in positivo o negativo indicato tra parentesi.

\begin{table}[H]
	\begin{center}
		\begin{tabular}{|c|c|c|c|c|c|c|c|}
			\hline
			\textbf{Nome} & \multicolumn{6}{c|}{\textbf{Ore per ruolo}} & \textbf{Totale} \\\cline{2-7}
			& \textbf{Resp} & \textbf{Amm} & \textbf{An} & \textbf{Proj} & \textbf{Prog} & \textbf{Ver} & \\
			\hline
			\MC			&		&		&	16	&		&		&	15 (-1)	&	31 (-1)	\\
			\hline
			\AN			&		&	3	&	6	&	 	&		&	22 (-2)	& 	31 (-2)	\\
			\hline
			\DAN		&		&	3	&	29 	&		&		&		&	32	\\
			\hline
			\AS			&	20	&	 	&	12 (+1) 	&		&	 	& 		&	32 (+1)	\\
			\hline
			\NS 		&	18 (-1)	&	3	&	11	&		&		& 		&	32 (-1)	\\
			\hline
			\DS			& 		&	2	&	5	&		&		&	24 (-2)	&	31 (-2)	\\
			\hline
			\textbf{Totali per ruolo}	& 	38 (-1)	&	11	&	79 (+1)	&		&		&	61 (-5)	&	189 (-5)	\\
			\hline
			\textbf{Scostamento preventivo}	& 	+30€	&		&	-25€	&		&		&	+75€	&	+80€	\\
			\hline
		\end{tabular}
	\end{center}
	\caption{Scostamento ore per componente, \AdR}
\end{table}

\subsubsection{Considerazioni}
Come si evince, l'impegno di alcune ore in meno rispetto a quanto preventivato, per quanto concerne specialmente l'attività di Verifica, ha prodotto uno scostamento in positivo del preventivo di € 80. Tale costo è ininfluente ai fini del preventivo a finire, in quando il costo dell'\textit{Analisi dei Requisiti}\ non è a carico del \textit{Proponente}.


\newpage
\subsection{Analisi dei Requisiti Dettagliata}

Di seguito, vengono analizzate le ore realmente impiegate per ciascun componente, in relazione al periodo di Analisi dei Requisiti Dettagliata.

\begin{table}[H]
	\begin{center}
		\begin{tabular}{|c|c|c|c|c|c|c|c|}
			\hline
			\textbf{Nome} & \multicolumn{6}{c|}{\textbf{Ore per ruolo}} & \textbf{Totale} \\\cline{2-7}
			& \textbf{Resp} & \textbf{Amm} & \textbf{An} & \textbf{Proj} & \textbf{Prog} & \textbf{Ver} & \\
			\hline
			\MC			&		&	1	&	 	&		&		&	2 	&	 3	\\
			\hline
			\AN			&		&		&	3 (-1) 	&	 	&		&	 	& 	 3 (-1)	\\
			\hline
			\DAN		&		&	 	&	2 	&		&		&		&	 2	\\
			\hline
			\AS			&	2	&	 	&	  	&		&	 	& 		&	 2	\\
			\hline
			\NS 		&	2	&		&	 	&		&		& 		&	 2	\\
			\hline
			\DS			& 		&	 	&	 	&		&		&	3 	&	 3	\\
			\hline
			\textbf{Totali per ruolo}	& 	4	&	1	&	5 (-1)	&		&		&	5	&	15 (-1)	\\
			\hline
			\textbf{Scostamento preventivo}	& 		&		&	+25€	&		&		&	&	+25€	\\
			\hline
		\end{tabular}
	\end{center}
	\caption{Scostamento ore per componente, \ARD}
\end{table}

\subsubsection{Considerazioni}
Il periodo dell'Analisi dei Requisiti Dettagliata ha prodotto un risparmio minimo sul costo preventivato. Il positivo di € 25 per questo periodo, è un risparmio del team, in quanto il costo non è a carico del \textit{Proponente}\ e quindi questo risparmio non va ad intaccare il preventivo a finire.

\newpage
\subsection{Progettazione Architetturale}

Di seguito, si analizzano le ore realmente impiegate per ciascun componente, in relazione al periodo di Progettazione Architetturale.

\begin{table}[H]
	\begin{center}
		\begin{tabular}{|c|c|c|c|c|c|c|c|}
			\hline
			\textbf{Nome} & \multicolumn{6}{c|}{\textbf{Ore per ruolo}} & \textbf{Totale} \\\cline{2-7}
			& \textbf{Resp} & \textbf{Amm} & \textbf{An} & \textbf{Proj} & \textbf{Prog} & \textbf{Ver} & \\
			\hline
			\MC			&		&	3 (-1)	&		&	31		&		&		&   33	(-1)\\
			\hline
			\AN			&3 (-1)	&			&		&	31		&		&		& 	34 (-1)	\\
			\hline
			\DAN		&		&	2		&		&	17		&		&	14	&	33	\\
			\hline
			\AS			&		&		 	&	 	&	13 (+1)	&	 	& 	19	&	32 (+1)	\\
			\hline
			\NS 		&		&			&		&	14 (+2)	&		& 	18 (-1)	&	32 (+1)	\\
			\hline
			\DS			& 	2	&			&		&	30 (-1)	&		&		&	32 (-1)	\\
			\hline
			\textbf{Totali per ruolo}	& 	5 (-1)	&	5 (-1)	&		&	137 (+2)	&		&	50 (-1)	&	197 (-1)	\\
			\hline
			\textbf{Scostamento preventivo}	& 	+30€	&	+20€	&		&	-50€	&		&	+15€	&	+15€	\\
			\hline
		\end{tabular}
	\end{center}
	\caption{Scostamento ore per componente, Progettazione Architetturale}
\end{table}

\subsubsection{Considerazioni}
L'attività di progettazione ha causato diversi problemi e quindi un numero maggiore di ore impiegate in Progettazione. A causa di alcune incomprensioni progettuali e necessità di approfondire alcune situazioni critiche con il Proponente, si è verifcato uno scostamento in negativo di € 50 nella progettazione, ma la gestione del gruppo e del progetto ha portato a una riduzione delle ore del Responsabile, dell'Amministratore e del Verificatore.

\newpage
\subsection{Progettazione Architetturale Dettagliata}

Di seguito, si analizzano le ore realmente impiegate per ciascun componente, in relazione al periodo di Progettazione Architetturale Dettagliata.

\begin{table}[H]
	\begin{center}
		\begin{tabular}{|c|c|c|c|c|c|c|c|}
			\hline
			\textbf{Nome} & \multicolumn{6}{c|}{\textbf{Ore per ruolo}} & \textbf{Totale} \\\cline{2-7}
			& \textbf{Resp} & \textbf{Amm} & \textbf{An} & \textbf{Proj} & \textbf{Prog} & \textbf{Ver} & \\
			\hline
			\MC			&	2	&		&		&	6	&		&	11	&	19	\\
			\hline
			\AN			&		&		&		&	8	&   	&	11	& 	19	\\
			\hline
			\DAN		&	3	&		&		&	17	&		&		&	20	\\
			\hline
			\AS			&		&	3	&	 	&	17	&	 	& 		&	20	\\
			\hline
			\NS 		&		&	3	&		&	17 (+1)	&		& 		&	20 (+1)	\\
			\hline
			\DS			& 		&		&		&	7 (+2)	&		&	13	&	20 (+2)	\\
			\hline
			\textbf{Totali per ruolo}	& 	5 	&	6 	&		&	72 (+3)	&		&	50 	&	131 (+3)	\\
			\hline
			\textbf{Scostamento preventivo}	& 		&		&		&	-75€	&		&		&	-75€	\\
			\hline
		\end{tabular}
	\end{center}
	\caption{Scostamento ore per componente, Progettazione Architetturale Dettagliata}
\end{table}


\subsubsection{Considerazioni}
L'attività di progettazione dettagliata ha visto diverse ulteriori lacune da colmare, anche in seguito alle segnalazioni emerse in sede di Revisione di Progettazione. Ciò ha richiesto di conseguenza un numero maggiore di ore impiegate in Progettazione, anche in questo caso. Si registra dunque in questa sede uno scostamento in negativo di € 75, mantenendo invariati gli altri ruoli

\newpage
\subsection{Codifica}

Di seguito, si analizzano le ore realmente impiegate per ciascun componente, per quanto concerne il periodo di Codifica.

\begin{table}[H]
	\begin{center}
		\begin{tabular}{|c|c|c|c|c|c|c|c|}
			\hline
			\textbf{Nome} & \multicolumn{6}{c|}{\textbf{Ore per ruolo}} & \textbf{Totale} \\\cline{2-7}
			& \textbf{Resp} & \textbf{Amm} & \textbf{An} & \textbf{Proj} & \textbf{Prog} & \textbf{Ver} & \\
			\hline
			\MC			&		&		&		&		&	18	&	18	&	36	\\
			\hline
			\AN			&	3	&		&		&	 	&	32	&		& 	35	\\
			\hline
			\DAN		&		&		&		&		&	14	&	21 (-1)	&	35 (-1)	\\
			\hline
			\AS			&		&	 	&	 	&		&	14 	& 	22 (-1)	&	36	(-1)\\
			\hline
			\NS 		&		&	3	&		&		&	32	& 		&	35	\\
			\hline
			\DS			& 	3	&		&		&		&	33	&		&	36	\\
			\hline
			\textbf{Totali per ruolo}	& 	6 	&	3 	&		&		&	143	&	50 (-2) 	&	202 (-2)	\\
			\hline
			\textbf{Scostamento preventivo}	& 		&		&		&		&		&	+30€	&	+30€	\\
			\hline
		\end{tabular}
	\end{center}
	\caption{Scostamento ore per componente, Periodo di Codifica}
\end{table}

\subsubsection{Considerazioni}
Durante l'attività di Codifica tutto si è svolto in maniera relativamente ordinaria, permettendoci di rilevare a seguito della consuntivazione un risparmio di 2 ore nell'attività di verifica del codice prodotto. 

\newpage
\subsection{Consuntivo parziale}

Analizzando i dati ottenuti, si ottiene un consuntivo parziale per il periodo rendicontato. Esso è consultabile nella tabella sottostante:

\begin{table}[H]
	\begin{center}
		\begin{tabular}{|c|c|c|}
			\hline
			\textbf{Periodo analizzato}	& \textbf{Scostamento ore}	& \textbf{Scostamento preventivo} \\
			\hline
			\AdR	&	-5	&	+ 80 €	\\
			\hline
			\ARD	&	-1	&	+ 25€	\\
			\hline
			\PA   &		-1  &	+ 15 €	\\
			\hline
			\PD   &	+3  &	- 75 €	\\
			\hline
			\CO   &		-2  &	+ 30 €	\\
			\hline
			\textbf{Totale} & \textbf{-6} & \textbf{+75 €} \\
			\hline
		\end{tabular}
	\end{center}
	\caption{Scostamento ore parziale}
\end{table}

Il conteggio finale, in data attuale, mostra come le previsioni, nonostante gli imprevisti nel periodo di Progettazione Architetturale e Progettazione Architetturale Dettagliata, risultano in linea con le aspettative. Il saldo parziale risulta, infatti, in attivo di € 75: questo dato, seppur di moderata entità, ha l'importanza di mostrare come le stime effettuate siano corrette (per il periodo rendicontato) ai fini del calcolo di un preventivo.

\paragraph{Preventivo a finire}
Il risparmio è minimo, ma un bilancio attivo in questa fase e nel prossimo periodo di Progettazione Architetturale Dettagliata, permetterà di poter investire la plusvalenza nell'aumento di ore di Codifica permettendo di migliorare la qualità del prodotto in ingresso alla \textit{Revisione di Accettazione.}