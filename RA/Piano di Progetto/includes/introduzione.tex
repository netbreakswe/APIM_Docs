\newpage
\section{Introduzione}

\subsection{Scopo del documento}
Questo documento mostra come sono organizzate tutte le attività volte alla realizzazione del progetto \textit{API Market\ped{\textit{G}}}. In particolare, questo documento contiene: un'analisi dei rischi con relativo trattamento, la scelta del modello di ciclo di vita, una pianificazione del lavoro con annessa suddivisione per ruoli, un preventivo delle risorse necessarie a svolgere il progetto e un consuntivo delle attività.

\subsection{Scopo del prodotto}
Lo scopo del prodotto è la realizzazione di un API Market per l'acquisto e la vendita di \textit{microservizi\ped{G}}. Il sistema offrirà la possibilità di registrare nuove \textit{API\ped{G}} per la vendita, permetterà la consultazione e la ricerca di API ai potenziali acquirenti, gestendo i permessi di accesso ed utilizzo tramite creazione e controllo di relative \textit{API key\ped{G}}. Il sistema, oltre alla web app stessa, sarà corredato di un \textit{API Gateway\ped{G}} per la gestione delle richieste e il controllo delle chiavi, e fornirà funzionalità avanzate di statistiche per il gestore della piattaforma e per i fornitori dei microservizi.

\subsection{Riferimenti normativi}
	\begin{itemize}
		\item \textsc{NormeDiProgetto 3\_0\_0.pdf};
		\item \textbf{Capitolato d’appalto C1:} APIM: An API Market Platform\\ \url{http://www.math.unipd.it/~tullio/IS-1/2016/Progetto/C1.pdf};
		\item \textbf{Regolamento del progetto didattico:}\\
		\url{http://www.math.unipd.it/~tullio/IS-1/2016/Dispense/L09.pdf};
		\item \textbf{Organigramma e offerta tecnico-economica:}\\
		\url{http://www.math.unipd.it/~tullio/IS-1/2016/Progetto/PD01b.html}.
	\end{itemize}

\subsection{Riferimenti informativi}
\begin{itemize}
\item \textbf{Ingegneria del Software - Ian Sommerville - Ottava edizione:}
	\begin{itemize}
		\item Capitolo 4: Software management;
		\item Capitolo 5: Gestione di progetti.
	\end{itemize}
\item \textbf{Slides del corso di Ingegneria del Software:}\\
\url{http://www.math.unipd.it/~tullio/IS-1/2016/}.
\end{itemize}

\subsection{Glossario}
Per semplificare la consultazione e disambiguare alcune terminologie tecniche, le voci indicate con la lettera \textit{G} a pedice sono descritte approfonditamente nel documento \textsc{Glossario 3\_0\_0.pdf} e specificate solo alla prima occorrenza all'interno del suddetto documento.