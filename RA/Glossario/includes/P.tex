\newpage
\section{P}

\begin{itemize}
	\item \textbf{PDCA}: noto anche come ciclo di Deming, è un metodo di gestione in quattro fasi iterativo, utilizzato in attività per il controllo e il miglioramento continuo dei processi e dei prodotti.
	\item \textbf{PHP}: acronimo ricorsivo per Hypertext Preprocessor (preprocessore di ipertesti). \MakeUppercase{è} un linguaggio di scripting interpretato, concepito per la programmazione di pagine web dinamiche. L'interprete PHP è un software gratuito, distribuito sotto licenza PHP. Attualmente, è principalmente utilizzato per sviluppare applicazioni web lato server, ma può essere usato anche per scrivere script a riga di comando o applicazioni stand-alone con interfaccia grafica.
	\item \textbf{PHP7}: al momento è l'ultima versione stabile di PHP, rilasciata nel dicembre 2015.
	\item \textbf{PHP Code Checker}: servizio gratuito online, che aiuta a controllare la validità dei documenti PHP.
	\item \textbf{PostgreSQL}: completo sistema di gestione di basi di dati ad oggetti.
	\item \textbf{Protractor}: framework che consente di eseguire test end-to-end per applicazioni sviluppate in Angular e AngularJS. Protractor esegue i test sull'applicazione come se fosse un utente umano, interagendo e riportando eventuali errori.
	\item \textbf{Python3}: linguaggio di programmazione ad alto livello, orientato agli oggetti, adatto per sviluppare applicazioni distribuite, scripting, computazione numerica e system testing.
\end{itemize}


