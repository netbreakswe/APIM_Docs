\newpage
\section{Tecnologie}
\subsection{AngularJS}
Per lo sviluppo della parte \textit{Front-End\ped{G}} dell'applicazione si è scelto di usare il \textit{framework\ped{G}} \textit{JavaScript\ped{G}} \textit{AngularJS\ped{G}}, che permette lo sviluppo di applicazioni in singola pagina. Esso consente di utilizzare HTML\ped{G} come linguaggio di template e permette di estenderne la sintassi per esprimere i componenti dell'applicazione in modo chiaro e conciso.

\subsection{Node.js}
La Web app necessita di \textit{node.js\ped{G}} per essere eseguita. Esso permette di realizzare applicazioni Web, come appunto API Market, utilizzando il linguaggio JavaScript, tipicamente client-side, per la scrittura server-side. La caratteristica principale di Node.js risiede nella possibilità che offre di accedere alle risorse del sistema operativo in modalità \textit{event-driven\ped{G}} non sfruttando il classico modello basato su processi o thread concorrenti, utilizzato dai classici web server. In particolare, si utilizzano diversi pacchetti npm dall'ecosistema Node.js quali Http-server e Bower. 

\subsection{MySQL}
MySQL o \textit{Database MySQL\ped{G}} è il più diffuso Relational Database Management system (RDBMS) composto da un client a riga di comando e un server. Entrambi i software sono disponibili sia per sistemi Unix e Unix-like che per Windows; le piattaforme principali di riferimento sono Linux e Oracle Solaris. Il software supporta numerosi applicativi per la corretta gestione dei database ad esso associato, ed è rilasciato con licenza Open Source da Oracle. 


\subsection{Jolie}
\textit{Jolie\ped{G}} è il primo linguaggio esplicitamente orientato ai microservizi. E' un linguaggio open-source utilizzato per lo sviluppo back-end a microservizi di API Market. Ogni microservizio del progetto è stato sviluppato a partire da questo linguaggio e integrato con opportune librerie Java. Jolie infatti supporta nativamente un integrazione con Java, dal quale deriva direttamente. 

\subsection{Java}
 \textit{Java\ped{G}} è un linguaggio di programmazione ad alto livello, orientato agli oggetti e a tipizzazione statica, specificatamente progettato per essere il più possibile indipendente dalla piattaforma di esecuzione. Vista la diffusione di Java e l'integrazione nativa con il linguaggio Jolie, esso si presta per l'integrazione di librerie di terze parti, quali ad esempio controlli JDBC per la connessione ai database. 
 
\subsection{Javascript}
JavaScript è un linguaggio di scripting orientato agli oggetti e agli eventi, comunemente utilizzato nella programmazione Web lato client per la creazione, in siti web e applicazioni web, di effetti dinamici interattivi tramite funzioni di script invocate da eventi innescati a loro volta in vari modi dall'utente sulla pagina web in uso (mouse, tastiera, caricamento della pagina ecc...). Tali funzioni di script possono essere opportunamente inserite in file HTML, in pagine JSP o in appositi file separati con estensione .js poi richiamati nella logica di business.

\subsection{JQuery}
\textit{jQuery\ped{G}} è una libreria JavaScript per applicazioni web. Nasce con l'obiettivo di semplificare la selezione, la manipolazione, la gestione degli eventi e l'animazione di elementi DOM in pagine HTML, nonché implementare funzionalità AJAX. 

\subsection{bootstrap 3}
\textit{Bootstrap 3\ped{G}} è una raccolta di strumenti per la creazione di siti e applicazioni Web. Essa contiene modelli di progettazione basati su HTML e \textit{CSS\ped{G}}, sia per la tipografia, che per le varie componenti dell’interfaccia, come moduli, così come alcune estensioni opzionali di JavaScript.