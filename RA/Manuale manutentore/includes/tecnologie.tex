\newpage
\section{Tecnologie}
\subsection{AngularJS}
Per lo sviluppo della parte Front-End dell'applicazione si è scelto di usare il framework\ped{G} JavaScript AngularJS, che permette lo sviluppo di applicazioni in singola pagina. Esso consente di utilizzare HTML\ped{G} come linguaggio di template e permette di estenderne la sintassi per esprimere i componenti dell'applicazione in modo chiaro e conciso.

\subsection{Node.js}
La webApp necessita di node.js per essere eseguita. Esso permette di realizzare applicazioni Web, come appunto QuizziPedia, utilizzando il linguaggio JavaScript, tipicamente client-side, per la scrittura server-side. La caratteristica principale di Node.js risiede nella possibilità che ooffre di accedere alle risorse del sistema operativo in modalità event-drivenG non sfruttando il classico modello basato su processi o thread concorrenti, utilizzato dai classici web server.

\subsection{MySQL}
MySQL o Oracle MySQL  è un Relational database management system (RDBMS) composto da un client a riga di comando e un server. Entrambi i software sono disponibili sia per sistemi Unix e Unix-like che per Windows; le piattaforme principali di riferimento sono Linux e Oracle Solaris.


\subsection{Jolie}
Jolie è il primo linguaggio esplicitamente orientato ai microservizi e ogni microservizio è scritto in questo linguaggio.
da ampliare

\subsection{Java}
 Java è un linguaggio di programmazione ad alto livello, orientato agli oggetti e a tipizzazione statica, specificatamente progettato per essere il più possibile indipendente dalla piattaforma di esecuzione.
 
\subsection{Javascript}
In informatica JavaScript è un linguaggio di scripting orientato agli oggetti e agli eventi, comunemente utilizzato nella programmazione Web lato client per la creazione, in siti web e applicazioni web, di effetti dinamici interattivi tramite funzioni di script invocate da eventi innescati a loro volta in vari modi dall'utente sulla pagina web in uso (mouse, tastiera, caricamento della pagina ecc...). Tali funzioni di script possono essere opportunamente inserite in file HTML, in pagine JSP o in appositi file separati con estensione .js poi richiamati nella logica di business.

\subsection{JQuery}
jQuery è una libreria JavaScript per applicazioni web. Nasce con l'obiettivo di semplificare la selezione, la manipolazione, la gestione degli eventi e l'animazione di elementi DOM in pagine HTML, nonché implementare funzionalità AJAX.

\section{bootstrap}
Bootstrap è una raccolta di strumenti per la creazione di siti e applicazioni Web. Essa contiene modelli di progettazione basati su HTML e CSS, sia per la tipografia, che per le varie componenti dell’interfaccia, come moduli, così come alcune estensioni opzionali di
JavaScript.