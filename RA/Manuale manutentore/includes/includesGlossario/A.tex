\subsection{A}
\begin{itemize}
	\item \textbf{AJAX: }In informatica AJAX, acronimo di Asynchronous JavaScript and XML, è una tecnica di sviluppo software per la realizzazione di applicazioni web interattive. Lo sviluppo di applicazioni HTML con AJAX si basa su uno scambio di dati in background fra web browser e server, che consente l'aggiornamento dinamico di una pagina web senza esplicito ricaricamento da parte dell'utente.
	AJAX è asincrono nel senso che i dati extra sono richiesti al server e caricati in background senza interferire con il comportamento della pagina esistente. Normalmente le funzioni richiamate sono scritte con il linguaggio JavaScript. Tuttavia, e a dispetto del nome, l'uso di JavaScript e di XML non è obbligatorio, come non è detto che le richieste di caricamento debbano essere necessariamente asincrone.
	\item \textbf{AngularJS:} è un framework web open source principalmente sviluppato da Google e dalla comunità di sviluppatori individuali che ruotano intorno al framework nato per affrontare le molte difficoltà incontrate nello sviluppo di applicazioni singola pagina.
	\item \textbf{API Gateway:} strumento che filtra e reindirizza le richieste utente per le varie API, fornendo il servizio, anche se questo non è presente sul server del marketplace.
	\item \textbf{API Key:} codice che identifica univocamente una specifica API e funge da token segreto che ne regola l'accesso e l'utilizzo.
	\item \textbf{API Market:} piattaforma che permette il commercio di API.
\end{itemize}