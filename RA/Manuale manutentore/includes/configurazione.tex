\newpage
\section{Configurazione software}
\subsection{Pre-requisiti}

\subsection{Installazione di Git}
	\subsubsection{Windows 10}
	Scaricare l’eseguibile di msysGit dal sito: http://msysgit.github.io/ e procedere con
	l’installazione.
	\subsubsection{Mac OSX}
	Eseguire da terminale il seguente comando, dopo aver installato Homebrew (http://
	brew.sh/):
	
	\begin{center}
	  \verb| $ brew install git|
	\end{center}

	\subsubsection{Ubuntu 16.04 LTS}
	Eseguire da terminale il seguente comando:
	
	\begin{center}
	  \verb| $ apt install git|
\end{center}

\subsubsection{Installazione node.js}
Scaricare l'installer corretto in base al proprio sistema operativo, reperibile alla pagina ufficiale di download di Node.js (https://nodejs.org/it/download/). Procedere dunquecon l’installazione di Node.js.

\subsubsection{Download dei file da GitHub}
Eseguire da terminale il seguente comando:
\begin{center}
	\verb| $  git clone ––recursive-submodules https://github.com/netbreakswe/APIM_Source|
\end{center}


\subsubsection{Installazione Java JDK}
%Scaricare l’eseguibile compatibile con il proprio terminale di jdk8 dal sito: \hyphenation{http://www.oracle.com/technetwork/java/javase/downloads/jdk8-downloads-2133151.html} e procedere con l’installazione.

\subsubsection{Installazione Jolie}
Prima di eseguire questo passaggio è importante aver eseguito quanto descritto nella sezione precedente.

Scaricare l’eseguibile di Jolie dal sito: http://www.jolie-lang.org/downloads.html. 
\paragraph{Windows}
Aprire un terminale con privilegi di amministratore nella cartella del download ed eseguire:
	\begin{center}
	 \verb| java -jar jolie-1.6.0.jar|
	\end{center}
\paragraph{Linux/MacOS}
Aprire un terminale nella cartella del download ed eseguire:
	\begin{center}
	 \verb| sudo java -jar jolie-1.6.0.jar|
	\end{center}
\subsection{Installazione Dipendenze WebApp}
Aprire il terminale nella cartella con il seguente percorso: \verb|*Download del punto 5.2.5*/Frontend| ed eseguire il seguente comando:
	\begin{center}
	\verb| npm install|
\end{center}


\subsection{Installazione piattaforma}
 dire quali servizi avviare
 - non ancora fatto per capire se si fa bath oppure no e bisogna avviare tutti i microservizi

\subsection{Configurazione database}

dire come configurare i database su heroku!!!!

\subsection{Avvio dell'applicazione}
Aprire il terminale nella cartella con il seguente percorso: \verb|*Download del punto 5.2.5*/Frontend| ed eseguire il seguente comando:
\begin{center}
	\verb| npm start|
\end{center}

