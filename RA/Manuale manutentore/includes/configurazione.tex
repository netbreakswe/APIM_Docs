\newpage
\section{Configurazione software}
\subsection{Pre-requisiti}

\subsection{Installazione di Git}
	\subsubsection{Windows 10}
	Scaricare l’eseguibile di msysGit dal sito: http://msysgit.github.io/ e procedere con
	l’installazione.
	\subsubsection{Mac OSX}
	Eseguire da terminale il seguente comando, dopo aver installato Homebrew (http://
	brew.sh/):
	
	\begin{center}
	  \verb| $ brew install git|
	\end{center}

	\subsubsection{Ubuntu 16.04 LTS}
	Eseguire da terminale il seguente comando:
	
	\begin{center}
	  \verb| $ apt install git|
\end{center}

\subsubsection{Installazione node.js}
Scaricare l'installer corretto in base al proprio sistema operativo, reperibile alla pagina ufficiale di download di Node.js (https://nodejs.org/it/download/). Procedere dunquecon l’installazione di Node.js.

\subsubsection{Download dei file da GitHub}
Eseguire da terminale il seguente comando:
\begin{center}
	\verb| $  git clone ––recursive-submodules https://github.com/netbreakswe/APIM_Source|
\end{center}


\subsubsection{Installazione Java JDK}
Scaricare l’eseguibile compatibile con il proprio terminale di JDK8 dal sito:
	\begin{center}
		 \url{http://www.oracle.com/technetwork/java/javase/downloads/} 
	 \end{center}
e procedere con l’installazione.

\subsubsection{Installazione Jolie}
Prima di eseguire questo passaggio è importante aver eseguito quanto descritto nella sezione precedente.

Scaricare l’eseguibile di Jolie dal sito: http://www.jolie-lang.org/downloads.html. 
\paragraph{Windows}
Aprire un terminale con privilegi di amministratore nella cartella del download ed eseguire:
	\begin{center}
	 \verb| java -jar jolie-1.6.0.jar|
	\end{center}
\paragraph{Linux/MacOS}
Aprire un terminale nella cartella del download ed eseguire:
	\begin{center}
	 \verb| sudo java -jar jolie-1.6.0.jar|
	\end{center}
\subsection{Installazione Dipendenze WebApp}
Aprire il terminale nella cartella con il seguente percorso: \verb|*Download del punto 5.2.5*/Frontend| ed eseguire il seguente comando:
	\begin{center}
	\verb| npm install|
\end{center}


\subsection{Installazione piattaforma}
In questa sezione è spiegato come avviare i microservizi neccessari per una corretta fruizione della web app e del gateway. Si suppone che la cartella di partenza sia la cartella scaricata al punto 5.2.5 del manuale corrente.
\subsubsection{Microservizi relativi alla Web App}
Di seguito saranno elencati i microservizi da avviare per una corretta funzione della web app. L'ordine di avvio non è rilevante e può avvenire in ordine casuale.
	\paragraph{Microservizio per gestione utenti}
	Per avviare il microservizio relativo alla gestione utenti è necesario recarsi nella cartella:
	\begin{center}
		\verb| \Backend\services\usersDB|
	\end{center}
	
	ed eseguire da terminale il comando:
	
	\begin{center}
		\verb| jolie user_db.ol |
	\end{center}

	\paragraph{Microservizio per il Service Level Agreement}
	Per avviare il microservizio relativo al Service Level Agreement è necesario recarsi nella cartella:
	\begin{center}
		\verb| \Backend\services\slaDB|
	\end{center}
	
	ed eseguire da terminale il comando:
	
	\begin{center}
		\verb| jolie sla_db.ol |
	\end{center}

	\paragraph{Microservizio per la gestione di file}
	Per avviare il microservizio relativo alla gestione di file è necesario recarsi nella cartella:
	\begin{center}
		\verb| \Backend\services\filehandlerDB|
	\end{center}
	
	ed eseguire da terminale il comando:
	
	\begin{center}
		\verb| jolie filehandler.ol |
	\end{center}

	\paragraph{Microservizio per la gestione di microservizi}
	Per avviare il microservizio relativo alla gestione di microservizi è necesario recarsi nella cartella:
	\begin{center}
		\verb| \Backend\services\microservicesDB|
	\end{center}
	
	ed eseguire da terminale il comando:
	
	\begin{center}
		\verb| jolie microservices_db.ol |
	\end{center}

	\paragraph{Microservizio per la gestione delle transazioni}
Per avviare il microservizio relativo alla gestione delle transazioni è necesario recarsi nella cartella:
\begin{center}
	\verb| \Backend\services\transactionsDB|
\end{center}

ed eseguire da terminale il comando:

\begin{center}
	\verb| jolie transactions_db.ol |
\end{center}

\subsubsection{Microservizi relativi all'API Gateway}
Di seguito saranno elencati i microservizi da avviare per una corretta funzione dell'API Gateway. L'ordine di avvio  è rilevante e come prerequisito si richiede di aver avviato tutti i servizi elencati al punto 5.4.1.

\paragraph{Microservizio per gestione microservizi del gateway}
	Per avviare il microservizio relativo alla gestione delle transazioni è necesario recarsi nella cartella:
	\begin{center}
		\verb| \Backend\services\microservicesDB|
	\end{center}
	
	ed eseguire da terminale il comando:
	
	\begin{center}
		\verb| jolie serviceinteractionhandler.ol |
	\end{center}


\paragraph{Avvio del gateway}
	Per avviare il microservizio relativo alla gestione delle transazioni è necesario recarsi nella cartella:
	\begin{center}
		\verb| \Backend\gateway|
	\end{center}
	
	ed eseguire da terminale il comando:
	
	\begin{center}
		\verb| jolie gateway.ol |
	\end{center}


\subsection{Configurazione database}

dire come configurare i database su heroku!!!!

\subsection{Avvio dell'applicazione}
Aprire il terminale nella cartella con il seguente percorso:
\begin{center}
	\verb|\Frontend| 
\end{center} 
ed eseguire il seguente comando:
\begin{center}
	\verb| npm start|
\end{center}

