\newpage

\section{Processi di supporto}

	\subsection{Documentazione}
	In questa sezione del documento verrà spiegata la struttura, la classificazione e il metodo di revisione dei documenti del gruppo \textit{\gruppo}.
	L'attività di documentazione è di supporto a tutte le altre attività, difatti ogni cosa deve essere tracciata e documentata.
	L'obiettivo prefissato è:
	\begin{itemize}
		\item Produrre documenti corretti grammaticalmente e sintatticamente;
		\item Leggibili dal più largo numero di persone e nel rispetto dell'indice di Gulpease;
		\item Avere un glossario comune per evitare ambiguità di nomenclatura tecnica;
		\item Produrre documenti di struttura uniforme.
	\end{itemize}
	
		\subsubsection{Nomenclatura e versione del documento}
		Ogni documento deve rispettare una denominazione comune, il più possibile chiara, affinché venga individuato facilmente il nome e la versione del documento. Il team ha scelto di utilizzare la seguente nomenclatura:
		\begin{center}
			\textit{NomeDelDocumento X\_Y\_Z.pdf}
		\end{center}
		All'interno del nome, \textit{X\_Y\_Z} indica la versione del documento nel seguente modo:
		\begin{itemize}
			\item \textit{\textbf{X}}: indica il numero di pubblicazioni del documento. Questo indice è incrementato esclusivamente dal \textit{\RdP} in seguito alla sua approvazione finale. L’incremento di tale indice, azzera automaticamente gli indici Y e Z;
			\item \textit{\textbf{Y}}: indica il numero di verifiche e viene incrementato esclusivamente dai \textit{\Vers}. L'incremento di tale indice, azzera automaticamente l'indice Z;
			\item \textit{\textbf{Z}}: indica il numero di aggiornamenti minori effettuati prima o in seguito a una verifica o approvazione. Viene incrementato progressivamente e viene azzerato solo in seguito a una modifica degli indici X e Y.
		\end{itemize}
		Ogni modifica della versione del documento deve riflettersi nel changelog, nel nome del documento e nel frontespizio in corrispondenza della voce \textit{Versione}.
	
		\subsubsection{Classificazione del documento}
		Per tutti i documenti redatti dal gruppo occorre sempre specificare una classe, che può essere:
		\begin{itemize}
		\item \textbf{Interna}: la consultazione è limitata al solo gruppo e comprende tutti i documenti riguardandi l'organizzazione del gruppo e le regole imposte ai membri e al team;
		\item \textbf{Esterna}: la consultazione è destinata ad enti esterni al gruppo, quindi a proponente e committente.
		\end{itemize} 
	
		\subsubsection{Ciclo di vita del documento}
		Un documento può trovarsi in uno dei seguenti stati:
		\begin{itemize}
			\item \textbf{Bozza}:documento in fase di stesura dal relativo redattore;
			\item \textbf{In attesa di verifica}: la stesura del documento è terminata e in attesa di verifica da parte del verificatore;
			\item \textbf{Approvato}: il documento è nella sua versione finale.
		\end{itemize}
	
		\subsubsection{Struttura del documento}
		Ogni documento redatto dal gruppo avrà una struttura chiara e di facile comprensione per chiunque sia il lettore.\\
		Una struttura semplice aiuta nella lettura e nell'individuare immediatamente l'argomento trattato nello specifico documento e l'identità di chi ha redatto, verificato e approvato il documento.
		Di seguito, viene riportata la struttura utilizzata nei documenti:

			\paragraph{Frontespizio}
			\begin{itemize}
				\item{Nome del gruppo;}
				\item{Nome del progetto;}
				\item{Logo del gruppo;}
				\item{Nome del documento;}
				\item{Data di creazione;}
				\item{Ultima modifica;}
				\item{Versione;}
				\item{Stato del documento;}
				\item{Nome e cognome del/i redattore/i;}
				\item{Nome e cognome del/dei \textit{\Ver/i};}
				\item{Nome e cognome del \textit{\RdP} che ha approvato il documento;}
				\item{Classificazione del documento;}
				\item{Distribuzione;}
				\item{Destinatari del documento;}
				\item{Email di riferimento;}
				\item{Abstract del documento.}
			\end{itemize}
	
			\paragraph{Changelog}
			La seconda sezione del documento è una lista che raccoglie e tiene traccia di tutte le modifiche effettuate al documento, dalla sua creazione alla sua approvazione finale. Le modifiche sono indicate secondo questo schema:
			\begin{itemize}
				\item \textbf{Versione}: indica la versione del documento;
				\item \textbf{Data}: indica la data di modifica del documento;
				\item\textbf{Autore}: indica l'autore della modifica effettuata;
				\item \textbf{Ruolo}: indica il ruolo dell'autore che ha effettuato la modifica;
				\item\textbf{Descrizione}: indica le aggiunte e/o modifiche effettuate al documento.
			\end{itemize}	

			\paragraph{Indice}
			Ogni documento, ad eccezione dei verbali, deve possedere un indice contenente tutte le sezioni presenti nel documento.\\
			L'indice è ordinato nella sequenza in cui appaiono i capitoli, le sezioni e sottosezioni, e contiene il titolo di ognuna.\\
			Se nel documento sono presenti immagini o tabelle, occorre indicarle con il relativo indice.
			
			\paragraph{Intestazione e piè di pagina}
			Tutte le pagine del documento, ad eccezione della prima, hanno un'intestazione e un piè di pagina. L'intestazione è suddivisa in due parti:
			\begin{itemize}
				\item Nell'angolo sinistro, sono indicati il nome del progetto e del documento in questione;
				\item Nell'angolo destro, è presente il nome del capitolo contente la sezione che si sta consultando, ad esclusione del changelog.
			\end{itemize}
			Il piè di pagina, invece, è strutturato nel seguente modo:
			\begin{itemize}
				\item Nell'angolo sinistro, vengono indicati nome e indirizzo email ufficiale del gruppo;
				\item Nell'angolo destro, è presente una numerazione in numeri romani per le pagine relative a changelog e indice. Per tutte le altre pagine, invece, viene utilizzata una numerazione in numeri arabi, che indica il numero di pagina attuale e il numero totale di pagine del documento.
			\end{itemize} 

			\paragraph{Norme tipografiche}
			Questa sezione del documento contiene i criteri riguardanti l'ortografia e la tipografia, utilizzate nel corso dello sviluppo della documentazione del progetto.

				\subparagraph{Stili di testo e punteggiatura}
				Stili:
				\begin{itemize}
					\item \textbf{Grassetto}: deve essere utilizzato per parole importanti, all’interno di frasi o elenchi puntati;
					\item \textit{Corsivo}: deve essere utilizzato per:
					\begin{itemize}
						\item Citazioni;
						\item Parole del glossario (unitamente ad una G maiuscola in pedice) alla loro prima occorrenza nel documento di riferimento;
						\item Riferimenti ai ruoli;
						\item Riferimento all'intero gruppo.
					\end{itemize}
					\item MAIUSCOLO: viene utilizzato unicamente per scrivere acronimi e macro \LaTeX\ presenti nei documenti;
					\item \textsc{maiuscoletto}: viene utilizzato per i riferimenti ad altri documenti;
					\item \LaTeX: viene usato il comando \textbackslash{LaTeX} per ogni occorrenza del termine \LaTeX.
					\end{itemize}
				Punteggiatura:
				\begin{itemize}
					\item Punteggiatura: non deve seguire spazi, ad eccezione del punto, del punto interrogativo e del punto esclamativo, che sono seguiti da uno spazio e lettera maiuscola.
					\item elenchi puntati: Ogni voce di un elenco puntato deve terminare con punto e virgola, ad eccezione dell'ultima che termina con un punto. 
					
				\end{itemize}
	
				\subparagraph{Formato data}
				All'interno di ogni documento, tutte le date seguiranno lo standard \textit{\textbf{ISO 8601:2004\ped{G}}}:
				\begin{center}
					\textbf{YYYY-MM-DD}
				\end{center}
				Dove:
				\begin{itemize}
					\item \textbf{YYYY}: indica l'anno;
					\item \textbf{MM}: indica il mese;
					\item \textbf{DD}: indica il giorno.
				\end{itemize}
		
			\subparagraph{Formato ora}
			All'interno di ogni documento, tutti gli orari seguiranno lo standard \textit{\textbf{ISO 8601:2004\ped{G}}}:
			\begin{center}
				\textbf{hh-mm}
			\end{center}
			Dove:
			\begin{itemize}
				\item \textbf{hh}: l'ora;
				\item \textbf{MM}: indica il minuto.
			\end{itemize}
		
		\subsubsection{Documenti da consegnare}
			\paragraph{\SdF}
			\begin{itemize}
				\item \textbf{Classificazione}: interno;
				\item \textbf{Destinazione}: gruppo e committente;
				\item \textbf{Contenuto}: questo documento contiene lo studio effettuato dal gruppo \textit{\gruppo} su tutti i capitolati e, per ognuno di essi, le motivazioni che hanno portato alla relativa scelta o rifiuto.
			\end{itemize}

			\paragraph{\NdP}
			\begin{itemize}
				\item \textbf{Classificazione}: interno;
				\item \textbf{Destinazione}: gruppo e committente;
				\item \textbf{Contenuto}: lo scopo del documento è raccogliere tutte le convenzioni, gli strumenti e le regole che il gruppo \textit{\gruppo} adotterà durante l'intera realizzazione del progetto. 
			\end{itemize}	

			\paragraph{\AdR}
			\begin{itemize}
				\item \textbf{Classificazione}: esterno;
				\item \textbf{Destinazione}: gruppo, committente e proponente;
				\item \textbf{Contenuto}: questo documento si prefigge lo scopo di dare una visione generale dei requisiti essenziali del progetto e dei relativi casi d'uso.
			\end{itemize}

			\paragraph{\PdP}
			\begin{itemize}
				\item \textbf{Classificazione}: esterno;
				\item \textbf{Destinazione}: gruppo, committente e proponente;
				\item \textbf{Contenuto}: questo documento descrive come il gruppo \textit{\gruppo} ha impiegato tempo e risorse umane, ma anche la pianificazione delle stesse, per le attività future previste per la realizzazione del prodotto richiesto dal progetto.
			\end{itemize}

			\paragraph{\PdQ}
			\begin{itemize}
				\item \textbf{Classificazione}: esterno;
				\item \textbf{Destinazione}: gruppo, committente e proponente;
				\item \textbf{Contenuto}: questo documento descrive come il gruppo \textit{\gruppo} intende raggiungere gli obiettivi di qualità prefissati.
			\end{itemize}

			\paragraph{\G}
			\begin{itemize}
				\item \textbf{Classificazione}: esterno;
				\item \textbf{Destinazione}: gruppo, committente e proponente;
				\item \textbf{Contenuto}: questo documento ha lo scopo di fornire una definizione di tutti i termini tecnici e acronimi, al fine di rendere la lettura comprensibile a tutti i destinatari della documentazione fornita.
			\end{itemize}
		\paragraph{\ST}
		\begin{itemize}
			\item \textbf{Classificazione}: esterno;
			\item \textbf{Destinazione}: gruppo, committente e proponente;
			\item \textbf{Contenuto}: lo scopo di questo documento è fornire una visione generale del prodotto e dei suoi requisiti. In particolare, viene mostrata una progettazione ad
			alto livello, basata su diagrammi dei package, per la descrizione delle componenti, diagrammi di sequenza e di attività, per la descrizione degli scenari d'uso.\\
			Inoltre, vengono elencate le tecnologie che si intendono utilizzare per la realizzazione del progetto.
		\end{itemize}
		
		\paragraph{\DDP}
		\begin{itemize}
			\item \textbf{Classificazione}: esterno;
			\item \textbf{Destinazione}: gruppo, committente e proponente;
			\item \textbf{Contenuto}: questo documento ha lo scopo di fornire una progettazione dettagliata del
			prodotto. In esso vengono forniti tutti i dettagli implementativi del prodotto, fondamentali in fase di codifica, che comprendono i diagrammi delle classi e i relativi metodi.
		\end{itemize}
	
		\paragraph{\MU}
		\begin{itemize}
			\item \textbf{Classificazione}: esterno;
			\item \textbf{Destinazione}: gruppo, committente e proponente;
			\item \textbf{Contenuto}: questo documento ha lo scopo di fornire all'utente una guida dettagliata riguardo tutte le funzionalità offerte dal prodotto realizzato.
		\end{itemize}
	
		\paragraph{Manuale Manutentore}
		\begin{itemize}
			\item \textbf{Classificazione}: esterno;
			\item \textbf{Destinazione}: gruppo, committente e proponente;
			\item \textbf{Contenuto}: questo documento ha lo scopo di fornire agli sviluppatori e manutentori dell'applicazione, una guida dettagliata riguardo tutte le specifiche tecniche e tecnologiche per il mantenimento del prodotto.
		\end{itemize}

		\subsubsection{Strumenti utilizzati}
		Per la stesura dei documenti è stato scelto di utilizzare \LaTeX, un linguaggio di markup utilizzato per la stesura di documenti scientifici.\\
		\LaTeX è un linguaggio semplice e modulare, in grado di evitare possibili conflitti provenienti dall'utilizzo di software e piattaforme differenti.
		\paragraph{TeXstudio}
		In seguito a un breve periodo di test, la scelta del team per l'editor di documenti \textit{\LaTeX\ped{G}} è ricaduta su \textit{\textbf{TeXstudio\ped{G}}}. Esso è un noto fork di un altrettanto famoso editor, \textit{TexMaker\ped{G}}, ma con la peculiarità di essere completamente gratuito, open source e disponibile per i più diffusi sistemi operativi.\\ L'editor consente un semplice utilizzo dei file \textit{.tex} e una comoda gestione di inclusione e accorpamento di documenti. Tale caratteristica viene incontro alle esigenze del gruppo per via della predisposizione alla modularità dei documenti richiesti dal progetto.\\
		L'editor scelto, inoltre, possiede numerose features rilevanti, quali l'autocompletamento e il controllo della sintassi e dell'ortografia, e l'anteprima in PDF incorporata. Per questi motivi, TeXstudio sarà il principale strumento tramite cui verranno redatti tutti i documenti.
	
	\newpage
	\subsection{Verifica}
	Il processo di verifica ha lo scopo di controllare ed assicurare che documentazione e software prodotti rispecchino quanto previsto dai requisiti.\\
	In particolare, la verifica serve a stabilire che il prodotto rispetti le specifiche, mentre la validazione accerta che le specifiche siano rispettate nel modo più consono.

		\subsubsection{Verifica documenti}
		
		\paragraph{Descrizione}
		L'attività di verifica dei documenti è svolta dai \textit{Verificatori}. Essi hanno il compito di segnalare ogni incongruenza o errore, facendo presente eventuali punti non chiari, in modo tale da permetterne la correzione.\\
		Ogni problema deve essere segnalato tramite gli strumenti di comunicazione e issue tracking internamente utilizzati, con una breve descrizione del problema e un riferimento alla posizione del documento da controllare.\\
		In alternativa, il \textit{Verificatore} può effettuare precise annotazioni nei documenti PDF resi disponibili per il controllo, da consegnare in seguito alla persona responsabile di effettuare le modifiche nel documento.\\
		Qualsiasi dubbio aggiuntivo può essere chiarito tramite gli strumenti di comunicazione abitualmente utilizzati dal team.
		
		\paragraph{Strategia generale}
		Si considerano, in aggiunta alle descrizioni e alle strategie descritte nella sottosezione corrente, le metodologie sottoriportate:
		
		\begin{itemize}
			\item Si effettua un controllo preliminare della documentazione correlata ed eventuali aggiornamenti che possano essere rilevanti in attività di verifica;
			\item Si valutano le \textit{Norme di Progetto} in vigore;
			\item Si procede ad una correzione preliminare tramite controllo ortografico standard.
		\end{itemize}
	
		\paragraph{Obiettivi attesi}
		Gli obiettivi di qualità attesi dall'attività di verifica dei documenti sono:
		
		\begin{itemize}
			\item L'ottemperanza delle \textit{Norme di Progetto} per l'ultima versione del documento durante il processo di verifica;
			\item L'eliminazione degli errori sintattici e ortografici;
			\item L'eliminazione di porzioni di difficile leggibilità e comprensione;
			\item La corrispondenza con la restante documentazione.
		\end{itemize}
				
		\paragraph{Indice Gulpease}
		Per verificare l'indice di leggibilità dei documenti, così come definito nel documento \textsc{PianoDiQualifica 4\_0\_0.pdf}, si è deciso di utilizzare uno strumento realizzato da un gruppo degli anni precedenti.\\
		Il calcolo dell'indice Gulpease è effettuato direttamente attraverso la piattaforma \textit{NetBreakDB}, la quale è in grado di analizzare i vari file \textit{.tex} all'interno di una cartella e restituire il nome del documento, il valore dell'indice e un'indicazione se il valore è superiore al minimo prefissato.\\
		Questo script è un ottimo strumento per automatizzare il lavoro, in quanto, dopo alcune modifiche, è in grado di esportare i risultati in codice \LaTeX\ e semplificarne l'integrazione, indicando in quali documenti è necessario intervenire per migliorarne la leggibilità del testo.
		
		\paragraph{Modalità di analisi}
			\subparagraph{Analisi statica}
			L'analisi statica è applicata ai documenti di testo, e consiste nel trovare errori sintattici ed ortografici. Una prima verifica viene effettuata dai \textit{\Vers} e, successivamente, dal \textit{\RdP}. Le due tecniche scelte sono la \textbf{\textit{Formal Walkthrough\ped{G}}} e la \textbf{\textit{Fagan Inspection\ped{G}}}.
			\begin{itemize}
				\item \textbf{Formal Walkthrough}: consiste nell'individuare quanti più errori di sintassi e di ortografia possibili, senza concentrarsi su particolari e specifici errori. Questa tecnica sarà utilizzata dai \textit{\Vers}, i quali annoteranno gli errori più frequenti in un'apposita lista, necessaria per la successiva tecnica di Fagan Inspection;
				\item \textbf{Fagan Inspection}: si basa su una lettura attenta dei documenti, basandosi sulla lista degli errori stilata durante la Formal Walkthrough. Questo processo acquisirà rilevanza in concomitanza con l'incremento della lista di possibili errori, stilata dai \textit{\Vers}.
			\end{itemize}

			\paragraph{Analisi dinamica}
			L'analisi dinamica è applicata esclusivamente sul software prodotto, in quanto consiste nell'effettuare dei test per verificare il corretto funzionamento dell'applicativo prodotto.
	
		\subsubsection{Verifica del codice}
		
		\paragraph{Descrizione}
		L’attività di verifica del codice è effettuata tramite analisi statica e dinamica. In particolare, l'analisi dinamica viene applicata al codice solamente, poichè richiede la compilazione dello stesso e la sua esecuzione.\\
		Lo scopo di questa attività è che il software prodotto fornisca i risultati attesi.
		
		\paragraph{Strategia generale}
		Si considerano, in aggiunta alle descrizioni e alle strategie descritte nella sottosezione corrente, le metodologie sottoriportate:
		
		\begin{itemize}
			\item Il software viene analizzato secondo le metriche stabilite e ogni discrepanza dai risultati attesi viene segnalata tempestivamente ai responsabili della porzione di codifica considerata.
		\end{itemize}
		
		\paragraph{Obiettivi attesi}
		Gli obiettivi di qualità attesi dall'attività di verifica del codice sono:
		
		\begin{itemize}
			\item I test effettuati devono essere quanto più estesi possibili, per coprire ogni caso d'uso del prodotto finale;
			\item I test devono avere l'apporto minimo da parte del \textit{Verificatore}, ed essere quindi, quanto più automatizzati possibile.
		\end{itemize}
		
		\paragraph{Test}
		Il sistema adottato per testare l'operato prodotto, prende atto della struttura comune a tutto il software, decomponendo l'attività, a partire da elementi semplici fino a giungere ai test dell'intero sistema, seguendo opportune strategie di integrazione.
		
		\subparagraph{Test di unità}
		Questa tipologia di test è atta ad analizzare ogni piccola unità software singolarmente. Così facendo, si tende a minimizzare il numero di errori, a partire da ogni singola componente. Testando i singoli moduli, si rende necessaria la presenza di componenti fittizie per simulare parti di sistema incomplete e altrimenti non testabili singolarmente. Il codice utilizzato per denotare questa fase di test sarà:
		
		\begin{center}
			TU[Identificativo test]
		\end{center}
	
		\subparagraph{Test di integrazione}
		I test denominati di integrazione riguardano insiemi di moduli, che vengono testati una volta assemblati. Alternando passi di integrazione e di controllo, come in questo caso, significa necessariamente predisporre test su componenti ancora incomplete. Il codice utilizzato per denotare questa fase di test sarà:
		
		\begin{center}
			TI[Identificativo test]
		\end{center}
		
		\subparagraph{Test di sistema}
		I test di sistema valutano ogni caratteristica di qualità di prodotto nella sua completezza. Il prodotto, sottoposto a questa tipologia di test, si ritiene giunto ad una realizzazione definitiva. I test riguardano numerosi aspetti cruciali, e si possono riassumere nelle seguenti categorie:
		
		\begin{itemize}
			\item \textbf{Facility test}: vengono testate le funzionalità del prodotto, affinché svolgano correttamente le attività preposte;
			\item \textbf{Compatibility test}: si testa la compatibilità del prodotto con software di terze parti che interagirà con il sistema. Nel nostro caso specifico, trattasi, ad esempio, dei web browser che verranno utilizzati dagli utilizzatori finali;
			\item \textbf{Security test}: viene valutata la robustezza del sistema in materia di sicurezza;
			\item \textbf{Performance test}: si analizzano la reattività e le prestazioni del sistema, anche in situazioni di particolare carico.
		\end{itemize}
	
		Il codice utilizzato per denotare questa fase di test sarà:
		
		\begin{center}
			TS[Identificativo test]
		\end{center}
	
		\subparagraph{Test di non regressione}
		Durante lo sviluppo, qualora si renda necessaria la modifica di alcune componenti, i test di non regressione consistono nel rieseguire tutti i test riguardanti le componenti aggiornate, in modo tale da verificare che il funzionamento non sia stato compromesso. Il codice utilizzato per denotare questa fase di test sarà:
		
		\begin{center}
			TNR[Identificativo test]
		\end{center}
	
		\subparagraph{Test di validazione}
		Esso è il controllo ultimo e definitivo che il prodotto realizzato dal fornitore sia conforme alla richiesta del proponente. Il test è effettuato in presenza del proponente, ed in caso di esito positivo, il software può essere rilasciato.
		Il codice utilizzato per denotare questa fase di test sarà:
		
		\begin{center}
			TV[Identificativo test]
		\end{center}
			
		\paragraph{Strumenti}
		
		\subparagraph{Strumenti per l'analisi statica}
		
		\begin{itemize}
			\item \textbf{\textit{W3C Markup Validator Service\ped{G}}}: validatore online di codice \textit{HTML\ped{G}}, utile per trovare eventuali errori nel codice. L'indirizzo web di riferimento è: \url{https://validator.w3.org/};
			\item \textbf{\textit{CSSLint\ped{G}}}: validatore online di codice \textit{CSS\ped{G}}, utile per trovare eventuali errori nel codice. L'indirizzo web di riferimento è: \url{http://csslint.net/};
			\item \textit{\textbf{JSHint\ped{G}}}: è uno strumento che aiuta a rilevare possibili
			errori nel codice \textit{JavaScript\ped{G}}. L'indirizzo web di riferimento è: \url{http://jshint.com/};
			\item \textbf{\textit{SQLFiddle\ped{G}}}: validatore online di codice \textit{SQL\ped{G}}, utile per verificare la consistenza del codice del database, e compatibile con \textit{MySQL\ped{G}} e \textit{PostgreSQL\ped{G}}. L'indirizzo web di riferimento è: \url{http://sqlfiddle.com/}.
		\end{itemize}
	
	\subparagraph{Strumenti per l'analisi dinamica}
		\begin{itemize}
			\item \textbf{\textit{Google Chrome DevTools\ped{G}}}: è uno strumento atto a controllare l'utilizzo di risorse del prodotto, l'usabilità su differenti dispositivi, nonchè simulare situazioni di carico della rete. L'indirizzo web di riferimento è: \url{https://www.google.it/chrome/};
			\item \textbf{\textit{Jasmine\ped{G}}}: framework utile ad eseguire test di unità su codice JavaScript. L'indirizzo di riferimento è: \url{https://jasmine.github.io/};
			\item \textbf{\textit{Protractor\ped{G}}}: framework per effettuare test su applicazioni sviluppate in AngularJS. L'indirizzo di riferimento è: \url{http://www.protractortest.org/#/}.
		\end{itemize}
	\newpage
	\subsection{Validazione}
	
	\subsubsection{Validazione dei documenti}
	
	\paragraph{Descrizione}
	L’attività di validazione dei documenti riguarda la fase preliminare al rilascio di una nuova versione di documento. Il responsabile di tale attività è il \textit{Responsabile di Progetto}, e prevede una lettura e veloce analisi del documento sottoposto ad approvazione. 
	
	\paragraph{Strategia generale}
	Durante l'attività di approvazione, il \textit{Responsabile} deve procedere ad una lettura complessiva del documento, per approvare le modifiche e correzioni stabilite.\\
	Viene controllato, prima dell'approvazione e distribuzione, l'indice Gulpease affinchè rispetti gli standard prefissati.

	\paragraph{Obiettivi attesi}
	I documenti devono essere coerenti con la strategia del team e la restante documentazione.	
	
	\subsubsection{Validazione del codice}
	
	\paragraph{Descrizione}
	L’attività di validazione del codice riguarda un controllo manuale per quanto riguarda i requisiti del software, in modo che le funzioni siano effettivamente coerenti con i casi d'uso stabiliti.
	
	\paragraph{Strategia generale}
	Un test generale del prodotto verrà svolto seguendo le procedure come indicate nel \textit{Manuale Manutentore}, per verificare che il risultato atteso sia fruibile completamente per l'utilizzatore finale.\\
	Inoltre, vengono verificati tutti i casi d'uso, in modo che l'applicazione soddisfi in pieno i requisiti obbligatori pattuiti con il proponente.\\
	Qualora alcuni requisiti non vengano soddisfatti, il software viene sottoposto ad una nuova fase di incremento per integrare le modifiche necessarie alla sua approvazione e accettazione.
	
	\paragraph{Obiettivi attesi}
	L'obiettivo principale è che il software rilasciato rispetti tutti i requisiti attesi e pattuiti in fase di definizione contrattuale.