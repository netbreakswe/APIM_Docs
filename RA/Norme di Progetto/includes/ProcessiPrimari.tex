\newpage
\section{Processi primari}

	\subsection{Fornitura}
		\subsubsection{Studio di Fattibilità}
		Questa fase vedrà la realizzazione di un documento redatto a partire da ciò che emergerà dalle riunioni, alle quali è richiesta la partecipazione dell'intero team. Ciò consentirà di scegliere di comune accordo il capitolato più adatto. Durante tali riunioni, verrà stilata una lista di pro e contro per ogni capitolato d'appalto proposto. Successivamente, gli \textit{\Anas} avranno l'incarico di effettuare la stesura degli studi per ogni singolo capitolato. Il documento prodotto sarà strutturato nel seguente modo:
		\begin{itemize}
		\item \textbf{Descrizione:} introduzione generale del capitolato, nel quale si evidenziano le funzionalità minime richieste per il prodotto finale;
		\item \textbf{Dominio applicativo:} descrizione del bacino di utenza e dell'ambito in cui il prodotto sarà utilizzato;
		\item \textbf{Tecnologie:} breve elenco delle tecnologie di interesse, fondamentali nella realizzazione del prodotto richiesto;
		\item \textbf{Aspetti critici:} elenco delle problematiche, dei rischi e delle difficoltà che potrebbero nascere in fase di sviluppo del prodotto;
		\item \textbf{Considerazioni conclusive:} valutazione finale e soggettiva del gruppo, che fornisce le motivazioni per cui è stato scelto oppure scartato il capitolato in analisi.
		\end{itemize}
		
		\subsubsection{Pianificazione}
		la pianificazione è l'attività che gestisce le scadenze temporali del progetto e il documento dedicato a questa attività è il \textit{Piano di Progetto}. Una corretta pianificazione rispetta i seguenti vincoli:
		\begin{itemize}
			\item La pianificazione non deve superare il budget o il numero di ore previste;
			\item La pianificazione deve tener conto di aventuali rischi associati allo sviluppo del progetto;
			\item La pianificazione non deve superare la data di scadenza concordata con il cliente.
		\end{itemize}
	
		\subsubsection{Preventivazione}
		La preventivazione è l'attività che stima i costi e si basa sull'\textit{Analisi dei Requisiti}. L'attività di analisi viene eseguita prima di partecipare alla gara d'appalto e i risultati sono riportati nel \textit{Piano di Progetto}. Una corretta preventivazione rispetta i seguenti vincoli:
		\begin{itemize}
			\item Il preventivo deve basarsi sui requisiti obbligatori e opzionali riportati nell'\textit{Analisi dei Requisiti};
			\item Il preventivo deve indicare le ore esatte di ogni ruolo;
			\item  Il preventivo deve seguire il costo imposto dal proponente.
		\end{itemize}
	
		\subsubsection{Preparazione al collaudo}
		La preparazione al collaudo è la fase precedente alla consegna del prodotto. Al collaudo sono presenti tutti stakeholders, in particolare il proponente e il committente. Affinchè il collaudo avvenga con successo, bisogna rispettare i seguenti vincoli:
		\begin{itemize}
			\item Tutti i requisiti indicati come obbligatori nel documento  \textit{Analisi dei Requisiti 3.0.0.pdf} devono essere soddisfatti;
			\item Tutti i test di validazione previsti devono essere soddisfatti;
			\item I requisiti indicati come desiderabili o opzionali ma richiesti espressamente  dal proponente, siano soddisfatti.
			\item Devono essere predisposti dei test che dimostrino il corretto funzionamento del prodotto finale.
		\end{itemize}
		
		
	\newpage
	\subsection{Sviluppo}
		 \subsubsection{Analisi dei Requisiti}
		L'attività di \AdR\ dovrà essere successiva allo \SdF. Essa, infatti, analizzerà, in modo quanto più accurato possibile, i requisiti necessari, per ciascun ambito del progetto scelto. Inoltre, verranno stilati i casi d'uso del prodotto, corredati da un'opportuna descrizione ed analisi. La struttura del documento prodotto sarà organizzata nei seguenti punti:
		\begin{itemize}
			\item \textbf{Casi d'uso};
			\item \textbf{Requisiti progettuali}.
		\end{itemize}
		
		\paragraph{Casi d'uso}
		
			\subparagraph{Nomenclatura}
			La scelta del nome per i casi d'uso avverrà secondo la seguente codifica:
			\begin{center}
				UC[Codice categoria].[Codice progressivo]
			\end{center}
			dove:
			\begin{itemize}
				\item\textbf{Codice categoria}: identifica il codice entro cui lo specifico caso d'uso viene raggruppato. Può essere organizzato in ulteriori sottocategorie oppure omesso;
				\item\textbf{Codice progressivo}: identificativo dello specifico caso d'uso.
			\end{itemize}
			
			\subparagraph{Struttura}
			L'analisi di ciascun caso d'uso dovrà essere strutturata come segue, avendo cura di mantenere l'ordine indicato:
			\begin{itemize}
				\item\textbf{Sigla e nome}: identifica il caso d'uso. Va indicato nel titolo del paragrafo corrispondente;
				\item\textbf{Diagramma}: immagine del diagramma \textit{UML\ped{G}} per il caso d'uso;
				\item\textbf{Attori}: descrizione degli attori coinvolti;
				\item\textbf{Descrizione}: breve descrizione del caso d'uso;
				\item\textbf{Pre-Condizioni}: descrive lo state iniziale per il caso d'uso;
				\item\textbf{Post-Condizioni}: descrive lo stato finale che deve valere al termine dello scenario descritto;
				\item\textbf{Scenario Principale}: analisi completa del flusso di esecuzione principale del caso d'uso;
				\item\textbf{Scenari Alternativi}: se presenti, elenca e descrive gli eventuali scenari alternativi per il caso d'uso.
			\end{itemize}
		
		\paragraph{Requisiti progettuali}
		Gli \textit{\Anas} hanno il compito di produrre il documento \textsc{AnalisiDeiRequisiti 3\_0\_0.pdf} contenente un elenco di requisiti, con annesse peculiarità richieste ed informazioni sulla loro tipologia e rilevanza.
		
			\subparagraph{Nomenclatura}
			La scelta del nome per i requisiti progettuali avverrà secondo la seguente procedura:
			\begin{center}
			R[Tipologia][Rilevanza][Codice]
			\end{center}
			\textbf{Tipologia:} può assumere uno dei seguenti valori:
			\begin{itemize}
				\item \textbf{V:} requisito di vincolo;
				\item \textbf{F:} requisito di funzionalità;
				\item \textbf{Q:} requisito di qualità;
				\item \textbf{P:} requisito prestazionale.
			\end{itemize}
			\textbf{Rilevanza:} può assumere uno dei seguenti valori, elencati in ordine di importanza:
			\begin{itemize}
				\item \textbf{O:} requisito obbligatorio;
				\item \textbf{D:} requisito desiderabile;
				\item \textbf{F:} requisito facoltativo.
			\end{itemize}
			\textbf{Codice:} assume un numero sequenziale e univoco, necessario a catalogare e riconoscere ogni requisito progettuale.
		
			\subparagraph{Struttura}
			L'analisi di ciascun requisito dovrà essere strutturata come segue, avendo cura di mantenere l'ordine indicato:
			\begin{itemize}
			  \item \textbf{Descrizione:} descrizione sintetica e concisa del requisito;
			  \item \textbf{Fonte:} descrive la fonte del requisito, ovvero da chi è stato sollevato e in che ambito. Può assumere i seguenti valori: capitolato, caso d'uso, interno.
			\end{itemize}
	
	\subsubsection{Progettazione}
	La Progettazione è l'attività fondamentale in cui viene realizzata un'astrazione di quella che diverrà la struttura software del prodotto. \MakeUppercase{è} successiva alla produzione di una completa \AdR, in quanto da essa vengono tracciate le linee guida per la progettazione.\\
	I documenti risultanti da questa attività fungeranno, poi, da percorso per la produzione del software vero e proprio. Una progettazione svolta al meglio è utile per un'attività di codifica ottimale.\\
	I \textit{\Progs} sono responsabili delle attività di progettazione, e sono tenuti ad avere:
	\begin{itemize}
		\item una profonda conoscenza di tutto ciò che riguarda il processo di sviluppo del software;
		\item la capacità di saper anticipare i cambiamenti;
		\item una forte inventiva per riuscire a trovare una soluzione progettuale accettabile anche in mancanza di una metodologia che sia sufficientemente espressiva;
		\item la capacità di individuare con rapidità e sicurezza le soluzioni più opportune.
	\end{itemize}
	
		\paragraph{Obiettivi}
		Durante questa attività, il team si prefigge i seguenti obiettivi:
		\begin{itemize}
			\item Fornire una visione macroscopica del percorso da seguire nella codifica software;
			\item Acquisire una profonda conoscenza di tutto ciò che riguarda lo sviluppo del software;
			\item Realizzare un prodotto rispettando gli standard prefissati nelle attività di \SdF\ e \AdR;
			\item Flessibilità del prodotto, per poter far fronte, in modo rapido, a repentine ed eventuali modifiche rilevanti, pur non condizionando il lavoro pregresso;
			\item Soddisfare le richieste del committente.
		\end{itemize}
		
		\paragraph{Diagrammi}
		Per la progettazione, il gruppo ha deciso di utilizzare quattro tipologie di diagrammi UML:
		\begin{itemize}
			\item \textbf{\textit{Diagrammi dei package\ped{G}:}} raggruppano le classi in un'unità di alto livello;
			\item \textbf{\textit{Diagrammi delle classi\ped{G}:}} descrivono le caratteristiche delle singole unità;
			\item \textbf{\textit{Diagrammi di attività\ped{G}:}} illustrano il flusso di operazioni relativo ad un'attività che è possibile svolgere sul prodotto. Possono essere utilizzati per descrivere la logica di un algoritmo specifico;
			\item \textbf{\textit{Diagrammi di sequenza\ped{G}:}} descrivono le sequenze di azioni dove tutte le decisioni sono già state effettuate, quindi sono assenti scelte dell'utente e flussi alternativi.
		\end{itemize}
	
	\paragraph{Specifica tecnica}
	I \textit{Progettisti} devono definire la struttura ad alto livello dell'architettura del sistema e delle singole componenti, raccogliendo il tutto nel documento \textsc{Specifica Tecnica 2\_0\_0.pdf}. Inoltre, hanno il compito di definire i test di integrazione tra le varie componenti, che verranno inseriti in appendice al documento \textsc{Piano di Qualifica 3\_0\_0.pdf}. \\
	Il documento di \textit{Specifica Tecnica} conterrà i seguenti argomenti:
	\begin{itemize}
		\item \textbf{Diagrammi UML:} diagrammi dei packages, diagrammi delle classi, diagrammi di sequenza e diagrammi di attività.
		\item \textbf{Design patterns:} i \textit{Progettisti} devono fornire una descrizione dei design patterns adottati nella definizione dell'architettura. Questa descrizione dovrà essere accompagnata da un diagramma UML, che ne esemplifichi il funzionamento, e dalle motivazioni che hanno portato all'adozione di tale pattern;
		\item \textbf{Tracciamento delle componenti:} ogni componente dovrà essere tracciato ed associato ad almeno un requisito. Così facendo, si potrà essere certi che tutti i requisiti vengano soddisfatti, e che ogni componente presente nell'architettura soddisfi almeno un requisito. Tale tracciamento dovrà essere effettuato tramite \textit{NetBreakDB}, che si occupa di generare in modo automatico le relative tabelle.
		\item \textbf{Test d'integrazione:} i \textit{Progettisti} devono definire delle strategie di verifica per poter dimostrare la corretta integrazione tra le varie componenti definite.
	\end{itemize}
	
	\paragraph{Definizione di Prodotto}
	I \textit{\Progs}, a partire dal documento di \textit{Specifica Tecnica}, hanno il compito di produrre la \textit{Definizione di Prodotto}, nella quale viene descritta la progettazione di dettaglio del sistema.\\
	Lo scopo di questo documento è quello di definire dettagliatamente ogni singola unità
	di cui è composto il sistema, in modo da semplificare l’attività di codifica, e allo stesso
	tempo, non fornire alcun grado di libertà ai \textit{\Progrs}.\\
	Parallelamente alla progettazione di dettaglio dei componenti software, si dovranno prevedere e progettare i relativi test di unità.\\
	Il documento di \textit{Definizio di Prodotto} conterrà i seguenti argomenti:
	\begin{itemize}
		\item \textbf{Diagrammi UML:} diagrammi dei packages, diagrammi delle classi, diagrammi di sequenza.
		\item \textbf{Definizione delle classi:} ogni classe progettata viene descritta più in dettaglio, fornendo una descrizione più approfondita dello scopo, delle sue funzionalità e del suo funzionamento. Per ogni classe, dovranno essere anche definiti i vari metodi e attributi che la caratterizzano;
		\item \textbf{Tracciamento delle classi:} ogni classe deve essere tracciata ed associata ad almeno un requisito, così facendo è possibile avere la certezza che tutti i requisiti vengano soddisfatti, e che ogni classe presente nell'architettura soddisfi almeno un requisito.\\
		Questo tracciamento dev'essere effettuato tramite \textit{DocumentsDB}, che si occupa di generare in modo automatico le tabelle di tracciamento.
		\item \textbf{Test di unità:} i \textit{\Progs} devono definire le strategie di verifica delle varie classi, in modo che durante l'attività di codifica sia possibile verificare che la classe si comporti in modo corretto.
	\end{itemize}
	
	\paragraph{Norme progettuali}
	\subparagraph{Tecniche di modularizzazione}
	Prima di iniziare l'attività di progettazione, i \textit{\Progs} hanno bisogno di definire delle tecniche di modularizzazione, affinchè l'architettura software progettata sia di elevata qualità.\\
	Essi devono necessariamente adottare delle tecniche che consentano la scomposizione del sistema in moduli e definire una descrizione precisa della struttura modulare e delle relazioni che esistono tra i singoli moduli.\\
	La modularizzazione porterà i seguenti vantaggi:
	\begin{itemize}
		\item Semplificazione della verifica della correttezza semantica e nella correzione di errori;
		\item Leggibilità del codice;
		\item Riusabilità del software;
		\item Possibilità di realizzazione di prototipi;
		\item Semplificazione dell'attività di manutenzione.
	\end{itemize}
	
	\subsubsection{Codifica}
	L'attività di Codifica si prefigge la realizzazione pratica del codice dell'applicativo richiesto. I \textit{\Progrs} sono i principali responsabili di questa attività, i quali, tenendo conto degli stadi precedenti e seguendo nel dettaglio quanto precedentemente stilato, realizzano un prodotto efficace. Ogni \textit{\Progr} deve essere, come requisito prioritario, strettamente ancorato alle linee guida stabilite.
	
		\paragraph{Best practices}
		L'attività di Codifica richiede di essere molto scrupolosi, attenendosi alle linee guida definite all'interno del presente documento. Esse compongono gli standard di codifica per la realizzazione del progetto. I seguenti standard hanno il compito di:
		\begin{itemize}
			\item Fornire uno strumento per garantire una più agevole cooperazione tra i diversi sviluppatori;
			\item Favorire la creazione di un codice valido e ben formato;
			\item Favorire la realizzazione di un software di qualità;
			\item Rendere più agevole la lettura e la modifica del codice a tutti i componenti del team.
		\end{itemize}
		Le regole stilistiche del codice riguardano le parti non prettamente implementative, ma di ausilio ai \textit{Programmatori}, al fine di permettere una miglior cooperazione e comprensione del codice stesso. Le tecniche di scrittura del codice, descritte nel presente documento, sono suddivise in tre categorie: Impostazione, Nomenclatura e Commenti.
	
			\subparagraph{Impostazione}
			L'impostazione stilistica del documento consente una più semplice comprensione e lettura del codice, e pertanto è importante fornire una logica comune. Di seguito sono elencate le principali convenzioni relative alla formattazione del codice: 
			\begin{itemize}
				\item I blocchi di codice vanno indentati tramite i rientri standard, evitando spaziature singole. Ogni porzione di codice interna ad un altra dev'essere rigorosamente indentata a sua volta;
				\item Utilizzare gli spazi per separare gli operatori, ove possibile. Aggiungere spazi che aumentino la leggibilità del codice in tutte le situazioni dove ciò sia permesso, a patto che il funzionamento non venga compromesso;
				\item Dividere il codice in più moduli e non utilizzare un unico file di ingenti dimensioni e di difficile modifica. Ciò è un requisito fondamentale per un corretto e miglior utilizzo dei sistemi di versionamento.
			\end{itemize}
	
			\subparagraph{Nomenclatura}
			Di seguito, sono elencati gli standard di nomenclatura utilizzati per il codice che verrà prodotto. Tale aspetto è fondamentale per una miglior comprensione del codice.
			\begin{itemize}
			\item Assegnare nomi univoci ai costrutti, evitando possibili ambiguità dovute a nomi similari. I costrutti devono avere nomi quanto più significativi possibili, anche se utilizzati in porzioni marginali. Le variabili formate da un solo carattere sono, invece, consentite qualora si tratti di iterazioni circoscritte;
			\item Scegliere nomi descrittivi per file e cartelle. \MakeUppercase{è} importante che indichino, nel modo più accurato possibile, il proprio contenuto;
			\item Accertarsi che ogni parola utilizzata sia scritta nella corretta forma della lingua a cui appartiene;
			\item Nella scrittura di nomi composti utilizzare la forma \textit{Camel Case\ped{G}}, ottenuta scrivendo con iniziale minuscola la prima parola di un nome composto, seguito dalle successive parole, ognuna con iniziale sempre maiuscola;
			\item Nella descrizione di funzioni, utilizzare come prima voce il verbo dell'azione svolta, seguito dal nome dell'azione o dell'oggetto su cui viene eseguita;
			\item Accertarsi che non vi sia ambiguità nelle abbreviazioni utilizzate, avendo cura di verificare che non possano venir impiegate due abbreviazioni simili per funzionalità differenti.
			\end{itemize}
					
		\subparagraph{Commenti}
		All'interno del codice sorgente è di fondamentale importanza la presenza di commenti, al fine di facilitare la comprensione del flusso logico.
		Di seguito, vengono elencati alcuni metodi di inserimento dei commenti:
		\begin{itemize}
			\item Produrre e/o mantenere aggiornata un'intestazione per ciascun file. Essa deve sempre avere una breve descrizione di ciò che è incluso nello specifico file;
			\item Indicare pre e post-condizione di ogni funzionalità di rilievo, in modo da favorire un riutilizzo di eventuale codice già scritto. \MakeUppercase{è}, inoltre, consigliato introdurre una breve descrizione concisa riguardante la funzionalità;
			\item Ogni qualvolta si modifica il codice, occorre mantenere sempre aggiornati i relativi commenti;
			\item All'inizio di ogni funzione o procedura, è utile fornire commenti che ne indichino le limitazioni, i presupposti e lo scopo;
			\item Evitare l'aggiunta di commenti alla fine di una riga di codice, poichè può rendere più difficoltosa la lettura del codice. Tuttavia, questo tipo di commento è accettabile in caso di annotazione di dichiarazioni di variabili;
			\item Evitare commenti confusi come righe intere di asterischi, bensì si raccomanda di utilizzare spazi vuoti per separare i commenti dal codice;
			\item Quando si scrivono commenti, utilizzare frasi di senso compiuto. La funzione dei commenti consiste nel chiarire il significato del codice senza aggiungere alcun tipo di ambiguità;
			\item Inserire commenti in fase di scrittura del codice, in quanto ciò potrebbe non essere possibile in un secondo momento;
			\item Evitare commenti superflui o inappropriati, come annotazioni umoristiche;
			\item Commentare tutto ciò che si ritiene poco chiaro all'interno del codice;
			\item Utilizzare sempre i commenti nel caso di codice relativo a correzioni di errori e a potenziali soluzioni, al fine di evitare problemi ricorrenti e reintrodurre errori già corretti in precedenza;
			\item Aggiungere commenti al codice costituito da cicli e diramazioni logiche, per facilitarne la comprensione;
			\item Creare commenti adottando uno stile uniforme e una struttura e una punteggiatura coerenti.
		\end{itemize}
	
	\subsubsection{Testing}
	\paragraph{Descrizione}
	L'attività di testing ha lo scopo di verificare che ogni funzione codificata rispetti la progettazione e che i risultati in output per ciascuna componente soddisfi le aspettative.
	
	\paragraph{Strategie generali}
	Le strategie principali che verranno intraprese in fase di testing sono le seguenti:
	
	\begin{itemize}
		\item Ridurre tempi e costi dell'attività di testing aumentando il livello di automatizzazione quanto possibile
		\item Seguire le linee guida del progettista riguardo i risultati attesi e lo svolgersi dei test
	\end{itemize}
	
	\paragraph{Obiettivi attesi}
	Il team si impegna per soddisfare i seguenti obbiettivi:
	
	\begin{itemize}
		\item I test di integrazione devono essere quanto più possibile concordi alle componenti integrate all'interno del software
		\item I requisiti obbligatori pattuiti in fase di definizione contrattuale verranno testati estensivamente.
	\end{itemize}

	\subsubsection{Strumenti utilizzati}
	\paragraph{TeXstudio}
		In seguito a un breve periodo di test, la scelta del team per l'editor di documenti \textit{\LaTeX\ped{G}} è ricaduta su \textit{\textbf{TeXstudio\ped{G}}}. Esso è un noto fork di un altrettanto famoso editor, \textit{TexMaker\ped{G}}, ma con la peculiarità di essere completamente gratuito, open source e disponibile per i più diffusi sistemi operativi. L'editor consente un semplice utilizzo dei file \textit{.tex} e una comoda gestione di inclusione e accorpamento di documenti. Tale caratteristica viene incontro alle esigenze del gruppo per via della predisposizione alla modularità dei documenti richiesti dal progetto. L'editor scelto, inoltre, possiede numerose features rilevanti, quali l'autocompletamento e il controllo della sintassi e dell'ortografia, e l'anteprima in PDF incorporata. Per questi motivi, TeXstudio sarà il principale strumento tramite cui verranno redatti tutti i documenti.
	
		\paragraph{Astah}
		Per la modellazione dei diagrammi dei casi d'uso in linguaggio UML, la scelta è ricaduta su \textit{\textbf{Astah\ped{G}}}, editor multi-piattaforma e gratuito. Il suo punto forte rispetto ad altre soluzioni è la semplicità di utilizzo. Tra le funzionalità interessanti ai fini del team, c'è la possibilità di esportare i diagrammi creati direttamente in un file immagine con estensione \textit{.png}. Il software Astah sarà utilizzato per la creazione di tutti i diagrammi necessari richiesti durante le attività di Progettazione ed \AdR.
		
		\paragraph{Smartsheet}
		Per la realizzazione dei \textit{diagrammi di Gantt\ped{G}} richiesti dall'attività di Pianificazione, si è scelto di utilizzare lo strumento \textit{\textbf{Smartsheet\ped{G}}}, il quale consente di creare diagrammi secondo le attività pianificate per uno specifico periodo di tempo, indicato dall'utente.
		Smartsheet è uno strumento gratuito e intuitivo, fruibile tramite interfaccia web. Le funzioni offerte sono ampie e immediate, ed è possibile personalizzare i diagrammi con una formattazione condizionale per evidenziare attività, punti critici o \textit{milestone\ped{G}}. Smartsheet semplifica notevolmente la collaborazione in tempo reale e permette a diverse persone di rimanere aggiornate in tempo reale sugli sviluppi del progetto. \MakeUppercase{è} risultato molto utile per la creazione dei grafici presenti nel documento \textsc{PianoDiProgetto 2\_0\_0.pdf}.
		
		\paragraph{NetBreakDB}
		Per il tracciamento dei documenti, preponderante specialmente nei periodi di \AdR e Progettazione, la scelta del team è ricaduta sull'utilizzo di uno strumento dedicato, per permettere una più agevole gestione. Il software utilizzato è un fork di un progetto esistente, \textit{PragmaDB}, creato da un gruppo di studenti degli scorsi anni e modificato secondo le esigenze del team \textit{\gruppo}. NetBreakDB permette la gestione di casi d'uso, attori, componenti, metriche, fonti e il calcolo automatizzato dell'indice Gulpease, effettuato direttamente sui documenti contenuti in una repository di \textit{GitHub\ped{G}}. I risultati dell'indice vengono visualizzati in una tabella secondo questo ordine:
		\begin{center}
			Nome Documento | Valore Indice Gulpease | Esito
		\end{center}
		L'applicativo è in grado di generare codice \LaTeX\ a seguito di un'indicizzazione dei requisiti all'interno del database. La possibilità di esportare le voci necessarie al tracciamento in maniera automatizzata rende il lavoro più uniforme, ordinato e semplificato.
		
		\paragraph{Docker}
		\textbf{\textit{Docker\ped{G}}} è un progetto open source per gli sviluppatori e gli amministratori di sistema, che permette di automatizzare la distribuzione di applicazioni all'interno di contenitori software su macchine virtuali e nel cloud. Il principio su cui si basa, promuove lo sviluppo agile, il quale garantisce maggiore efficienza e flessibilità alle iniziative cloud e di modernizzazione delle applicazioni. Docker è un contenitore che permette di integrare applicazioni o parti di esse, e fornisce tutto il necessario per l'esecuzione ed installazione su un server. Ciò garantisce che il software funzionerà sempre allo stesso modo, indipendentemente dall'ambiente in cui verrà eseguito. Questo strumento permette di creare software più performanti, accelerare lo sviluppo ed eliminare possibili problemi dovuti all'esecuzione su macchine diverse. Il contenitore isola l'applicazione dai livelli sottostanti della struttura, in modo da garantire protezione all'applicazione. 
			
		\paragraph{SwaggerHub}
		\textbf{\textit{SwaggerHub\ped{G}}} è uno strumento software online, sviluppato da SmartBear Software, per la produzione di API condivise tra un massimo di 25 sviluppatori. Esso si prefigge di accompagnare lo sviluppo durante l'intero ciclo di vita del software, in modo incrementale. SwaggerHub supporta numerosi differenti linguaggi di programmazione e permette l'integrazione con servizi di controllo della versione come GitHub. 
		Una caratteristica rilevante è la creazione in tempo reale della documentazione relativa all'API che si sta sviluppando. La manutenzione è facile ed è possibile distribuire l'API prodotta direttamente su piattaforme come \textit{AWS\ped{G}} e \textit{Microsoft Azure\ped{G}}.
		
		\paragraph{Sublime Text}
		\textbf{\textit{Sublime Text\ped{G}}} è l'editor scelto dal team per la scrittura del codice. Esso è caratterizzato da un'interfaccia intuitiva, un'ampia gamma di features innovative e ottime prestazioni. Questo editor è compatibile con diversi sistemi operativi ed è personalizzabile in molti aspetti. Una caratteristica peculiare è il \textit{simultaneous editing}, ovvero la possibilità di modificare più porzioni di codice contemporaneamente, premendo il tasto Shift e posizionando il cursore nel punto esatto in cui si intende effettuare delle modifiche. Le caratteristiche di personalizzazione riguardano i temi, l'interfaccia e le modalità di visione del codice. Inoltre, l'auto-completamento del codice in base al linguaggio scelto e il salvataggio automatico sono altre funzionalità che hanno portato il gruppo a scegliere questo strumento.
