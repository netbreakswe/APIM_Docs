\newpage
\section{Ricerca e visualizzazione API}

\subsection{Ricerca}
La funzionalità di ricerca è disponibile per qualsiasi categoria di utente. Essa permette, in base ad una parola chiave, di visualizzare le API relative contenute nella piattaforma. In seguito a ricerca, effettuata scrivendo sull'apposita barra la parola chiave desiderata, si può accedere ad un elenco dei risultati come mostrato.

\label{Risultati ricerca}
\begin{figure}[H]
	\centering
	\fbox{\includegraphics[scale=0.31]{img/APIM_ricerca.JPG}}
	\caption{Risultati ricerca}
\end{figure}

Selezionando un API dall'elenco dei risultati, è possibile visualizzare i dati nel dettaglio, con relative specifiche. 

\label{Visualizzazione API}
\begin{figure}[H]
	\centering
	\fbox{\includegraphics[scale=0.28]{img/APIM_dettagliApi.png}}
	\caption{Visualizzazione API}
\end{figure}

Ciascuna API presente nella piattaforma è caratterizzata da una policy di vendita, descritta all'interno di ciascun prodotto. E' possibile visualizzare i dettagli della policy clickando sull'apposito link nella schermata.

\label{Visualizza policy API}
\begin{figure}[H]
	\centering
	\fbox{\includegraphics[scale=0.31]{img/APIM_policy.JPG}}
	\caption{Visualizza policy API}
\end{figure}

Qualora si decidesse di effettuare l'acquisto, nella schermata è presente un pulsante Acquista che porterà alla schermata relativa al completamento della transazione se l'utente è loggato, altrimenti chiederà di effettuare il login o la registrazione. L'utente riceverà un API key con la quale potrà utilizzare il servizio acquistato. Nella pagina relativa al completamento della transizione, l'utente potrà scegliere l'importo da acquistare a seconda della policy dell'API che intende acquistare. 

\label{Acquisto API}
\begin{figure}[H]
	\centering
	\fbox{\includegraphics[scale=0.60]{img/APIM_acquisto.PNG}}
	\caption{Acquisto API}
\end{figure}