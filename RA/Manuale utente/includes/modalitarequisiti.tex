\newpage
\section{Modalità di utilizzo e requisiti}

\subsection{Categorie di utenza}
Sono presenti due categorie di utilizzo per l'utente. Esse permettono di accedere in modo differente alle varie aree del sito. Riassumiamo nell'elenco sottostante le categorie presentate:

\begin{itemize}
	\item \textbf{Utente non autenticato}: rappresenta l'utente che visita il sito per la prima volta oppure quando non ha effettuato il login nella piattaforma. All'utente non autenticato è permessa la visualizzazione dei prodotti inseriti nella piattaforma, così come la ricerca. Essi però non hanno accesso alle aree e funzionalità riservate;
	\item \textbf{Utente autenticato}: un utente appartenente a questa categoria possiede le funzionalità di ricerca come per l'utenza non autenticata. Inoltre, ha a disposizione il proprio profilo utente ed è abilitato all'acquisto dei servizi offerti nel sito, con l'accesso alle relative funzionalità statistiche.
	\item \textbf{Utente autenticato (sviluppatore)}: questa categoria di utente è una sottoclasse dell'utente autenticato, ma risulta abilitato all'inserimento e gestione dei propri servizi per la vendita ad altri utenti della piattaforma. 
\end{itemize}

\subsection{Requisiti di sistema}
APIM, essendo un applicazione web-based, richiede l'accesso ad internet per poter essere utilizzato. La compatibilità browser è riassumibile nell'elenco sottostante, fermo restando che seppur non garantita potrebbe esserci compatibilità con versioni antecedenti a quelle indicate. L'utente per usufruire della piattaforma non deve installare alcun componente, ma deve attivare javascript sul browser web scelto per usufruire dei contenuti della web app.
L'hardware del dispositivo non è oggetto di vincolo all'utilizzo.

\subsubsection{Browser supportati}
Il software API Market supporta i \textit{browser\ped{G}} a partire dalle versioni: Google Chrome 56.0, Mozilla Firefox 51.0, Safari 10.0, Microsoft Edge 38.0, Android browser 5.1 e Safari per iOS 10.